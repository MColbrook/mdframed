%Documenation of the package mdframed
%%$Id: mdframed-doc-en.tex 232 2011-11-26 21:08:16Z marco $
\setcounter{errorcontextlines}{999}
\documentclass[parskip=false,english,11pt]{ltxmdf}
\svnInfo $Id: mdframed-doc-en.tex 232 2011-11-26 21:08:16Z marco $
\usepackage{babel}
\usepackage[utf8]{inputenc}
\usepackage[T1]{fontenc}
\usepackage{lipsum}
\usepackage[T1,altbullet]{lucidabr}
\usepackage[scaled=0.82]{beramono}  

\usepackage[framemethod=TikZ]{mdframed}

\title{The \mdname package\footnote{Extending the package \texttt{framed.sty}}
\subtitle{auto-split frame environment}
\author{\href{mailto:marco.daniel@mada-nada.de}{Marco Daniel}%
       \footnote{With thanks to Heiko Oberdiek, Rolf Niepraschk,
                 Martin Scharrer and Herbert Voss.\newline Sorry for bad English.}
          \href{mailto:marco.daniel@mada-nada.de}{Elke Schubert}}}
\version{\mdversion}
\date{\svnToday}
\introduction{The standard methods for framing text (\mdcommand{fbox} or \mdcommand{fcolorbox}) require you to handle page breaks by hand, meaning that you have to split the \mdcommand{fbox} into two. The present package defines the environment \mdname which automatically deals with pagebreaks in framed text.\par
By defining new environments the user may choose between several individual designs.%
\par\kern.5\baselineskip\noindent
Linked files:\quad \begin{tabular}[t]{ll}
            \url{mdframed-example-default.pdf} &
            \url{mdframed-example-tikz.pdf} \\
            \url{mdframed-example-pstricks.pdf}  &
            \url{mdframed-example-texsx.pdf}
            \end{tabular}
\par\kern.5\baselineskip\noindent
FYI:\quad\parbox[t]{.8\linewidth}{I create a repository for \mdname on \href{https://github.com/marcodaniel/mdframed}{github} where you can \href{https://github.com/marcodaniel/mdframed}{download} the current development status.}
}




\begin{document}
\maketitle


\section{Motivation}
Many users wish to (further) emphasize lemmata, definitions, proofs, etc. The package \mdname allows you to create environments with breakable frames. I think an example is the best way to demonstrate its properties.


%\newtheorem{mdtheorem}{Theorem}[section]
\newmdtheoremenv[outerlinewidth=2,leftmargin=40,rightmargin=40,%
    backgroundcolor=yellow,outerlinecolor=blue,innertopmargin=0pt,%
    splittopskip=\topskip,skipbelow=\baselineskip,%
    skipabove=\baselineskip,ntheorem,roundcorner=5pt]{theorem}{Theorem}[section]

\begin{theorem}[Pythagorean theorem]
 In any right triangle, the area of the square whose side is the hypotenuse
 is equal to the sum of the areas of the squares whose sides are the two legs.
 
\[ a^2+b^2=c^2 \]
\end{theorem}

The frame was defined with the following settings.

\begin{mdexample}
\newmdtheoremenv[outerlinewidth=2,leftmargin=40,%
    rightmargin=40,backgroundcolor=yellow,%
    outerlinecolor=blue,innertopmargin=0pt,%
    splittopskip=\topskip,skipbelow=\baselineskip,%
    skipabove=\baselineskip,ntheorem]{theorem}%
    {Theorem}[section]
\begin{theorem}[Pythagorean theorem]
...
\end{theorem}
\end{mdexample}


\section{Syntax}\label{sec:syntax}
The package itself loads the packages \mdpack{kvoptions}, \mdpack{etoolbox} and \mdpack{color}.
By setting the correct options \mdpack{mdframed} will load \mdpack{xcolor}, \mdpack{tikz} or \mdpack{pstricks}.

Load the package as usual:
\begin{mdexample}
 \usepackage[<GLOBAL OPTIONS>]{mdframed}
\end{mdexample}

The package defines only one environment with the following syntax:
\begin{mdexample}
 \begin{mdframed}[<LOCAL OPTIONS>]
    <CONTENT>
 \end{mdframed}
\end{mdexample}
To create own environments with \mdpack{mdframed} see section \ref{sec:cmds}.

\minisec{Autodetecting floats}
I added a detection of \mdpack{float} or \mdpack{minipage} environments. If you use \mdname in such an environment \mdname will use the option \mdoption{nobreak} automaticly. 

\minisec{Twoside-mode}
When you are using \mdname inside \mdpack{twoside} mode you  can set the option \mdoption{innermargin}
and \mdoption{outermargin} (see section \ref{optlength}).

%%%%%%%%%%%%%%%%%%%%%%%
\section{Commands}\label{sec:cmds}
The following commands should countenance your by the handling with \mdpack{mdframed}

\mdDescribeCmd[]{newmdenv} The command has the following syntax:
\begin{mdexample}
\newmdenv[<MDFRAMED OPTIONS>]{Name of the environment}
\end{mdexample}
In this way you can simply use:
\begin{mdexample}
 \newmdenv[linecolor=red,frametitle=Infobox]{infobox}
 ...
 \begin{infobox}[backgroundcolor=yellow]
  foo   foo   foo   foo   foo   foo 
 \end{infobox}
\end{mdexample}
\mdDescribeCmd[]{renewmdenv} By using this command you can redefine environments which are created by \mdcommand{newmdenv}.

\mdDescribeCmd[]{newmdtheoremenv} 
Since the package is often used to highlight theorem environments, I have created a command\footnote{Thanks to Martin Scharrer and Enrico Gregorio:\newline \href{http://tex.stackexchange.com/questions/26298/own-command-to-create-new-environment}{Own command to create new environment}} to simplify this process. The command has the following syntax:

\begin{mdexample}[deletekeywords={mdframed}]
\newmdtheoremenv[<mdframed-options>]{<envname>}%
                [<numberedlike>]{<caption>}[<within>]
\end{mdexample}
The last four arguments are equivalent to the command \mdcommand{newtheorem}. Only the first optional argument is able to pass \mdpack{mdframed}-options. A simple example is:
\begin{mdexample}
\theoremstyle{<some style>}
\newmdtheoremenv[linecolor=blue]{lemma}%
                {Lemma}[section]
... 
\begin{lemma}[Some title]
  foo   foo   foo   foo   foo   foo
\end{lemma}
\end{mdexample}
So far there is no \mdcommand{renewmdtheoremenv}!


\mdDescribeMacro{\textbackslash mdfsetup}To set the options 
you can use the optional argument of \mdcommand{usepackage} or
you can use the command \mdcommand{mdfsetup} which is not limited to the preamble. Inside a group the settings work only local.

\mdDescribeMacro{\textbackslash mdfdefinestyle} \mdcommand{mdfdefinestyle} allows the user to define different styles and use as an option of \mdpack{mdframed} via \mdoption{style}. The option \mdoption{style} is explained in section \ref{genopt}.

Here a small example:
\begin{mdexample}[morekeywords={mystyle}]
\mdfdefinestyle{mystyle}{leftmargin=0pt,%
                         linecolor=blue}
....
\begin{mdframed}[style=mystyle]
foo
\end{mdframed}
\end{mdexample}

\mdDescribeMacro{\textbackslash mdfapptodefinestyle} This commands allows to expand a defined style.\footnote{Thanks to Martin Scharrer and Enrico Gregorio:\newline \href{http://tex.stackexchange.com/questions/34684/argument-of-setkeys}{http://tex.stackexchange.com/questions/34684/argument-of-setkeys}}



%%%%%%%%%%%%%%%%%%%%%%%
\section{Options}
The package provides various options to manipulate frames. In the following section all options are listed. Some internal macros which can be manipulated are not shown in this documentation.
The listed option are divided in global and local options. The global options can not be used inside \mdcommand{mdfsetup}.

\subsection{Global Options}\label{globopt}
The following options are only global options.

\mdDescribeMacro[none]{xcolor}
By setting this key, the package \mdpack{xcolor} will
be loaded with the given value(s). Without any value \mbox{\mdname} loads
the package \mdpack{color} without any options. If the package \mdpack{xcolor} is already loaded the given option will be ignored. I recommend to load \mdpack{xcolor} before \mdname.

\mdDescribeMacro[default]{framemethod} 
With this key you can change the way frames are drawn. You can decide whether the frame is drawn with
\begin{enumerate}
 \item \LaTeX-commands {\small\mdcommand{hrule}, \mdcommand{vrule}, \mdcommand{rule}},
 \item \mdpack{TikZ} {\small(the package \mdpack{TikZ} will be loaded)} or
 \item \mdpack{PSTricks} {\small(the package \mdpack{pstricks} will be loaded)}.
\end{enumerate}
The option \mdoption{framemethod} requires a string. Allowed combinations are listed in the following table.
\begin{table}[!ht]
\centering
\renewcommand*\arraystretch{1.3}
\caption{Allowed keys for \mdoption{framemethod}}
\begin{tabular}{@{}>{\small\bfseries}l>{\small\ttfamily}l@{}}
\toprule
\multicolumn{1}{@{}l}{\textbf{Method}} & \multicolumn{1}{l}{Allowed keys} \\\midrule
\LaTeX-commands   & default, tex, latex, none, 0  \\
\mdpack{TikZ}     & tikz, pgf, 1                  \\
\mdpack{PSTricks} & pstricks, ps, postscript, 2   \\\bottomrule
\end{tabular}
\end{table}

\noindent{\small\textbf{FYI}\qquad It is independently whether the \texttt{method} is written with no, one or more capital letter.}

\NOTE  The manipulation of the frames depends on the option \mdoption{framemethod}. For further information see below.

\subsection{Global and Local Options}
The options listed below can be set globally or locally and they are not limited to the preamble.

\subsubsection{Options with lengths}\label{optlength}

In figure \eqref{fig:laengen} you can see the adjustable lengths which will be described below. 
All lengths accept two kinds of input. The first one is a length (e.g. 2pt) and
the second one is a number (e.g. 2) which will be multiplied by \mdoption{1 defaultunit}.
The figure shows three different colored frames. Only \mdoption{framemethod=tikz} is able to draw such triple lines.

I know that the predefined lengths are not well prepared. Maybe I will change it later.

\mdDescribeMacro[pt]{defaultunit} see the sentence above.

\begin{center}
% \includegraphics{laengen}
\begin{tikzpicture}[scale=0.9]
     %Declaration
     \newlength{\linew}
     \setlength{\linew}{5pt}
     \tikzstyle{every node}=[font=\bfseries\large\sffamily]
     \coordinate (innersep) at (3.2,2.25);
     \coordinate (outersep) at (3.2,2);
     %Center
     \node[fill=green!10,draw=black,framed,rounded corners,minimum width=3cm,minimum height=1.5cm]%
           (CenterNode) {Contents};
     
     %frames
     \draw[draw=red!70!black,line width=\linew] %
               ($(CenterNode)+(innersep)$) %
               rectangle ($(CenterNode)-(innersep)$);
     \draw[draw=blue!70!black,line width=\linew] %
               ($(CenterNode)+(innersep)+(\linew,\linew)$)%
               rectangle ($(CenterNode)-(innersep)+(-\linew,-\linew)$);
     \draw[draw=yellow!70!black,line width=\linew]
               ($(CenterNode)+(innersep)+(2\linew,2\linew)$)%
               rectangle ($(CenterNode)-(innersep)+(-2\linew,-2\linew)$);
     \draw[draw=black,line width=1] ($(CenterNode)+(innersep)+(outersep)$)%
               rectangle ($(CenterNode)-(innersep)-(outersep)$);
     
     
     \tikzstyle{every node}=[font=\small\ttfamily]
     %%innertopmargin
     \draw[draw=black,triangle 45-triangle 45]let \p1=(innersep) in
           (CenterNode.north)  -- ($(CenterNode)+(0,\y1)-(0,.5\linew)$) ;
     \path let \p1=(innersep) in
           (CenterNode.north)  edge node[align=center,framed,fill=white,draw=white]{innertopmargin} 
           ($(CenterNode)+(0,\y1)-(0,.5\linew)$) ;
     %%innerbottommargin
     \draw[draw=black,triangle 45-triangle 45]let \p1=(innersep) in
           (CenterNode.south)  -- ($(CenterNode)-(0,\y1)+(0,.5\linew)$) ;
     \path let \p1=(innersep) in
           (CenterNode.south)  edge node[align=center,framed,fill=white,draw=white]{innerbottommargin} 
           ($(CenterNode)-(0,\y1)+(0,.5\linew)$) ;
     
     %%innerleftmargin
     \draw[draw=black,triangle 45-triangle 45]let \p1=(innersep) in
           (CenterNode.west)  -- ($(CenterNode)-(\x1,0)+(.5\linew,0)$) ;
     \path let \p1=(innersep) in
           (CenterNode.west)  edge 
           node[align=center,rotate=90,framed,fill=white,draw=white]{innerleftmargin}  
           ($(CenterNode)-(\x1,0)+(.5\linew,0)$) ;
     %%innerrightmargin
     \draw[draw=black,triangle 45-triangle 45]let \p1=(innersep) in
           (CenterNode.east)  -- ($(CenterNode)+(\x1,0)-(.5\linew,0)$) ;
     \path let \p1=(innersep) in
           (CenterNode.east)  edge
           node[align=center,rotate=90,framed,fill=white,draw=white]{innerrightmargin}  
           ($(CenterNode)+(\x1,0)-(.5\linew,0)$) ;
     
     
     %%leftmargin
     \draw[draw=black,triangle 45-triangle 45]let \p1=(innersep) , \p2=(outersep) in
           ($(CenterNode)-(\x1,0)-(2.5\linew,0)$)  -- ($(CenterNode)-(\x1,0)-(\x2,0)$) ;
     \path let \p1=(innersep) , \p2=(outersep) in
           ($(CenterNode)-(\x1,0)-(2.5\linew,0)$)  edge 
           node[align=center,rotate=0,framed,fill=white,draw=white]{leftmargin}  
           ($(CenterNode)-(\x1,0)-(\x2,0)$) ;
     %%rightmargin
     \draw[draw=black,triangle 45-triangle 45]let \p1=(innersep) , \p2=(outersep) in
           ($(CenterNode)+(\x1,0)+(2.5\linew,0)$)  -- ($(CenterNode)+(\x1,0)+(\x2,0)$) ;
     \path let \p1=(innersep) , \p2=(outersep) in
           ($(CenterNode)+(\x1,0)+(2.5\linew,0)$)  edge 
           node[align=center,rotate=0,framed,fill=white,draw=white]{rightmargin}  
           ($(CenterNode)+(\x1,0)+(\x2,0)$)  ;
     
     %%skipabove
     \draw[draw=black,triangle 45-triangle 45]let \p1=(innersep) , \p2=(outersep) in
           ($(CenterNode)+(0,\y1)+(0,2.5\linew)$)  -- ($(CenterNode)+(0,\y1)+(0,\y2)$) ;
     \path let \p1=(innersep) , \p2=(outersep) in
           ($(CenterNode)+(0,\y1)+(0,2.5\linew)$)  edge 
           node[align=center,rotate=0,framed,fill=white,draw=white]{skipabove}  
           ($(CenterNode)+(0,\y1)+(0,\y2)$) ;
     %%skipbelow
     \draw[draw=black,triangle 45-triangle 45]let \p1=(innersep) , \p2=(outersep) in
           ($(CenterNode)-(0,\y1)-(0,2.5\linew)$)  -- ($(CenterNode)-(0,\y1)-(0,\y2)$) ;
     \path let \p1=(innersep) , \p2=(outersep) in
           ($(CenterNode)-(0,\y1)-(0,2.5\linew)$)  edge 
           node[align=center,rotate=0,framed,fill=white,draw=white]{skipbelow}  
           ($(CenterNode)-(0,\y1)-(0,\y2)$) ;
\end{tikzpicture}

 \captionof{figure}{adjustable lengths of \mdname}
 \label{fig:laengen}
\end{center}

\mdDescribeMacro[0pt]{skipabove} Sets an additional skip above the frame.
\mdDescribeMacro[0pt]{skipbelow} Sets an additional skip below the frame.
\mdDescribeMacro[]{margin} This option is not longer supported. Use \mdoption{leftmargin}
                           and \mdoption{rightmargin} instead.
\mdDescribeMacro[0pt]{leftmargin} Sets the length of the left margin of the environment.
\mdDescribeMacro[0pt]{rightmargin} Sets the length of the right margin of the environment.

\mdDescribeMacro[10pt]{innerleftmargin} Sets the length of the inner left margin of the environment.
\mdDescribeMacro[10pt]{innerrightmargin} Sets the length of the inner right margin of the environment.
\mdDescribeMacro[.4\textbackslash baselineskip]{innertopmargin} Sets the length of the inner top margin of the environment.
\mdDescribeMacro[.4\textbackslash baselineskip]{innerbottommargin} Sets the length of the inner bottom margin of the environment.

\paragraph*{The following lengths are not shown in figure \eqref{fig:laengen}.}

\mbox{}
\mdDescribeMacro[0pt]{userdefinedwidth} Sets the width of the whole \mdname environment. The width represent the width including the line width and the inner margins. The outer margins will be ignored.

\mdDescribeMacro{outermargin} Sets the length of the outer margin. This option is only available in \texttt{twoside}-mode.
\mdDescribeMacro{innermargin} Sets the length of the inner margin. This option is only available in \texttt{twoside}-mode.


\mdDescribeMacro[0pt]{splittopskip} Sets the length of the skip above the split part of the environment.
\mdDescribeMacro[0pt]{splitbottomskip} Sets the length of the skip below the split part of the environment.
\mdDescribeMacro[0.4pt]{linewidth} Sets the width of the line around the environment.
\mdDescribeMacro[0pt]{roundcorner} Sets the size of the radius of the corners of the frames.
                                   \limitation[TikZ or PSTricks]
\mdDescribeMacro[0pt]{innerlinewidth} Sets the width of the inner line around the environment.
                                     \limitation
\mdDescribeMacro[0pt]{outerlinewidth} Sets the width of the outer line around the environment.
                                      \limitation
\mdDescribeMacro[linewidth]{middlelinewidth} Sets the width of the middle line around the environment.\limitation


\subsubsection{Colored Options}
\mdDescribeMacro[black]{linecolor} Sets the color of the line around the environment.
\mdDescribeMacro[white]{backgroundcolor} Sets the color of the background of the environment.
\mdDescribeMacro[black]{fontcolor} Sets the color of the contents of the environment.
\mdDescribeMacro[linecolor]{innerlinecolor} Sets the color of the inner line around the environment.
                                        \limitation
\mdDescribeMacro[\newline linecolor]{middlelinecolor} Sets the color of the middle line around the environment. \limitation
\mdDescribeMacro[\newline linecolor]{outerlinecolor} Sets the color of the outer line around the environment.\limitation


\subsubsection{General options}\label{genopt}

\mdDescribeMacro[false]{ntheorem} Before setting this boolean key, you have to load the package \mdpack{ntheorem}. 

                                 With this option you set the values \mdcommand{theorempreskipamount}
                                 and\newline \mdcommand{theorempostskipamount} to 0\;pt.
                                 
\mdDescribeMacro[false]{nobreak} Sometimes it is useful to prevent a frame from splitting.  The \mdoption{nobreak} option is used for this purpose. If you activate this option you can enable it by setting \mdoption{nobreak=false}.


\mdDescribeMacro[0pt]{needspace} Sometimes it is useful to set a minimum height before a frame should be splitted. For such cases you can use \mdoption{needspace}. The option requires a length which sets the minimum height before a frame will be splitted.

\mdDescribeMacro{style} If you define a special style with \mdcommand{mdfdefinstyle} you can use the key \mdoption{style} to load the style. \mdname has no predefined styles.

%%\mdDescribeMacro[none]{settings} With version 0.9 of the package \mdname this option was added. This option allows the user to commit some macros. An example is shown in the appendix.

%\mdDescribeMacro[none]{printheight} This key allows the entries \texttt{none}, \texttt{info} or \texttt{warning}. In relation to the key the vertical space at the beginning of the environment \mdname on the current page will be printed in the \texttt{log} file.

\mdDescribeMacro[left]{align} Sometimes it is useful to align the environment itself. For this you have the option \mdoption{align} which can be set to the following strings: \mdoption{left}, \mdoption{right} and \mdoption{center}. The alignments \mdoption{left} or \mdoption{right} depend on the given lengths \mdoption{leftmargin} and \mdoption{rightmargin}. Later I will present an example to demonstrate my bad English explanation.

\mdDescribeMacro[none]{pstrickssetting} With this key you can pass several options to \mdcommand{psset}. For example 
                                        if you want all lines dashed you will have to set 
                                        \mdoption{pstrickssetting=\{linestyle=dashed\}}. It is very important
                                        to put the options of \mdoption{pstrickssetting} in brackets.
                                        \limitation[PSTricks]
\mdDescribeMacro[none]{tikzsetting} With this key you can pass several options to \mdcommand{tikzset}. Some examples are listed in the next section. It is very important
                                        to put the options of \mdoption{tikzsetting} in brackets.
                                        \limitation
                                        
\mdDescribeMacro[none]{apptotikzsetting} With this key you can add several options to \mdoption{tikzsetting}. This key based on the idea of manipulation of predefined keys of \mdname.
The package \mdname defines via \mdcommand{tikzset} the following keys to draw frames.
\begin{itemize}
 \item \mdcommand{tikzset\{mdfbox/.style\}}
 \item \mdcommand{tikzset\{mdfcorners./style\}}
 \item \mdcommand{tikzset\{mdfbackground./style\}}
 \item \mdcommand{tikzset\{mdfinnerline./style\}}
 \item \mdcommand{tikzset\{mdfouterline./style\}}
 \item \mdcommand{tikzset\{mdfmiddleline./style\}}
\end{itemize}
\textbf{Before you change one please have a look at the file \texttt{md-frame-1.mdf} to see the settings.}\limitation

\subsection{Hidden Lines}


\mdDescribeMacro[true]{topline} Draws a line at the top.
\mdDescribeMacro[true]{bottomline} Draws a line at the bottom.
\mdDescribeMacro[true]{leftline} Draws a line on the left.
\mdDescribeMacro[true]{rightline} Draws a line on the right.

\mdDescribeMacro[false]{hidealllines} With this option you can decide whether all lines should be drawn or not. 

\subsection{Frametitle}


\mdDescribeMacro[none]{frametitle} The environment gets a title. To set a title use \mdoption{frametitle=\{The Title of the frame\}} as an option of the environment.
\mdDescribeMacro[\mbox{} \mdcommand{bfseries}\mdcommand{large}]{frametitleformat} Sets the format of the \mdoption{frametitle}.

\subsection{Footnotes}
Inside the environment you can use the command \mdcommand{footnote} as usual. \mdname uses the syntax of environment \mdpack{minipage} with the same counter.

Every footnote text will be collected inside a box and will be displayed at the end of the environment \mdname. 

\mdDescribeMacro[\mbox{} \mdcommand{bigskipamount}]{footnotedistance} The length is the distance between the end of the environement \mdname and the displaying of the \mdcommand{footnoterule}.

\mdDescribeMacro[true]{footnoteinside} The position of the footnotes can be changed with the option \mdoption{footnoteinside}. The footnotes will be displayed at the end of the environment but you can decide whether the output is inside \mdname or after.

\vskip\baselineskip
\noindent\textbf{Note}\qquad  The ouput of the footnotes with the option \mdoption{footnoteinside=false} are not in a splitted frame. I think it isn't useful because the first line of a new page shouldn't be a footnote.

\section{Examples}

I outsource the examples in four files to limit the documentation. The files are
\begin{description}[style=nextline]
\item[\texttt{mdframed-example-default}] Demonstration of examples created with \mdoption{framemethod=default}.
\item[\texttt{mdframed-example-tikz}] Demonstration of examples created with \mdoption{framemethod=TikZ}.
\item[\texttt{mdframed-example-pstricks}] Demonstration of examples created with \mdoption{framemethod=pstricks}.
\item[\texttt{mdframed-example-texsx}] Demonstration of examples like interaction with \mdpack{listings}
\end{description}

The examples are often not equivalent but normally they can be adapted to another method.

\section{Errors, Warnings and Messages}

The package \mdpack{mdframed} provides different errors, warnings and
messages in the \texttt{log}-file. Some \LaTeX-editors like
\TeX Maker or \TeX Studio have a special tab for errors and warnings
but not for messages. So you should look in the \texttt{log-File} itself.

The followings errors and warnings are generated by \mdpack{mdframed}.

\begin{mdexample}
  The package ...  does not exist but
   needed by mdframed
\end{mdexample}        
To avoid this problem you should install the required
packages which are listed in section \ref{sec:syntax}.

\begin{mdexample}
  package option style is depreciated
  use framemethod instead style
\end{mdexample}    
With version 0.9d \mdname changed the meaning of the option \mdoption{style}. The option is used to load a defined style by \mdcommand{mdfdefinestyle}. Instead use \mdoption{framemethod} (see section \ref{globopt}). 

\begin{mdexample}
 Unknown framemethod .... mdframed
\end{mdexample}   
The input string for the option \mdoption{framemethod} is unkown. See section \ref{globopt}.

\begin{mdexample}
  You have not loaded ntheorem yet
\end{mdexample}   
To use the option \mdoption{ntheorem} you have to 
load the package \mdpack{ntheorem}.

\begin{mdexample}
 You have only a width of 3cm
\end{mdexample}   
The package \mdpack{mdframed} calculates the width of the contents
based on the given options. If the width of the contents smaller than
3\,cm you will get this warnings. You should change the 
settings to get a greater width.

\begin{mdexample}
 You got a bad break
 you have to change it manually
 by changing the text, the space
 or something else
\end{mdexample} 
Sometimes you have enough vertical space for the rules and the space
between the rules and the contents but not for the contents itself.
In this situation you will get this warning because the contents
of this box is empty. You have the possibility to change the
settings or include a \mdcommand{clearpage} in front of 
the environment \mdpack{mdframed}.
So far I have no idea how to avoid such things.

\begin{mdexample}
 You got a bad break
 because the split box is empty
 You have to change the page settings
 like enlargethispage or something else
 You got a bad break
\end{mdexample} 
See the explanation above.

\begin{mdexample}
 You got a bad break
 because the last split box is empty
 You have to change the settings
\end{mdexample} 
The same reason as above but only in the last box.


\begin{mdexample}
 Option ... is already consumed
 and has no effect on input line ...
\end{mdexample} 
If you set a global option inside the document
body you will get this warning.

\section{Known Problems}
 In this section I will collect known problems. In case you encounter any further problems, please
 drop me an email, \href{mailto:marco.daniel@mada-nada.de}{marco.daniel at mada-nada.de}.

   Do you have any ideas / wishes on further extensions to this package? Please let me know!

\begin{enumerate}
 \item So far the environment isn't compatible with the package \mdpack{gmverb}.
\end{enumerate}


\section{ToDo}
\begin{enumerate}
  \item see \glqq Known Problems\grqq.
 \item So far it isn't possible to combine the environment  \mdcommand{begin\{multicols\}} of the package \mdpack{multicol} with \mdpack{mdframed} with the whole option list.
  \item Create new styles.
  \item Improve page breaks.
  \item Improve footnotes
  \item Improve documentation and examples 
  \item Create styles for \mdoption{frametitle}
\end{enumerate}                

\section{Previous versions}
If you have trouble with the new version of \mdname you can load \texttt{mdframedpre} instead. In this way you load the version v0.6 (see \nameref{rev}).
%With the new version 0.7a I changed the internal commands (I use \mdpack{etoolbox}) and the algorithm of the splitting. So I don't know whether I will get new bugs or not. For the last case and the previous version (v0.6) works well you can load \mdpack{mdframedpre} instead of \mdpack{mdframed}.

\section{Acknowledgements}

\vskip\baselineskip



\textsf{Dick Nickalls;\quad Dietrich Grau;\quad Piazza Luca\quad Jobst Hoffmann}.
\medskip

\noindent Thanks for proofreading 

\textsf{Alan Munn} and \textsf{Nahid Shajari}                                       
\clearpage
\appendix
\section{More information}
%\addtocontents{toc}{\protect\setcounter{tocdepth}{-4}}
In the following section I want to present how to create
your own frame.
\subsection{How does \mdpack{mdframed} work?}
With the environment \mdcommand{begin\{mdframed\}} \ldots \mdcommand{end\{mdframed\}}
the whole contents will be saved in a \mdcommand{savebox} called \mdcommand{@tempboxa}.
After the calculation of the width and the height of the \mdcommand{@tempboxa} (done by \mdpack{mdframed.sty}) the box 
will be set sequently (done by \mdpack{md-frame-X.mdf}). The following figure demonstrates this.

\begin{center}
\begin{tikzpicture}
\newcommand{\mybox}[1]{\parbox{3.5cm}{\centering\small #1}}
\tikzstyle{every node}=[font=\sffamily]
\tikzset{mymatrixnodes/.style={%
     matrix of nodes,
     nodes={framed,draw=black,fill=yellow!10,},
     column sep=4.2cm,
     row sep=1.2cm,
     }}
\matrix(m)[mymatrixnodes]
  {%
   \mybox{read in and save contents in \mdcommand{@tempboxa}} &
   \mybox{\mdcommand{md@putbox@single}} \\
   \mybox{split \textbackslash @tempboxa in  \mdcommand{tw@} to the free space of the page} &
   \mybox{\mdcommand{md@putbox@first}} \\
   \mybox{contents of \mdcommand{@tempboxa} have enough space on the new page} &
   \mybox{\mdcommand{md@putbox@second}} \\
   \mybox{split \mdcommand{@tempboxa} in \mdcommand{tw@} to the free space of the page} &
   \mybox{\mdcommand{md@putbox@middle}} \\
};

\tikzstyle{every path}=[black,-triangle 45,thick,shorten >=4pt,shorten <=4pt]
\draw  (m-1-1) edge node[align=center,anchor=south]{enough space on page} (m-1-2);
\draw  (m-2-1) edge node[align=center,anchor=south]{type out \mdcommand{tw@}} (m-2-2);
\draw  (m-3-1) edge node[align=center,anchor=south]{yes} (m-3-2);
\draw  (m-4-1) edge node[align=center,anchor=south]{type out \mdcommand{tw@}} (m-4-2);

\draw  (m-2-2) edge node[align=center,fill=white]%
       {calculate the height of the new \mdcommand{@tempboxa}} (m-3-1);


\draw  (m-1-1) edge node[align=center,anchor=west]{not enough space on page} (m-2-1);
\draw  (m-3-1) edge node[align=center,anchor=west]{no} (m-4-1);

\draw  let \p1=($(m-4-2.south)-(0,1)$), \p2=($(m-3-1.west)-(0.8,0)$) in
      (m-4-2.south) -- ($(m-4-2.south)+(0,-1)$) --
           node[align=center,anchor=north]%
           {calculate the height of the new \mdcommand{@tempboxa}} (\x2,\y1)
       -- ($(m-3-1.west)-(0.8,0)$)-- ($(m-3-1.west)$) ;


\end{tikzpicture}

\captionof{figure}{Setting the contents of \mdpack{mdframed}}\label{fig:way}
\end{center}

The width of the contents is the result of the settings of \mdoption{leftmargin}, \mdoption{rightmargin}, \mdoption{linewidth}, \mdoption{innerleftmargin} and \mdoption{innerrightmargin} (see figure \eqref{fig:laengen}).



\subsection{The Framecommands}
The package \mdpack{mdframed} knows four kinds of \glqq Framecommand\grqq. These commands tell \LaTeX\ how to set the contents of \mdpack{mdframed}. 
\begin{description}
       \item[\mdcommand{md@putbox@single}] This command sets the contents of a single unsplit frame.
       \item[\mdcommand{md@putbox@first}]  This command sets the contents of the first frame of a split frame.
       \item[\mdcommand{md@putbox@middle}] This command sets the contents of the middle frame of a split frame.
       \item[\mdcommand{md@putbox@second}] This command sets the contents of the last frame of a split frame.
\end{description}

Using the explained commands we give an example. The command \mdcommand{box} uses the contents of the savebox and types them out. 

First we want to type out the single box without any settings (but with the calculated width).
\begin{mdexample}
\makeatletter
 \def\md@putbox@single{\box\@tempboxa}
\makeatother
\end{mdexample}
I am using the command \mdcommand{leftline} to start the \glqq Framecommands\grqq\ at the left.
\begin{mdexample}[deletekeywords={leftline},moretexcs={leftline}]
\makeatletter
 \def\md@putbox@single{\leftline{\box\@tempboxa}}
\makeatother
\end{mdexample}

Now you have to know how the lengths are named. Every length which can be modified by the options has the following syntax:
\begin{mdexample}
\mdf@<Name of the Length>@length
\end{mdexample}
For example the leftmargin is:
\begin{mdexample}
\mdf@leftmargin@length
\end{mdexample}

To create only a line at the left with the correct \mdoption{leftmargin} you can set \mdcommand{md@putboxsingle} as follows

\begin{mdexample}[deletekeywords={leftline},moretexcs={leftline}]
\makeatletter
\def\md@putbox@single{%
         \leftline{%
             \hspace*{\mdf@leftmargin@length}%
             \rule[-\dp\@tempboxa]{\mdf@linewidth}%
                  {\ht\@tempboxa+\dp\@tempboxa}%
             \box\@tempboxa
         }%
}
\makeatother
\end{mdexample}

In this way you can do what you want. If you create your own style you
can save the file as \mdpack{md-frame-X.mdf}.
\mdpack{X} must be an integer. 
In this way you can use the option \mdoption{framemethod} to
load the file by setting \mdoption{framemethod=X}.

\clearpage



%\clearpage
\subsection{Revision history}\label{rev}
\raggedright
\minisec{Version 1.0 submitted 13 Nov 2011}
\begin{itemize*}
\item add option \mdoption{userdefinedwidth}
\item add option \mdoption{align}
\item add option \mdoption{apptotikzsetting}
\item create new command \mdcommand{mdfapptodefinestyle}
\item changed internal algorithm
\item removed calc instead using \eTeX\ \verb+\dimexpr+
\item expand documentation
\item trying to fixe problems with \mdpack{xcolor}
\item fixed bug with \mdoption{framemethod=pstricks}
\item create file \mdpack{mdframed-example-default}
\item create file \mdpack{mdframed-example-tikz}
\item create file \mdpack{mdframed-example-pstricks}
\item create file \mdpack{mdframed-example-texsx} (texsx stands for tex stackexchange)
\end{itemize*}


\footnotesize
\minisec{Version 0.9g submitted 08 Oct 2011}
\begin{itemize*}
\item fixed documentation
\item added small footnote compatibility
\end{itemize*}

\minisec{Version 0.9f submitted 04 Oct 2011}
\begin{itemize*}
\item fixes bugs (thanks to Lars Madsen)
\item added option \mdoption{hidealllines}
\item fixed documentation
\end{itemize*}


\minisec{Version 0.9e submitted 11 Sep 2011}
\begin{itemize*}
\item working with \texttt{twoside} modus
\end{itemize*}

\minisec{Version 0.9d submitted 10 Sep 2011}
\begin{itemize*}
\item \textcolor{red}{changed the meaning of the option \mdoption{style}}!!! (inspired by Lars Madsen)
\item added option \mdoption{framemethod} (inspired by Lars Madsen)
\item added options \mdoption{needspace} (inspired by Lars Madsen)
\item added new command \mdcommand{mdfdefinestyle} (inspired by Lars Madsen)
\item fixes documentation
\item renamed \mdpack{md-frame-3.mdf} to \mdpack{md-frame-2.mdf}
\end{itemize*}



\minisec{Version 0.9b submitted 7 Sep 2011}
\begin{itemize*}
\item fixes bugs in \mdcommand{newmdtheoremenv} (Thanks to Enrico Gregorio)
\end{itemize*}

\minisec{Version 0.9a submitted 5 Sep 2011}
\begin{itemize*}
\item fixes bugs (Thanks to Lars Madson)
\item expanded documentation (added revision history)
\end{itemize*}


\minisec{Version 0.9 submitted 4 Sep 2011}
\begin{itemize*}
\item added option \mdoption{nobreak}
\item detecting float environments to prevent split calculation
\item expand documentation (Thanks to Alan Munn)
\end{itemize*}

\minisec{Version 0.8a}
\begin{itemize*}
\item fixes bugs
\item fixes documentation
\end{itemize*}

\minisec{Version 0.8 submitted 22 Aug 2011}
\begin{itemize*}
\item  added commands: \mdcommand{newmdenv}, \mdcommand{renewmdenv}, \mdcommand{newmdtheoremenv}
\item fixes bugs
\item fixes documentation
\end{itemize*}

\minisec{Version 0.7a submitted 6 August 2011}
\begin{itemize*}
\item added option \mdoption{frametitle}
\item added option \mdoption{frametitlefont}
\item allow twolumn-mode
\item changed the calculation
\item added option \mdoption{tikzsetting}
\item added options for hidden lines for all styles
\item fixes bugs
\end{itemize*}

\minisec{Version 0.6a submitted 22 Dec 2010}
\begin{itemize*}
\item fixes bugs
\item added \mdcommand{mdfsetup}
\item expanded documentation
\end{itemize*}



\minisec{Version 0.6 submitted 18 Dec 2010}
\begin{itemize*}
\item added \mdoption{style=3} with \mdpack{pstricks}
\item added option \mdoption{pstrickssetting}
\item added option \mdoption{splitbottomskip}
\item added option \mdoption{splittopskip}
\item added options for hidden lines 
\item changed the calculation
\item fixes bugs
\end{itemize*}

\minisec{Version 0.4a submitted 14 May 2010}
\begin{itemize*}
\item fixes bug in fontcolor
\end{itemize*}

\minisec{Version 0.4 submitted 13 May 2010}
\begin{itemize*}
\item Elke Schubert creates style file for \mdpack{tikz}
\item fixes some bugs -- calculation of the page dimen (thanks  Dick Nickalls)
\item using tikz for the frame with different styles
\end{itemize*}

\minisec{Version 0.3b submitted 1 May 2010}
\begin{itemize*}
\item fixes some bugs -- thanks to Dietrich Grau
\item added new options: \mdoption{ntheorem}
\end{itemize*}

\minisec{Version 0.3a submitted 23 Apr 2010}
\begin{itemize*}
\item added new options: \mdoption{leftmargin} and \mdoption{rightmargin}
\item fixes some bugs
\end{itemize*}

\minisec{Version 0.3 submitted16 Apr 2010}
\begin{itemize*}
\item first upload to \href{http://dante.ctan.org/upload}{CTAN}
\end{itemize*}
\end{document}
