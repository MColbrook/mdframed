\setcounter{errorcontextlines}{999}
\documentclass[12pt,ngerman,twoside]{scrbook}
\usepackage{babel,blindtext}
\usepackage{showframe}
\usepackage{microtype}
\usepackage[T1]{fontenc}

\usepackage{amsmath,ntheorem}

\usepackage[%
usetwoside=false,
%printheight=info,
framemethod=default,
%framemethod=tikz,
%framemethod=pstricks,
%style=1,
%globalstyle=3,
linewidth=30pt,
linecolor=blue,
%roundcorner=20,
%hidealllines=true,
backgroundcolor=yellow!30,
innerlinecolor=blue!40,
middlelinecolor=red!20,
outerlinecolor=yellow!60!black,
%%margin=0pt,
%%rightmargin=0cm,
innertopmargin=.5cm,
%%innerbottommargin=0cm,
%innerleftmargin=1cm,
%innerrightmargin=1cm,
%leftmargin=1cm,
%rightmargin=1cm,
%%%xcolor=cmyk,%
%splitbottomskip=1sp,
splittopskip=1cm,
%%globalstyle={0,1}
%innermargin=10pt,
%outermargin=40pt,,
%%footnoteinside=false,
%%leftline=false,
%fontcolor=red,
frametitlerule=true,
frametitlebackgroundcolor=green,
frametitlerulewidth=21pt,
frametitlerulecolor=red,
%align=center,
%leftline=false,
%rightline=false,
%topline=false,
%bottomline=false
]{mdframed}
\makeatletter


\usepackage{multicol}
\usepackage{pst-node}
\usepackage{lipsum}

\usepackage{svninfo}
\svnInfo $Id: md-test.tex 298 2012-01-02 00:28:01Z marco $
\mdfsetup{frametitle={Testtitle Hallo Welt Testtitle Hallo Welt Testtitle Hallo Welt Testtitle Hallo Welt Testtitle Hallo Welt Testtitle Hallo Welt Testtitle Hallo Welt Testtitle Hallo Welt Testtitle Hallo Welt Testtitle Hallo Welt Testtitle Hallo Welt Testtitle Hallo Welt
%Testtitle Hallo Welt Testtitle Hallo Welt Testtitle Hallo Welt Testtitle Hallo Welt Testtitle Hallo Welt Testtitle Hallo Welt Testtitle Hallo Welt Testtitle Hallo Welt Testtitle Hallo Welt Testtitle Hallo Welt Testtitle Hallo Welt Testtitle Hallo Welt
\lipsum[1]
}}
\mdfsetup{leftmargin=0cm,rightmargin=0cm,innerleftmargin=40pt,innerrightmargin=10pt,
innertopmargin=1cm,innerbottommargin=1cm}
%\mdfsetup{skipabove=0cm}
\mdfsetup{%
outerlinewidth=0.2cm,
middlelinewidth=0.1cm,
innerlinewidth=0.2cm,
%innerbottommargin=0pt,
%innertopmargin=0pt,
roundcorner=20pt,
splitbottomskip=1cm,
}
\mdfsetup{frametitle={Hallo Welt}}
%\mdfsetup{leftline=false,bottomline=true,topline=false,rightline=true}
%\mdfsetup{leftline=false,rightline=false}
\begin{document}
%\lipsum[1]
%\section{Example Env}
%\begin{mdframed}[frametitle={Theorem: Inhomegeneous Linear},
%frametitlerulecolor=green!70!black,
%frametitlerulewidth=.3pt,
%frametitlefont=\normalfont\bfseries,
%outerlinewidth=0pt,
%middlelinewidth=1.5pt,
%innerlinewidth=0pt,
%middlelinecolor=green!70!black,
%frametitlebackgroundcolor=gray!30,
%skipabove=\topskip,skipbelow=\topskip,
%leftmargin=.5cm,rightmargin=.5cm,
%]
%        An \textit{inhomogeneous linear} differential equation has the form
%         \begin{align}
%            L[v ] = f,
%         \end{align}
%        where $L$ is a linear differential operator, $v$ is
%        the dependent variable, and $f$ is a given non-zero
%        function of the independent variables alone.
%\end{mdframed}
%\lipsum[2]
%\mdfsetup{rightline=false,skipabove=2cm}
%%\enlargethispage{1\baselineskip}
Test\smash{\rule[-2cm]{2pt}{2cm}}
%
\lipsum[1]\lipsum[1]
%\mdfsetup{innertopmargin=.5cm,frametitlebelowskip=0.8cm}

\begin{mdframed}[frametitle={page break \lipsum[1]\lipsum[1]}]
Test

\begin{enumerate}
\itemsep=0pt
\item  $\psi$ beschreibt\smash{\rule[0.8\ht\strutbox]{2pt}{.5cm}} eine gedachte Gesamtheit von unendlich vielen gleich präparierten Systemen.
\item Die Wahrscheinlichkeiten werden als relative Häufigkeiten in einer solchen gedachten Gesamtheit interpretiert. Im Experiment wird sie 
\item Präparations- und Messapparate werden klassisch beschrieben. Die Konzepte Präparation und Messung gehen schon in die Formulierung der Grundgesetze ein.
\item. Die Phänomene (Wirkungsübertragungen) werden als real angesehen. Die Quantenmechanik liefert keinerlei Aussagen bezüglich Existenz bzw. Verhalten eines Einzelsystems zwischen Präparation
und Messunga ,
\item Präparations- und Messapparate werden klassisch beschrieben. Die Konzepte Präparation und Messung gehen schon in die Formulierung der Grundgesetze ein.
\item. Die Phänomene (Wirkungsübertragungen) werden als real angesehen. Die Quantenmechanik liefert keinerlei Aussagen bezüglich Existenz bzw. Verhalten eines Einzelsystems zwischen Präparation
%und Messunga ,
\item Anwendungsbereich sind von Mikrosystemen
\item Anwendungsbereich sind von Mikrosystemen
\end{enumerate}
\lipsum[1]\lipsum[1]
\lipsum[1]\lipsum[1]
%\lipsum[1]\lipsum[1]
\end{mdframed}

\noindent\null\smash{\rlap{\rule[4cm]{1cm}{2pt}}%
\rlap{\rule[6cm]{2cm}{2pt}}%
\rlap{\rule[8cm]{3cm}{2pt}}}
\hfill\smash{\llap{\rule[4cm]{1cm}{2pt}}%
\llap{\rule[6cm]{2cm}{2pt}}%
\llap{\rule[8cm]{3cm}{2pt}}}


\end{document}


%%\mdfframetitleboxtotalheight-\mdfboundingboxheight}}
\begin{enumerate}
%\itemsep=0pt
\item $\psi$ beschreibt eine gedachte Gesamtheit von unendlich vielen gleich präparierten Systemen.
\item Die Wahrscheinlichkeiten werden als relative Häufigkeiten in einer solchen gedachten Gesamtheit interpretiert. Im Experiment wird sie 
%\item Präparations- und Messapparate werden klassisch beschrieben. Die Konzepte Präparation und Messung gehen schon in die Formulierung der Grundgesetze ein.
%\item. Die Phänomene (Wirkungsübertragungen) werden als real angesehen. Die Quantenmechanik liefert keinerlei Aussagen bezüglich Existenz bzw. Verhalten eines Einzelsystems zwischen Präparation
%und Messunga ,

%\item Anwendungsbereich sind von Mikrosystemen
\end{enumerate}
%\clearpage
%Test

\begin{mdframed}[frametitlerulewidth=20pt,frametitlebelowskip=1cm,backgroundcolor=orange,frametitlerulecolor=red,]
Text\smash{\rule[0.8\ht\strutbox]{2pt}{.5cm}}

%Text\footnote{foo}

\lipsum[1]
%\lipsum[2]\lipsum[2]\lipsum[2]
%\lipsum[2]\lipsum[2]\lipsum[2]\lipsum[2]
%TExt\footnote{bar}
\end{mdframed}



%\vbadness=10000
%\begin{mdframed}[outermargin=1cm,innerbottommargin=1cm,skipbelow=1cm,footenotedistance=\bigskipamount]
%Text
%
%Text\footnote{foo}
%
%\lipsum[2]
%
%\the\hsize
%
%TExt\footnote{bar}
%\end{mdframed}
%\end{enumerate}


\lipsum[2]
%\end{multicols}

HHHALLLADAD

\lipsum[1-4]
%\enlargethispage{\baselineskip}

\begin{mdframed}[style=marco,needspace=5\baselineskip]
        \lipsum
         \lipsum[1]
        \begin{itemize}
        \item Hallo
        \end{itemize}
\end{mdframed}

\end{document}
%\begin{beispiel}
%\lipsum
%\end{beispiel}
%\mdfsetup{style=0,linewidth=5pt, innertopmargin=0.75cm, innerbottommargin=1cm,%
%          linecolor=red,innerleftmargin=0.0cm,innerrightmargin=2cm, backgroundcolor=green,
%          splittopskip=2cm,splitbottomskip=1cm,skipabove=1cm,
%          }


\mdfsetup{%
%topline=false,
%bottomline=false,
%leftline=false,
%rightline=false,
%psroundlinecolor=yellow,
%linewidth=10,
%skipabove=0cm,
%splitbottomskip=1cm,
%splittopskip=2cm,
%roundcorner=20pt,
%pstrickssetting={linestyle=dashed},
%frametitle={Hallo Welt},
%frametitlefont=\itshape,
}

%\noindent\begin{minipage}{\linewidth}
%\begin{mdframed}[leftmargin=0cm,rightmargin=0cm,innertopmargin=1cm, innerbottommargin=1cm,,linewidth=20,
%                 innerleftmargin=1cm,innerrightmargin=1cm,backgroundcolor=green,linecolor=blue,]
%%        \lipsum
%         \lipsum[1]
%        \begin{itemize}
%        \item Hallo
%        \end{itemize}
%        \the\linewidth
%        \the\hsize
%        \the\textwidth
%        \the\columnwidth
%\end{mdframed}
%\end{minipage}
%\lipsum
%\vspace{15cm}

%marco \the\pagegoal\the\pagetotal
%%\clearpage
%%\the\pagegoal



\begin{mdframed}[style=marco,]
%        \lipsum
         \lipsum[1]
        \begin{itemize}
        \item Hallo
        \end{itemize}
\end{mdframed}
%\clearpage
Box 2

\begin{mdframed}%[leftmargin=2cm,rightmargin=1cm,linewidth=0.5cm]
        \lipsum[2]
\end{mdframed}

\end{document}

\clearpage
test
\clearpage

\begin{mdframed}%[leftmargin=2cm,rightmargin=1cm,linewidth=0.5cm]
        \lipsum[2-16]
\end{mdframed}

Absatz nach der geteilten Box
\clearpage

%\end{document}

\the\vsize

\svnInfoRevision
\footnote{Hallo}\footnote{Hallo}

%\mdfsetup{leftline=true,topline=true,rightline=true,bottomline=false,roundcorner=15}



Text  Text  Text  Text  Text  Text  Text  Text  Text  Text  Text  Text  
\[a+v\]
Text  Text  Text  Text  Text  Text  Text  Text  Text  Text  Text  Text
Text  Text  Text  Text  Text  Text  Text  Text  Text  Text  Text  Text
\begin{mdframed}[innerlinewidth=8pt,middlelinewidth=0.4cm,outerlinewidth=10pt,roundcorner=5pt,%
                 linewidth=0.4cm,innerleftmargin=0pt,innerrightmargin=0pt%
                 splittopskip=\baselineskip,backgroundcolor=yellow,%
                 innertopmargin=00pt,innerbottommargin=00pt,
                 leftmargin=0cm,innerleftmargin=0.0cm,innerrightmargin=0.0cm,
                 pstrickssetting={linestyle=dashed},]
 \lipsum[1-2]
\end{mdframed}       
\lipsum[1]

%%\clearpage
\newtheorem{mdtheorem}{Theorem}[section]
\newenvironment{theorem}{\begin{mdframed}%
   [linewidth=0.4cm,leftmargin=1cm,ntheorem=true,%
                 rightmargin=0cm,innerleftmargin=0cm,innerrightmargin=0cm,%
                 innertopmargin=00pt,innerbottommargin=0pt,%\topskip,%
                 pstrickssetting={linestyle=dashed},splittopskip=0cm,backgroundcolor=yellow,
%                 leftline=false,topline=false,rightline=true,bottomline=false,
                 ]%
%
   \begin{mdtheorem}}%
 {\end{mdtheorem}\end{mdframed}}
\begin{theorem}[Pythagorean theorem]
 In any right triangle, the area of the square whose side is the hypotenuse
 is equal to the sum of the areas of the\goodbreak squares whose sides are the two legs.

\[ a^2+b^2=c^2 \]
Hallo
 In any right triangle, the area of the square whose side is the hypotenuse
 is equal to the sum of the areas of the squares whose sides are the two legs.
 In any right triangle, the area of the square whose side is the hypotenuse
 is equal to the sum of the areas of the squares whose sides are the two legs.
\end{theorem}

Test

\begin{mdframed}[linewidth=0.3330cm,leftmargin=1cm,pstrickssetting={linestyle=solid},
                 rightmargin=0cm,innerleftmargin=0cm,innerrightmargin=0cm,%
                 innertopmargin=00pt,innerbottommargin=00pt,%\topskip,%
                 pstrickssetting={linestyle=dashed},splittopskip=1cm,backgroundcolor=yellow,
%                  leftline=true,topline=false,rightline=false,bottomline=false,
                 ]
Text  Text  Text  Text  Text  Text  Text  Text  Text  Text  Text  Text  
\[a+v+z\]
Text  Text  Text  Text  Text  Text  Text  Text  Text  Text  Text  Text
Text  Text  Text  Text  Text  Text  Text  Text  Text  Text  Text  Text
\lipsum
\lipsum\lipsum
\end{mdframed} 
\clearpage
%
%\begin{mdframed}[skipabove=0cm, outerlinewidth=4pt , linewidth=0.4cm, innerlinewidth=1cm ,%
%outerlinecolor=blue , middlelinecolor=yellow , innerlinecolor=red ,%
%backgroundcolor=orange,roundcorner=10pt,skipbelow=2cm,innerleftmargin=1cm,%
%]
%\lipsum[1]
%\end{mdframed}
%\clearpage
%\begin{mdframed}[skipabove=0cm, outerlinewidth=4pt , middlelinewidth=2pt , innerlinewidth=1pt ,%
%outerlinecolor=blue , middlelinecolor=yellow , innerlinecolor=red ,%
%backgroundcolor=orange,roundcorner=10pt,skipbelow=3cm]
%hallo
%\begin{itemize}
% \item HALLO
%\end{itemize}
%\begin{align}
%        x+y
%\end{align}
%\lipsum[1-22]
%\end{mdframed}
%Hier geht es weiter
%    \begin{lemma}Text
%           \begin{equation}
%               x+y=2
%           \end{equation}
%           \par\noindent\rule{\linewidth}{2pt}
%    \end{lemma}%
%\clearpage%
%\begin{mdframed}%
%Some Text with an first empy line. The is only text to fill the line. It has no sence.
%
%Some Text with an first empy line. The is only text to fill the line. It has no sence.
%\end{mdframed}
%
\clearpage
   \section{foo}
% % % % % % % % % % %
   \begin{lemma}
%        \lipsum[1]\lipsum[1]\lipsum[1]
        \begin{itemize}
          \item HALLO
        \end{itemize}
    \end{lemma}

   \begin{lemma}
      HALLO WELT!
   \end{lemma}
    \begin{lemma}\mbox{ }\par\noindent
               \rule{\linewidth}{4pt}
    \end{lemma}
    \begin{mdframed}
       \rule{\linewidth}{4pt}
     \end{mdframed}
    \begin{lemma}
          \lipsum[5]\lipsum[5]
          \lipsum[5]\lipsum[5]
    \end{lemma}
    \lipsum[1]%
    \begin{lemma}Text
           \begin{equation}
               x+y=2
           \end{equation}
           \par\noindent\rule{\linewidth}{2pt}
    \end{lemma}%
    
        \begin{lemma}\mbox{ }
                   \begin{equation}
                       x+y=2
                   \end{equation}
                   \par\noindent\rule{\linewidth}{2pt}
            \end{lemma}
    \begin{mdframed}%
           \lipsum[1]\par\noindent\rule{\linewidth}{2pt}
    \end{mdframed}

\end{document}
