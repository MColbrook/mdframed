% \iffalse meta-comment
% !TEX program = pdfLaTeX
%<*internal>
\iffalse
%</internal>
%<*readme>
================================================================
================================================================
Working with the command \fbox or \fcolorbox, one has to
handle page breaks by hand. The present package defines the
environment mdframed which automatically deals with page breaks.

Author's name: Marco Daniel and Elke Schubert (supports tikz implementation)
License type: lppl

================================================================
The current development is available at github:
https://github.com/marcodaniel/mdframed

================================================================
The package provides two dtx files, one Makefile and 
one documentclass. By running the script `make all` you will
produce the following files:
- README.txt
- mdframed.sty
- mdframed.pdf
- md-frame-0.mdf
- md-frame-1.mdf
- md-frame-2.mdf
- md-frame-3.mdf
- mdframed-examples.pdf
- mdframed-example-default.tex
- mdframed-example-default.pdf
- mdframed-example-tikz.tex
- mdframed-example-tikz.pdf
- mdframed-example-pstricks.tex
- mdframed-example-pstricks.pdf
- mdframed-example-texsx.tex  
- mdframed-example-texsx.pdf
- ltxmdf.cls 
- donald-duck.jpg 

I recommend the following installation in your texmf-tree:

/doc/latex/mdframed/
- README.txt
- mdframed.pdf
- mdframed-example-default.tex
- mdframed-example-default.pdf
- mdframed-example-tikz.tex
- mdframed-example-tikz.pdf
- mdframed-example-pstricks.tex
- mdframed-example-pstricks.pdf
- mdframed-example-texsx.tex  
- mdframed-example-texsx.pdf


/source/latex/mdframed/
- Makefile
- mdframed.dtx
- mdframed-examples.dtx
- donald-duck.jpg

/tex/latex/mdframed
- mdframed.sty
- md-frame-0.mdf
- md-frame-1.mdf
- md-frame-2.mdf
- md-frame-3.mdf
- ltxmdf.cls
================================================================
================================================================
%</readme>
%<*internal>
\fi
\def\nameofplainTeX{plain}
\ifx\fmtname\nameofplainTeX\else
  \expandafter\begingroup
\fi
%</internal>
%<*install>
\input docstrip.tex
\keepsilent
\askforoverwritefalse
\preamble
----------------------------------------------------------------
Working with the command fbox or fcolorbox, one has to
handle page breaks by hand. The present package defines the
environment mdframed which automatically deals with page breaks.

Author's name: Marco Daniel and Elke Schubert (!new)
License type: lppl

==================================================
========Is based on the idea of framed.sty========
==================================================
===== Currently the package has a beta-Status ====
==================================================
 WITH THANKS TO (alphabetically):
 ROLF NIEPRASCHK
 HEIKO OBERDIEK
 HERBERT VOSS

 Copyright (c) 2010 Marco Daniel

 This package may be distributed under the terms of the LaTeX Project
 Public License, as described in lppl.txt in the base LaTeX distribution.
 Either version 1.0 or, at your option, any later version.


=================================================
 Erstellung eines Rahmens, der am Seitenende keine
 horizontale Linie einfuegt
>>>>>>>>>>>>>>>>>>>>>>>>>>>>>>>>>>>>>>>>>>>>>>>>>
      _______________                            
      |    page 1   |                              
      |    Text     |                            
      |  __Text__   |                            
      |  | Text |   |                            
     P A G E B R E A K                           
      |  | Text |   |                            
      |  |_Text_|   |                            
      |    Text     |                            
      |____page 2___|                            
                                                 
>>>>>>>>>>>>>>>>>>>>>>>>>>>>>>>>>>>>>>>>>>>>>>>>>
==================================================

\endpreamble
\postamble

================================================================
Copyright (C) 2011 by Marco Daniel

This work may be distributed and/or modified under the
conditions of the LaTeX Project Public License (LPPL), either
version 1.3c of this license or (at your option) any later
version.  The latest version of this license is in the file:

http://www.latex-project.org/lppl.txt

This work is "maintained" (as per LPPL maintenance status) by
Marco Daniel.

Have fun!

================================================================
\endpostamble
\usedir{tex/latex/mdframed}
\generate{\file{mdframed.sty}{\from{mdframed.dtx}{package}}}
\nopreamble\nopostamble
\generate{\file{md-frame-0.mdf}{\from{mdframed.dtx}{mdframedefault}}}
\nopreamble\nopostamble
\generate{\file{md-frame-1.mdf}{\from{mdframed.dtx}{mdframetikz}}}
\nopreamble\nopostamble
\generate{\file{md-frame-2.mdf}{\from{mdframed.dtx}{mdframepstricks}}}
\nopreamble\nopostamble
\generate{\file{md-frame-3.mdf}{\from{mdframed.dtx}{mdframepstricks}}}
\Msg{*********************************************************}
\Msg{*}
\Msg{* To finish the installation you have to move the}
\Msg{* following file into a directory searched by TeX:}
\Msg{*}
\Msg{* \space\space documentation.sty}
\Msg{*}
\Msg{* To produce the documentation run the file documentation.dtx}
\Msg{* once through LaTeX. Then, run}
\Msg{*}
\Msg{* \space\space makeindex -s gglo.ist -o mdframed.gls mdframed.glo}
\Msg{* \space\space makeindex -s gind.ist mdframed.idx}
\Msg{*}
\Msg{* through makeIndex to produce the glossary. Finally, run LaTeX once again.}
\Msg{* That's all!}
\Msg{*}
\Msg{* Happy TeXing!}
\Msg{*********************************************************}
%</install>
%<install>\endbatchfile
%<*internal>
\usedir{source/latex/mdframed}
\generate{\file{mdframed.ins}{\from{mdframed.dtx}{install}}}
\nopreamble\nopostamble
\usedir{doc/latex/mdframed}
\generate{\file{README.txt}{\from{mdframed.dtx}{readme}}}
\ifx\fmtname\nameofplainTeX
  \expandafter\endbatchfile
\else
  \expandafter\endgroup
\fi
%</internal>
%<*driver>
%%$Id: mdframed.dtx 287 2011-12-19 20:12:09Z marco $
\setcounter{errorcontextlines}{999}
\documentclass[parskip=false,11pt,]{ltxmdf}
\usepackage[framemethod=TikZ]{mdframed}
\ltxmdfsetifoot$Id: mdframed.dtx 287 2011-12-19 20:12:09Z marco $

\EnableCrossrefs
\CodelineIndex
\RecordChanges
\begin{document}
  \DocInput{mdframed.dtx}
\end{document}
%</driver>
% \fi
%
% CheckSum{3517}
%
% \CharacterTable
%  {Upper-case    \A\B\C\D\E\F\G\H\I\J\K\L\M\N\O\P\Q\R\S\T\U\V\W\X\Y\Z
%   Lower-case    \a\b\c\d\e\f\g\h\i\j\k\l\m\n\o\p\q\r\s\t\u\v\w\x\y\z
%   Digits        \0\1\2\3\4\5\6\7\8\9
%   Exclamation   \!     Double quote  \"     Hash (number) \#
%   Dollar        \$     Percent       \%     Ampersand     \&
%   Acute accent  \'     Left paren    \(     Right paren   \)
%   Asterisk      \*     Plus          \+     Comma         \,
%   Minus         \-     Point         \.     Solidus       \/
%   Colon         \:     Semicolon     \;     Less than     \<
%   Equals        \=     Greater than  \>     Question mark \?
%   Commercial at \@     Left bracket  \[     Backslash     \\
%   Right bracket \]     Circumflex    \^     Underscore    \_
%   Grave accent  \`     Left brace    \{     Vertical bar  \|
%   Right brace   \}     Tilde         \~}
%
% \GetFileInfo{mdframed.sty}
%
% \changes{v1.0a}{2011/11/26}{Created dtx and fixes bugs}
%
% \title{The \Pack{mdframed} package\footnote{Extending the package \texttt{framed.sty}}
% \subtitle{auto-split frame environment}
% \author{\href{mailto:marco.daniel@mada-nada.de}{Marco Daniel}%
%        \footnote{With thanks to Heiko Oberdiek, Rolf Niepraschk,
%                  Martin Scharrer and Herbert Voss.\newline Sorry for bad English.}
%           \href{mailto:marco.daniel@mada-nada.de}{Elke Schubert}}}
% \version{\mdversion}
% \introduction{The standard methods for framing text (\Cmd{fbox} or \Cmd{fcolorbox}) require you to handle page breaks by hand, meaning that you have to split the \Cmd{fbox} into two. The present package defines the environment \Pack{mdframed} which automatically deals with pagebreaks in framed text.\par
% By defining new environments the user may choose between several individual designs.%
% \par\kern.5\baselineskip\noindent
% Linked files:\quad \begin{tabular}[t]{ll}
%             \url{mdframed-example-default.pdf} &
%             \url{mdframed-example-tikz.pdf} \\
%             \url{mdframed-example-pstricks.pdf}  &
%             \url{mdframed-example-texsx.pdf}
%             \end{tabular}
% \par\kern.5\baselineskip\noindent
% FYI:\quad\parbox[t]{.8\linewidth}{I create a repository for \Pack{mdframed} on \href{https://github.com/marcodaniel/mdframed}{github} where you can \href{https://github.com/marcodaniel/mdframed}{download} the current development status.}
% }
% 
% 
% 
% 
% \maketitle
% 
% \vspace*{2\baselineskip} 
%
% \section{Motivation}
% Many users wish to (further) emphasize lemmata, definitions, proofs, etc. The package \Pack{mdframed} allows you to create environments with breakable frames. I think an example is the best way to demonstrate its properties.
% 
% 
% %\newtheorem{mdtheorem}{Theorem}[section]
% \newmdtheoremenv[outerlinewidth=2,leftmargin=40,rightmargin=40,%
%     backgroundcolor=yellow,outerlinecolor=blue,innertopmargin=0pt,%
%     splittopskip=\topskip,skipbelow=\baselineskip,%
%     skipabove=\baselineskip,ntheorem,roundcorner=5pt]{theorem}{Theorem}[section]
% 
% \begin{theorem}[Pythagorean theorem]
%  In any right triangle, the area of the square whose side is the hypotenuse
%  is equal to the sum of the areas of the squares whose sides are the two legs.
%  
% \[ a^2+b^2=c^2 \]
% \end{theorem}
% 
% The frame was defined with the following settings.
% 
% \iffalse
%<*example>
% \fi
\begin{tltxmdfexample}
 \newmdtheoremenv[outerlinewidth=2,leftmargin=40,%
     rightmargin=40,backgroundcolor=yellow,%
     outerlinecolor=blue,innertopmargin=0pt,%
     splittopskip=\topskip,skipbelow=\baselineskip,%
     skipabove=\baselineskip,ntheorem]{theorem}%
     {Theorem}[section]
 \begin{theorem}[Pythagorean theorem]
 ...
 \end{theorem}
\end{tltxmdfexample}
% \iffalse
%</example>
% \fi
% 
% 
% \section{Syntax}\label{sec:syntax}
% The package itself loads the packages \Pack{kvoptions}, \Pack{etoolbox} and \Pack{color}.
% By setting the correct options \Pack{mdframed} will load \Pack{xcolor}, \Pack{tikz} or \Pack{pstricks}.
% 
% Load the package as usual:
% \iffalse
%<*example>
% \fi
\begin{tltxmdfexample}
%  \usepackage[<GLOBAL OPTIONS>]{mdframed}
\end{tltxmdfexample}
% \iffalse
%</example>
% \fi
% 
% The package defines only one environment with the following syntax:
% \iffalse
%<*example>
% \fi
\begin{tltxmdfexample}
  \begin{mdframed}[<LOCAL OPTIONS>]
     <CONTENT>
  \end{mdframed}
\end{tltxmdfexample}
% \iffalse
%</example>
% \fi
% To create own environments with \Pack{mdframed} see section \ref{sec:cmds}.
% 
% \minisec{Autodetecting floats}
% I added a detection of \Pack{float} or \Pack{minipage} environments. If you use \Pack{mdframed} in such an environment \Pack{mdframed} will use the option \Opt{nobreak} automaticly. 
% 
% \minisec{Twoside-mode}
% When you are using \Pack{mdframed} inside \Pack{twoside} mode you  can set the option \Opt{innermargin}
% and \Opt{outermargin} (see section \ref{optlength}).
% 
% %%%%%%%%%%%%%%%%%%%%%%%
% \section{Commands}\label{sec:cmds}
% The following commands should countenance your by the handling with \Pack{mdframed}
% 
% \ExplCmd{newmdenv} The command has the following syntax:
% \iffalse
%<*example>
% \fi
\begin{tltxmdfexample}
 \newmdenv[<MDFRAMED OPTIONS>]{Name of the environment}
\end{tltxmdfexample}
% \iffalse
%</example>
% \fi
% In this way you can simply use:
% \iffalse
%<*example>
% \fi
\begin{tltxmdfexample}
  \newmdenv[linecolor=red,frametitle=Infobox]{infobox}
  ...
  \begin{infobox}[backgroundcolor=yellow]
   foo   foo   foo   foo   foo   foo 
  \end{infobox}
\end{tltxmdfexample}
% \iffalse
%</example>
% \fi
% \ExplCmd{renewmdenv} By using this command you can redefine environments which are created by \Cmd{newmdenv}.
% 
% \ExplCmd{newmdtheoremenv} 
% Since the package is often used to highlight theorem environments, I have created a command\footnote{Thanks to Martin Scharrer and Enrico Gregorio:\newline \href{http://tex.stackexchange.com/questions/26298/own-command-to-create-new-environment}{Own command to create new environment}} to simplify this process. The command has the following syntax:
% 
% \iffalse
%<*example>
% \fi
\begin{tltxmdfexample}[deletekeywords={mdframed}]
 \newmdtheoremenv[<mdframed-options>]{<envname>}%
                 [<numberedlike>]{<caption>}[<within>]
\end{tltxmdfexample}
% \iffalse
%</example>
% \fi
% The last four arguments are equivalent to the command \Cmd{newtheorem}.
% Only the first optional argument is able to pass \Pack{mdframed}-options. A simple example is:
% \iffalse
%<*example>
% \fi
\begin{tltxmdfexample}
 \theoremstyle{<some style>}
 \newmdtheoremenv[linecolor=blue]{lemma}%
                 {Lemma}[section]
 ... 
 \begin{lemma}[Some title]
   foo   foo   foo   foo   foo   foo
 \end{lemma}
\end{tltxmdfexample}
% \iffalse
%</example>
% \fi
% So far there is no \Cmd{renewmdtheoremenv}!
% 
% 
% \ExplCmd{mdfsetup}To set the options 
% you can use the optional argument of \Cmd{usepackage} or
% you can use the command \Cmd{mdfsetup} which is not limited to the preamble.
%  Inside a group the settings work only local.\par
% \textbf{\color{red!70!black}At this point I want to recommend the using of the command \Cmd{mdfsetup}
% instead of setting package option via the optional argument of \Cmd{usepackage}. 
% So your are avoiding breaking of non robust commands.}^^A
% \footnote{Thanks to Heiko Oberdieck and Philipp Stephani
% \href{https://groups.google.com/group/de.comp.text.tex/browse_thread/thread/15c5ad290490d426?hl=de}^^A
%       {kvoptions-Declaration von Optionen schl\"agt fehl}}
% 
% \ExplCmd{mdfdefinestyle} \Cmd{mdfdefinestyle} allows the user to define different styles and use as an option of \Pack{mdframed} via \Opt{style}. The option \Opt{style} is explained in section \ref{genopt}.
% 
% Here a small example:
% \iffalse
%<*example>
% \fi
\begin{tltxmdfexample}[morekeywords={mystyle}]
 \mdfdefinestyle{mystyle}{leftmargin=0pt,%
                          linecolor=blue}
 ....
 \begin{mdframed}[style=mystyle]
  foo
 \end{mdframed}
\end{tltxmdfexample}
% \iffalse
%</example>
% \fi
% 
% \ExplCmd{mdfapptodefinestyle} This commands allows to expand a defined style.\footnote{Thanks to Martin Scharrer and Enrico Gregorio:\newline \href{http://tex.stackexchange.com/questions/34684/argument-of-setkeys}{http://tex.stackexchange.com/questions/34684/argument-of-setkeys}}
% 
% 
% 
% %%%%%%%%%%%%%%%%%%%%%%%
% \section{Options}
% The package provides various options to manipulate frames. In the following section all options are listed. Some internal macros which can be manipulated are not shown in this documentation.
% The listed option are divided in global and local options. The global options can not be used inside \Cmd{mdfsetup}.
% 
% \subsection{Global Options}\label{globopt}
% The following options are only global options.
% 
% \ExplOpt[none]{xcolor}
% By setting this key, the package \Pack{xcolor} will
% be loaded with the given value(s). Without any value \mbox{\Pack{mdframed}} loads
% the package \Pack{color} without any options. If the package \Pack{xcolor} is already loaded the given option will be ignored. I recommend to load \Pack{xcolor} before \Pack{mdframed}.
% 
% \ExplOpt[default]{framemethod} 
% With this key you can change the way frames are drawn. You can decide whether the frame is drawn with
% \begin{enumerate}
%  \item \LaTeX-commands {\small\Cmd{hrule}, \Cmd{vrule}, \Cmd{rule}},
%  \item \Pack{TikZ} {\small(the package \Pack{TikZ} will be loaded)} or
%  \item \Pack{PSTricks} {\small(the package \Pack{pstricks} will be loaded)}.
% \end{enumerate}
% The option \Opt{framemethod} requires a string. Allowed combinations are listed in the following table.
% \begin{table}[!ht]
% \centering
% \renewcommand*\arraystretch{1.3}
% \caption{Allowed keys for \Opt{framemethod}}
% \begin{tabular}{@{}>{\small\bfseries}l>{\small\ttfamily}l@{}}
% \toprule
% \multicolumn{1}{@{}l}{\textbf{Method}} & \multicolumn{1}{l}{Allowed keys} \\\midrule
% \LaTeX-commands   & default, tex, latex, none, 0  \\
% \Pack{TikZ}     & tikz, pgf, 1                  \\
% \Pack{PSTricks} & pstricks, ps, postscript, 2   \\\bottomrule
% \end{tabular}
% \end{table}
% 
% \noindent{\small\textbf{FYI}\qquad It is independently whether the \texttt{method} is written with no, one or more capital letter.}
% 
% \NOTE  The manipulation of the frames depends on the option \Opt{framemethod}. For further information see below.
% 
% \subsection{Global and Local Options}
% The options listed below can be set globally or locally and they are not limited to the preamble.
% 
% \subsubsection{Options with lengths}\label{optlength}
% 
% In figure \eqref{fig:laengen} you can see the adjustable lengths which will be described below. 
% All lengths accept two kinds of input. The first one is a length (e.g. 2pt) and
% the second one is a number (e.g. 2) which will be multiplied by \Opt{1 defaultunit}.
% The figure shows three different colored frames. Only \Opt{framemethod=tikz} is able to draw such triple lines.
% 
% I know that the predefined lengths are not well prepared. Maybe I will change it later.
% 
% \ExplOpt[pt]{defaultunit} see the sentence above.
% 
% \begin{center}
% \begin{tikzpicture}[scale=0.9]
%      %Declaration
%      \newlength{\linew}
%      \setlength{\linew}{5pt}
%      \tikzstyle{every node}=[font=\bfseries\large\sffamily]
%      \coordinate (innersep) at (3.2,2.25);
%      \coordinate (outersep) at (3.2,2);
%      %Center
%      \node[fill=green!10,draw=black,framed,rounded corners,minimum width=3cm,minimum height=1.5cm]%
%            (CenterNode) {Contents};
%      
%      %frames
%      \draw[draw=red!70!black,line width=\linew] %
%                ($(CenterNode)+(innersep)$) %
%                rectangle ($(CenterNode)-(innersep)$);
%      \draw[draw=blue!70!black,line width=\linew] %
%                ($(CenterNode)+(innersep)+(\linew,\linew)$)%
%                rectangle ($(CenterNode)-(innersep)+(-\linew,-\linew)$);
%      \draw[draw=yellow!70!black,line width=\linew]
%                ($(CenterNode)+(innersep)+(2\linew,2\linew)$)%
%                rectangle ($(CenterNode)-(innersep)+(-2\linew,-2\linew)$);
%      \draw[draw=black,line width=1] ($(CenterNode)+(innersep)+(outersep)$)%
%                rectangle ($(CenterNode)-(innersep)-(outersep)$);
%      
%      
%      \tikzstyle{every node}=[font=\small\ttfamily]
%      %%innertopmargin
%      \draw[draw=black,triangle 45-triangle 45]let \p1=(innersep) in
%            (CenterNode.north)  -- ($(CenterNode)+(0,\y1)-(0,.5\linew)$) ;
%      \path let \p1=(innersep) in
%            (CenterNode.north)  edge node[align=center,framed,fill=white,draw=white]{innertopmargin} 
%            ($(CenterNode)+(0,\y1)-(0,.5\linew)$) ;
%      %%innerbottommargin
%      \draw[draw=black,triangle 45-triangle 45]let \p1=(innersep) in
%            (CenterNode.south)  -- ($(CenterNode)-(0,\y1)+(0,.5\linew)$) ;
%      \path let \p1=(innersep) in
%            (CenterNode.south)  edge node[align=center,framed,fill=white,draw=white]{innerbottommargin} 
%            ($(CenterNode)-(0,\y1)+(0,.5\linew)$) ;
%      
%      %%innerleftmargin
%      \draw[draw=black,triangle 45-triangle 45]let \p1=(innersep) in
%            (CenterNode.west)  -- ($(CenterNode)-(\x1,0)+(.5\linew,0)$) ;
%      \path let \p1=(innersep) in
%            (CenterNode.west)  edge 
%            node[align=center,rotate=90,framed,fill=white,draw=white]{innerleftmargin}  
%            ($(CenterNode)-(\x1,0)+(.5\linew,0)$) ;
%      %%innerrightmargin
%      \draw[draw=black,triangle 45-triangle 45]let \p1=(innersep) in
%            (CenterNode.east)  -- ($(CenterNode)+(\x1,0)-(.5\linew,0)$) ;
%      \path let \p1=(innersep) in
%            (CenterNode.east)  edge
%            node[align=center,rotate=90,framed,fill=white,draw=white]{innerrightmargin}  
%            ($(CenterNode)+(\x1,0)-(.5\linew,0)$) ;
%      
%      
%      %%leftmargin
%      \draw[draw=black,triangle 45-triangle 45]let \p1=(innersep) , \p2=(outersep) in
%            ($(CenterNode)-(\x1,0)-(2.5\linew,0)$)  -- ($(CenterNode)-(\x1,0)-(\x2,0)$) ;
%      \path let \p1=(innersep) , \p2=(outersep) in
%            ($(CenterNode)-(\x1,0)-(2.5\linew,0)$)  edge 
%            node[align=center,rotate=0,framed,fill=white,draw=white]{leftmargin}  
%            ($(CenterNode)-(\x1,0)-(\x2,0)$) ;
%      %%rightmargin
%      \draw[draw=black,triangle 45-triangle 45]let \p1=(innersep) , \p2=(outersep) in
%            ($(CenterNode)+(\x1,0)+(2.5\linew,0)$)  -- ($(CenterNode)+(\x1,0)+(\x2,0)$) ;
%      \path let \p1=(innersep) , \p2=(outersep) in
%            ($(CenterNode)+(\x1,0)+(2.5\linew,0)$)  edge 
%            node[align=center,rotate=0,framed,fill=white,draw=white]{rightmargin}  
%            ($(CenterNode)+(\x1,0)+(\x2,0)$)  ;
%      
%      %%skipabove
%      \draw[draw=black,triangle 45-triangle 45]let \p1=(innersep) , \p2=(outersep) in
%            ($(CenterNode)+(0,\y1)+(0,2.5\linew)$)  -- ($(CenterNode)+(0,\y1)+(0,\y2)$) ;
%      \path let \p1=(innersep) , \p2=(outersep) in
%            ($(CenterNode)+(0,\y1)+(0,2.5\linew)$)  edge 
%            node[align=center,rotate=0,framed,fill=white,draw=white]{skipabove}  
%            ($(CenterNode)+(0,\y1)+(0,\y2)$) ;
%      %%skipbelow
%      \draw[draw=black,triangle 45-triangle 45]let \p1=(innersep) , \p2=(outersep) in
%            ($(CenterNode)-(0,\y1)-(0,2.5\linew)$)  -- ($(CenterNode)-(0,\y1)-(0,\y2)$) ;
%      \path let \p1=(innersep) , \p2=(outersep) in
%            ($(CenterNode)-(0,\y1)-(0,2.5\linew)$)  edge 
%            node[align=center,rotate=0,framed,fill=white,draw=white]{skipbelow}  
%            ($(CenterNode)-(0,\y1)-(0,\y2)$) ;
% \end{tikzpicture}
% 
%  \captionof{figure}{adjustable lengths of \Pack{mdframed}}
%  \label{fig:laengen}
% \end{center}
% 
% \ExplOpt[0pt]{skipabove} Sets an additional skip above the frame.
% \ExplOpt[0pt]{skipbelow} Sets an additional skip below the frame.
% \ExplOpt[]{margin} This option is not longer supported. Use \Opt{leftmargin}
%                            and \Opt{rightmargin} instead.
% \ExplOpt[0pt]{leftmargin} Sets the length of the left margin of the environment.
% \ExplOpt[0pt]{rightmargin} Sets the length of the right margin of the environment.
% 
% \ExplOpt[10pt]{innerleftmargin} Sets the length of the inner left margin of the environment.
% \ExplOpt[10pt]{innerrightmargin} Sets the length of the inner right margin of the environment.
% \ExplOpt[.4\textbackslash baselineskip]{innertopmargin} Sets the length of the inner top margin of the environment.
% \ExplOpt[.4\textbackslash baselineskip]{innerbottommargin} Sets the length of the inner bottom margin of the environment.
% 
% \paragraph*{The following lengths are not shown in figure \eqref{fig:laengen}.}
% 
% \mbox{}
% \ExplOpt[0pt]{userdefinedwidth} Sets the width of the whole \Pack{mdframed} environment. The width represent the width including the line width and the inner margins. The outer margins will be ignored.
% 
% \ExplOpt{outermargin} Sets the length of the outer margin. This option is only available in \texttt{twoside}-mode.
% \ExplOpt{innermargin} Sets the length of the inner margin. This option is only available in \texttt{twoside}-mode.
% 
% 
% \ExplOpt[0pt]{splittopskip} Sets the length of the skip above the split part of the environment.
% \ExplOpt[0pt]{splitbottomskip} Sets the length of the skip below the split part of the environment.
% \ExplOpt[0.4pt]{linewidth} Sets the width of the line around the environment.
% \ExplOpt[0pt]{roundcorner} Sets the size of the radius of the corners of the frames.
%                                    \limitation[TikZ or PSTricks]
% \ExplOpt[0pt]{innerlinewidth} Sets the width of the inner line around the environment.
%                                      \limitation
% \ExplOpt[0pt]{outerlinewidth} Sets the width of the outer line around the environment.
%                                       \limitation
% \ExplOpt[linewidth]{middlelinewidth} Sets the width of the middle line around the environment.\limitation
% 
% 
% \ExplOpt[.2pt]{frametitlerulewidth} Sets the width of the line between the text and the title of \Env{mdframed}.
% \ExplOpt[5pt]{titleaboveskip} Sets the skip of the frame title to the margin above of \Env{mdframed}.
% \ExplOpt[5pt]{titlebelowskip} Sets the skip of the frame title to the rule of the frame title.
% 
% 
% \subsubsection{Colored Options}
% \ExplOpt[black]{linecolor} Sets the color of the line around the environment.
% \ExplOpt[white]{backgroundcolor} Sets the color of the background of the environment.
% \ExplOpt[black]{fontcolor} Sets the color of the contents of the environment.
% \ExplOpt[linecolor]{innerlinecolor} Sets the color of the inner line around the environment.
%                                         \limitation
% \ExplOpt[\newline linecolor]{middlelinecolor} Sets the color of the middle line around the environment. \limitation
% \ExplOpt[\newline linecolor]{outerlinecolor} Sets the color of the outer line around the environment.\limitation
% 
% 
%
% \subsubsection{General options}\label{genopt}
% 
% \ExplOpt[\{\}]{font} Sets the font of the environment.
% \ExplOpt[false]{ntheorem} Before setting this boolean key, you have to load the package \Pack{ntheorem}. 
% 
%                                  With this option you set the values \Cmd{theorempreskipamount}
%                                  and\newline \Cmd{theorempostskipamount} to 0\;pt.
%                                  
% \ExplOpt[false]{nobreak} Sometimes it is useful to prevent a frame from splitting.  The \Opt{nobreak} option is used for this purpose. If you activate this option you can enable it by setting \Opt{nobreak=false}.
% 
% 
% \ExplOpt[0pt]{needspace} Sometimes it is useful to set a minimum height before a frame should be splitted. For such cases you can use \Opt{needspace}. The option requires a length which sets the minimum height before a frame will be splitted.
% 
% \ExplOpt{style} If you define a special style with \Cmd{mdfdefinstyle} you can use the key \Opt{style} to load the style. \Pack{mdframed} has no predefined styles.
% 
% %%\ExplOpt[none]{settings} With version 0.9 of the package \Pack{mdframed} this option was added. This option allows the user to commit some macros. An example is shown in the appendix.
% 
% %\ExplOpt[none]{printheight} This key allows the entries \texttt{none}, \texttt{info} or \texttt{warning}. In relation to the key the vertical space at the beginning of the environment \Pack{mdframed} on the current page will be printed in the \texttt{log} file.
% 
% \ExplOpt[left]{align} Sometimes it is useful to align the environment itself. For this you have the option \Opt{align} which can be set to the following strings: \Opt{left}, \Opt{right} and \Opt{center}. The alignments \Opt{left} or \Opt{right} depend on the given lengths \Opt{leftmargin} and \Opt{rightmargin}. Later I will present an example to demonstrate my bad English explanation.
% 
% \ExplOpt[none]{pstrickssetting} With this key you can pass several options to \Cmd{psset}. For example 
%                                         if you want all lines dashed you will have to set 
%                                         \Opt{pstrickssetting=\{linestyle=dashed\}}. It is very important
%                                         to put the options of \Opt{pstrickssetting} in brackets.
%                                         \limitation[PSTricks]
% \ExplOpt[none]{tikzsetting} With this key you can pass several options to \Cmd{tikzset}. Some examples are listed in the next section. It is very important
%                                         to put the options of \Opt{tikzsetting} in brackets.
%                                         \limitation
%                                         
% \ExplOpt[none]{apptotikzsetting} With this key you can add several options to \Opt{tikzsetting}. This key based on the idea of manipulation of predefined keys of \Pack{mdframed}.
% The package \Pack{mdframed} defines via \Cmd{tikzset} the following keys to draw frames.
% \begin{itemize}
%  \item \Cmd{tikzset\{mdfbox/.style\}}
%  \item \Cmd{tikzset\{mdfcorners./style\}}
%  \item \Cmd{tikzset\{mdfbackground./style\}}
%  \item \Cmd{tikzset\{mdfinnerline./style\}}
%  \item \Cmd{tikzset\{mdfouterline./style\}}
%  \item \Cmd{tikzset\{mdfmiddleline./style\}}
% \end{itemize}
% \textbf{Before you change one please have a look at the file \texttt{md-frame-1.mdf} to see the settings.}\limitation
% 
% \subsection{Hidden Lines}
% 
% 
% \ExplOpt[true]{topline} Draws a line at the top.
% \ExplOpt[true]{bottomline} Draws a line at the bottom.
% \ExplOpt[true]{leftline} Draws a line on the left.
% \ExplOpt[true]{rightline} Draws a line on the right.
% 
% \ExplOpt[false]{hidealllines} With this option you can decide whether all lines should be drawn or not. 
% 
% \subsection{Frametitle}
% 
% 
% \ExplOpt[none]{frametitle} The environment gets a title. To set a title use \Opt{frametitle=\{The Title of the frame\}} as an option of the environment.
% \ExplOpt[\Cmd{bfseries}\Cmd{large}]{frametitleformat} Sets the format of the \Opt{frametitle}.
% \ExplOpt[\Cmd{raggedleft}]{frametitlealignment} Align the frametitle. This option must be set via \Cmd{mdfsetup}.
% \ExplOpt[true]{frametitlerule} Set this key to \Opt{false} to get no line between the frame title and the text.
%
% If a frame title is given the optional length \Opt{innertopmargin} is set between the rule under the frame title and and
% the contents of \Env{mdframed}.
% 
% \subsection{Footnotes}
% Inside the environment you can use the command \Cmd{footnote} as usual. \Pack{mdframed} uses the syntax of environment \Pack{minipage} with the same counter.
% 
% Every footnote text will be collected inside a box and will be displayed at the end of the environment \Pack{mdframed}. 
% 
% \ExplOpt[\mbox{} \Cmd{bigskipamount}]{footnotedistance} The length is the distance between the end of the environement \Pack{mdframed} and the displaying of the \Cmd{footnoterule}.
% 
% \ExplOpt[true]{footnoteinside} The position of the footnotes can be changed with the option \Opt{footnoteinside}. The footnotes will be displayed at the end of the environment but you can decide whether the output is inside \Pack{mdframed} or after.
% 
% \vskip\baselineskip
% \noindent\textbf{Note}\qquad  The ouput of the footnotes with the option \Opt{footnoteinside=false} are not in a splitted frame. I think it isn't useful because the first line of a new page shouldn't be a footnote.
% 
% \section{Examples}
% 
% I outsource the examples in four files to limit the documentation. The files are
% \begin{description}[style=nextline]
% \item[\texttt{mdframed-example-default}] Demonstration of examples created with \Opt{framemethod=default}.
% \item[\texttt{mdframed-example-tikz}] Demonstration of examples created with \Opt{framemethod=TikZ}.
% \item[\texttt{mdframed-example-pstricks}] Demonstration of examples created with \Opt{framemethod=pstricks}.
% \item[\texttt{mdframed-example-texsx}] Demonstration of examples like interaction with \Pack{listings}
% \end{description}
% 
% The examples are often not equivalent but normally they can be adapted to another method.
% 
% \section{Errors, Warnings and Messages}
% 
% The package \Pack{mdframed} provides different errors, warnings and
% messages in the \texttt{log}-file. Some \LaTeX-editors like
% \TeX Maker or \TeX Studio have a special tab for errors and warnings
% but not for messages. So you should look in the \texttt{log-File} itself.
% 
% The followings errors and warnings are generated by \Pack{mdframed}.
% 
% \iffalse
%<*example>
% \fi
\begin{tltxmdfexample}
   The package ...  does not exist but
    needed by mdframed
\end{tltxmdfexample}
% \iffalse
%</example>
% \fi        
% To avoid this problem you should install the required
% packages which are listed in section \ref{sec:syntax}.
% 
% \iffalse
%<*example>
% \fi
\begin{tltxmdfexample}
   package option style is depreciated
   use framemethod instead style
\end{tltxmdfexample}
% \iffalse
%</example>
% \fi    
% With version 0.9d \Pack{mdframed} changed the meaning of the option \Opt{style}. The option is used to load a defined style by \Cmd{mdfdefinestyle}. Instead use \Opt{framemethod} (see section \ref{globopt}). 
% 
% \iffalse
%<*example>
% \fi
\begin{tltxmdfexample}
  Unknown framemethod .... mdframed
\end{tltxmdfexample}
% \iffalse
%</example>
% \fi   
% The input string for the option \Opt{framemethod} is unkown. See section \ref{globopt}.
% 
% \iffalse
%<*example>
% \fi
\begin{tltxmdfexample}
   You have not loaded ntheorem yet
\end{tltxmdfexample}
% \iffalse
%</example>
% \fi   
% To use the option \Opt{ntheorem} you have to 
% load the package \Pack{ntheorem}.
% 
% \iffalse
%<*example>
% \fi
\begin{tltxmdfexample}
  You have only a width of 3cm
\end{tltxmdfexample}
% \iffalse
%</example>
% \fi   
% The package \Pack{mdframed} calculates the width of the contents
% based on the given options. If the width of the contents smaller than
% 3\,cm you will get this warnings. You should change the 
% settings to get a greater width.
% 
% \iffalse
%<*example>
% \fi
\begin{tltxmdfexample}
  You got a bad break
  you have to change it manually
  by changing the text, the space
  or something else
\end{tltxmdfexample}
% \iffalse
%</example>
% \fi 
% Sometimes you have enough vertical space for the rules and the space
% between the rules and the contents but not for the contents itself.
% In this situation you will get this warning because the contents
% of this box is empty. You have the possibility to change the
% settings or include a \Cmd{clearpage} in front of 
% the environment \Pack{mdframed}.
% So far I have no idea how to avoid such things.
% 
% \iffalse
%<*example>
% \fi
\begin{tltxmdfexample}
  You got a bad break
  because the split box is empty
  You have to change the page settings
  like enlargethispage or something else
  You got a bad break
\end{tltxmdfexample}
% \iffalse
%</example>
% \fi 
% See the explanation above.
% 
% \iffalse
%<*example>
% \fi
\begin{tltxmdfexample}
  You got a bad break
  because the last split box is empty
  You have to change the settings
\end{tltxmdfexample}
% \iffalse
%</example>
% \fi 
% The same reason as above but only in the last box.
% 
% 
% \iffalse
%<*example>
% \fi
\begin{tltxmdfexample}
  Option ... is already consumed
  and has no effect on input line ...
\end{tltxmdfexample}
% \iffalse
%</example>
% \fi 
% If you set a global option inside the document
% body you will get this warning.
% 
% \section{Known Problems}
%  In this section I will collect known problems. In case you encounter any further problems, please
%  drop me an email, \href{mailto:marco.daniel@mada-nada.de}{marco.daniel at mada-nada.de}.
% 
%    Do you have any ideas / wishes on further extensions to this package? Please let me know!
% 
% \begin{enumerate}
%  \item So far the environment isn't compatible with the package \Pack{gmverb}.
% \end{enumerate}
% 
% 
% \section{ToDo}
% \begin{enumerate}
%   \item see \enquote{Known Problems}.
%  \item So far it isn't possible to combine the environment  \Cmd{begin\{multicols\}} of the package \Pack{multicol} with \Pack{mdframed} with the whole option list.
%   \item Create new styles.
%   \item Improve page breaks.
%   \item Improve footnotes
%   \item Improve documentation and examples 
%   \item Create styles for \Opt{frametitle}
% \end{enumerate}                
% 
% \section{Previous versions}
% If you have trouble with the new version of \Pack{mdframed} you can load \texttt{mdframedpre} instead. In this way you load the version v0.6 (see \nameref{rev}).
% %With the new version 0.7a I changed the internal commands (I use \Pack{etoolbox}) and the algorithm of the splitting. So I don't know whether I will get new bugs or not. For the last case and the previous version (v0.6) works well you can load \Pack{mdframedpre} instead of \Pack{mdframed}.
% 
% \section{Acknowledgements}
% 
% \vskip\baselineskip
% 
% 
% 
% \textsf{Dick Nickalls;\quad Dietrich Grau;\quad Piazza Luca;\quad Jobst Hoffmann}.
% \medskip
% 
% \noindent Thanks for proofreading 
% 
% \textsf{Alan Munn} and \textsf{Nahid Shajari}                                       
% \clearpage
% \appendix
% \section{More information}
% %\addtocontents{toc}{\protect\setcounter{tocdepth}{-4}}
% In the following section I want to present how to create
% your own frame.
% \subsection{How does \Pack{mdframed} work?}
% With the environment \Cmd{begin\{mdframed\}} \ldots \Cmd{end\{mdframed\}}
% the whole contents will be saved in a \Cmd{savebox} called \Cmd{mdf@splitbox@one}.
% After the calculation of the width and the height of the \Cmd{mdf@splitbox@one} (done by \Pack{mdframed.sty}) the box 
% will be set sequently (done by \Pack{md-frame-X.mdf}). The following figure demonstrates this.
% 
% \begin{center}
% \begin{tikzpicture}
% \newcommand{\mybox}[1]{\parbox{3.5cm}{\centering\small #1}}
% \tikzstyle{every node}=[font=\sffamily]
% \tikzset{mymatrixnodes/.style={%
%      matrix of nodes,
%      nodes={framed,draw=black,fill=yellow!10,},
%      column sep=4.5cm,
%      row sep=1.5cm,
%      }}
% \matrix(m)[mymatrixnodes]
%   {%
%    \mybox{read in and save contents in \Cmd{mdf@splitbox@one}} &
%    \mybox{\Cmd{mdf@putbox@single}} \\
%    \mybox{split \textbackslash mdf@splitbox@one in  \Cmd{mdf@splitbox@two} to the free space of the page} &
%    \mybox{\Cmd{mdf@putbox@first}} \\
%    \mybox{contents of \Cmd{mdf@splitbox@one} have enough space on the new page} &
%    \mybox{\Cmd{mdf@putbox@second}} \\
%    \mybox{split \Cmd{mdf@splitbox@one} in \Cmd{mdf@splitbox@two} to the free space of the page} &
%    \mybox{\Cmd{mdf@putbox@middle}} \\
% };
% 
% \tikzstyle{every path}=[black,-triangle 45,thick,shorten >=4pt,shorten <=4pt]
% \draw  (m-1-1) edge node[align=center,anchor=south]{enough space on page} (m-1-2);
% \draw  (m-2-1) edge node[align=center,anchor=south]{type out \Cmd{mdf@splitbox@two}} (m-2-2);
% \draw  (m-3-1) edge node[align=center,anchor=south]{yes} (m-3-2);
% \draw  (m-4-1) edge node[align=center,anchor=south]{type out \Cmd{mdf@splitbox@two}} (m-4-2);
% 
% \draw  (m-2-2) edge node[align=center,fill=white]%
%        {calculate the height of the new \Cmd{mdf@splitbox@one}} (m-3-1);
% 
% 
% \draw  (m-1-1) edge node[align=center,anchor=west]{not enough space on page} (m-2-1);
% \draw  (m-3-1) edge node[align=center,anchor=west]{no} (m-4-1);
% 
% \draw  let \p1=($(m-4-2.south)-(0,1)$), \p2=($(m-3-1.west)-(0.8,0)$) in
%       (m-4-2.south) -- ($(m-4-2.south)+(0,-1)$) --
%            node[align=center,anchor=north]%
%            {calculate the height of the new \Cmd{mdf@splitbox@one}} (\x2,\y1)
%        -- ($(m-3-1.west)-(0.8,0)$)-- ($(m-3-1.west)$) ;
% 
% 
% \end{tikzpicture}
% 
% \captionof{figure}{Setting the contents of \Pack{mdframed}}\label{fig:way}
% \end{center}
% 
% The width of the contents is the result of the settings of \Opt{leftmargin}, \Opt{rightmargin}, \Opt{linewidth}, \Opt{innerleftmargin} and \Opt{innerrightmargin} (see figure \eqref{fig:laengen}).
% 
% 
% 
% \subsection{The Framecommands}
% The package \Pack{mdframed} knows four kinds of \enquote{Framecommand}. These commands tell \LaTeX\ how to set the contents of \Pack{mdframed}. 
% \begin{description}
%        \item[\Cmd{mdf@putbox@single}] This command sets the contents of a single unsplit frame.
%        \item[\Cmd{mdf@putbox@first}]  This command sets the contents of the first frame of a split frame.
%        \item[\Cmd{mdf@putbox@middle}] This command sets the contents of the middle frame of a split frame.
%        \item[\Cmd{mdf@putbox@second}] This command sets the contents of the last frame of a split frame.
% \end{description}
% 
% Using the explained commands we give an example. The command \Cmd{box} uses the contents of the savebox and types them out. 
% 
% First we want to type out the single box without any settings (but with the calculated width).
% \iffalse
%<*example>
% \fi
\begin{tltxmdfexample}
 \makeatletter
  \def\mdf@putbox@single{\box\mdf@splitbox@one}
 \makeatother
\end{tltxmdfexample}
% \iffalse
%</example>
% \fi
% I am using the command \Cmd{leftline} to start the \enquote{Framecommands} at the left.
% \iffalse
%<*example>
% \fi
\begin{tltxmdfexample}[deletekeywords={leftline},moretexcs={leftline}]
 \makeatletter
  \def\mdf@putbox@single{\leftline{\box\mdf@splitbox@one}}
 \makeatother
\end{tltxmdfexample}
% \iffalse
%</example>
% \fi
% 
% Now you have to know how the lengths are named. Every length which can be modified by the options has the following syntax:
% \iffalse
%<*example>
% \fi
\begin{tltxmdfexample}
 \mdf@<Name of the Length>@length
\end{tltxmdfexample}
% \iffalse
%</example>
% \fi
% For example the leftmargin is:
% \iffalse
%<*example>
% \fi
\begin{tltxmdfexample}
 \mdf@leftmargin@length
\end{tltxmdfexample}
% \iffalse
%</example>
% \fi
% 
% To create only a line at the left with the correct \Opt{leftmargin} you can set \Cmd{mdf@putboxsingle} as follows
% 
% \iffalse
%<*example>
% \fi
\begin{tltxmdfexample}[deletekeywords={leftline},moretexcs={leftline}]
 \makeatletter
 \def\mdf@putbox@single{%
          \leftline{%
              \hspace*{\mdf@leftmargin@length}%
              \rule[-\dp\mdf@splitbox@one]{\mdf@linewidth}%
                   {\ht\mdf@splitbox@one+\dp\mdf@splitbox@one}%
              \box\mdf@splitbox@one
          }%
 }
 \makeatother
\end{tltxmdfexample}
% \iffalse
%</example>
% \fi
% 
% In this way you can do what you want. If you create your own style you
% can save the file as \Pack{md-frame-X.mdf}.
% \Pack{X} must be an integer. 
% In this way you can use the option \Opt{framemethod} to
% load the file by setting \Opt{framemethod=X}.
% 
% \clearpage
% 
% \subsection{Revision history}\label{rev}
% \raggedright
% \minisec{Version 1.0b submitted 9 Dec 2011}
% \begin{itemize*}
% \item fixes documentation (Thanks to Dietrich Grau)
% \item fixes bug in \Cmd{newmdtheoremenv}
% \item fixes bug with overfull boxes (Thanks to Dietrich Grau)
% \item defined \Cmd{newpsstyle}\Opt{mdfbackgroundstyle} and \Opt{mdflinestyle} \limitation[PSTricks]
% \item created dtx-file (Thanks to Kevin Godby)
% \item added \Cmd{@parboxrestore} to \Cmd{mdf@lrbox}
% \end{itemize*}
% 
% \footnotesize 
% \minisec{Version 1.0 submitted 13 Nov 2011}
% \begin{itemize*}
% \item add option \Opt{userdefinedwidth}
% \item add option \Opt{align}
% \item add option \Opt{apptotikzsetting}
% \item create new command \Cmd{mdfapptodefinestyle}
% \item changed internal algorithm
% \item removed calc instead using \eTeX\ \verb+\dimexpr+
% \item expand documentation
% \item trying to fixe problems with \Pack{xcolor}
% \item fixed bug with \Opt{framemethod=pstricks}
% \item create file \Pack{mdframed-example-default}
% \item create file \Pack{mdframed-example-tikz}
% \item create file \Pack{mdframed-example-pstricks}
% \item create file \Pack{mdframed-example-texsx} (texsx stands for tex stackexchange)
% \end{itemize*}
% 
% 
% \minisec{Version 0.9g submitted 08 Oct 2011}
% \begin{itemize*}
% \item fixed documentation
% \item added small footnote compatibility
% \end{itemize*}
% 
% \minisec{Version 0.9f submitted 04 Oct 2011}
% \begin{itemize*}
% \item fixes bugs (thanks to Lars Madsen)
% \item added option \Opt{hidealllines}
% \item fixed documentation
% \end{itemize*}
% 
% 
% \minisec{Version 0.9e submitted 11 Sep 2011}
% \begin{itemize*}
% \item working with \texttt{twoside} modus
% \end{itemize*}
% 
% \minisec{Version 0.9d submitted 10 Sep 2011}
% \begin{itemize*}
% \item \textcolor{red}{changed the meaning of the option \Opt{style}}!!! (inspired by Lars Madsen)
% \item added option \Opt{framemethod} (inspired by Lars Madsen)
% \item added options \Opt{needspace} (inspired by Lars Madsen)
% \item added new command \Cmd{mdfdefinestyle} (inspired by Lars Madsen)
% \item fixes documentation
% \item renamed \Pack{md-frame-3.mdf} to \Pack{md-frame-2.mdf}
% \end{itemize*}
% 
% 
% 
% \minisec{Version 0.9b submitted 7 Sep 2011}
% \begin{itemize*}
% \item fixes bugs in \Cmd{newmdtheoremenv} (Thanks to Enrico Gregorio)
% \end{itemize*}
% 
% \minisec{Version 0.9a submitted 5 Sep 2011}
% \begin{itemize*}
% \item fixes bugs (Thanks to Lars Madson)
% \item expanded documentation (added revision history)
% \end{itemize*}
% 
% 
% \minisec{Version 0.9 submitted 4 Sep 2011}
% \begin{itemize*}
% \item added option \Opt{nobreak}
% \item detecting float environments to prevent split calculation
% \item expand documentation (Thanks to Alan Munn)
% \end{itemize*}
% 
% \minisec{Version 0.8a}
% \begin{itemize*}
% \item fixes bugs
% \item fixes documentation
% \end{itemize*}
% 
% \minisec{Version 0.8 submitted 22 Aug 2011}
% \begin{itemize*}
% \item  added commands: \Cmd{newmdenv}, \Cmd{renewmdenv}, \Cmd{newmdtheoremenv}
% \item fixes bugs
% \item fixes documentation
% \end{itemize*}
% 
% \minisec{Version 0.7a submitted 6 August 2011}
% \begin{itemize*}
% \item added option \Opt{frametitle}
% \item added option \Opt{frametitlefont}
% \item allow twolumn-mode
% \item changed the calculation
% \item added option \Opt{tikzsetting}
% \item added options for hidden lines for all styles
% \item fixes bugs
% \end{itemize*}
% 
% \minisec{Version 0.6a submitted 22 Dec 2010}
% \begin{itemize*}
% \item fixes bugs
% \item added \Cmd{mdfsetup}
% \item expanded documentation
% \end{itemize*}
% 
% 
% 
% \minisec{Version 0.6 submitted 18 Dec 2010}
% \begin{itemize*}
% \item added \Opt{style=3} with \Pack{pstricks}
% \item added option \Opt{pstrickssetting}
% \item added option \Opt{splitbottomskip}
% \item added option \Opt{splittopskip}
% \item added options for hidden lines 
% \item changed the calculation
% \item fixes bugs
% \end{itemize*}
% 
% \minisec{Version 0.4a submitted 14 May 2010}
% \begin{itemize*}
% \item fixes bug in fontcolor
% \end{itemize*}
% 
% \minisec{Version 0.4 submitted 13 May 2010}
% \begin{itemize*}
% \item Elke Schubert creates style file for \Pack{tikz}
% \item fixes some bugs -- calculation of the page dimen (thanks  Dick Nickalls)
% \item using tikz for the frame with different styles
% \end{itemize*}
% 
% \minisec{Version 0.3b submitted 1 May 2010}
% \begin{itemize*}
% \item fixes some bugs -- thanks to Dietrich Grau
% \item added new options: \Opt{ntheorem}
% \end{itemize*}
% 
% \minisec{Version 0.3a submitted 23 Apr 2010}
% \begin{itemize*}
% \item added new options: \Opt{leftmargin} and \Opt{rightmargin}
% \item fixes some bugs
% \end{itemize*}
% 
% \minisec{Version 0.3 submitted16 Apr 2010}
% \begin{itemize*}
% \item first upload to \href{http://dante.ctan.org/upload}{CTAN}
% \end{itemize*}
%
% \ltxmdfappendix
%
% \section{Implementation}\label{implementation}
%
% And finally, here's how it all works\ldots
%
%\StopEventually{^^A
%  \clearpage
%  \PrintChanges^^A
%  \clearpage
%  \PrintIndex^^A
%}
%\iffalse
%    \begin{macrocode}
%<*package>
%    \end{macrocode}
%\fi
%
% \subsection{The Explanation of mdframed.sty}
%
% %$Id: mdframed.dtx 287 2011-12-19 20:12:09Z marco $
% %$Rev: 287 $
% %$Author: marco $
% %$Date: 2011-12-19 21:12:09 +0100 (Mo, 19. Dez 2011) $
% \begin{macro}{mdversion,mdframedpackagename,mdf@maindate@svn}
% Set package information
%    \begin{macrocode}
\def\mdversion{v1.1beta}
\def\mdframedpackagename{mdframed}
\def\mdf@maindate@svn$#1: #2 #3 #4-#5-#6 #7 #8${#4/#5/#6\space }
%    \end{macrocode}
% \end{macro}
%
%    \begin{macrocode}
\NeedsTeXFormat{LaTeX2e}
\ProvidesPackage{mdframed}%
     [\mdf@maindate@svn$Id: mdframed.dtx 287 2011-12-19 20:12:09Z marco $%
      \mdversion: \mdframedpackagename]
%    \end{macrocode}
%
% \begin{macro}{mdf@PackageWarning,mdf@PackageInfo,mdf@LoadFile@IfExist}
%  Set short form of |\PackageWarning|, |\PackageInfo| and |IfFileExists| in combination
%  with |\RequirePackage|.
%    \begin{macrocode}
\newcommand*\mdf@PackageWarning[1]{\PackageWarning{\mdframedpackagename}{#1}}
\newcommand*\mdf@PackageInfo[1]{\PackageInfo{\mdframedpackagename}{#1}}
\newcommand*\mdf@LoadFile@IfExist[1]{%
 \IfFileExists{#1.sty}{%
          \RequirePackage{#1}%
        }{%
        \mdf@PackageWarning{The file #1 does not exist^^J
                           but needed by \mdframedpackagename^^J
                           see documentation fo further information
                           }%
       }
}
%    \end{macrocode}
% \end{macro}
%
% Loading required packages
%    \begin{macrocode}
\RequirePackage{kvoptions}
\RequirePackage{etex}
\RequirePackage{etoolbox}[2011/01/03]
\RequirePackage{zref-abspage}
\RequirePackage{color}
%    \end{macrocode}
%
% Set the family and the prefix of all options. (see documentation of |kvoptions|
%    \begin{macrocode}
\SetupKeyvalOptions{family=mdf,prefix=mdf@}
%    \end{macrocode}

% \begin{macro}{mdf@iflength,mdf@iflength@check,mdf@iflength@check}
% Command which checks the input of length options. If the length option is only 
% a number the |defaultunit| will be used. Syntax: |\mdf@iflength{<Input>}{<length>}{<no length>}|
%    \begin{macrocode}
\newlength{\mdf@templength}
\def\mdf@iflength#1{%
  \afterassignment\mdf@iflength@check%
  \mdf@templength=#1\mdf@defaultunit\relax\relax
  \expandafter\endgroup\next
}
\def\mdf@iflength@check#1{%
  \begingroup
  \ifx\relax#1\@empty
    \def\next{\@secondoftwo}
  \else
    \def\next{\@firstoftwo}
    \expandafter\mdf@iflength@cleanup
  \fi
}
\def\mdf@iflength@cleanup#1\relax{}
%    \end{macrocode}
% \end{macro}
%
% \begin{macro}{mdf@option@length,mdf@define@key@length}
% Command to define a new length width a default value. 
%| \mdf@option@length{<Laengenbezeichnung>}{<Defaultwert>}|
%    \begin{macrocode}
\newrobustcmd*{\mdf@option@length}[2]{%
 \expandafter\newlength\csname mdf@#1@length\endcsname%
 \expandafter\setlength\csname mdf@#1@length\endcsname{#2}%
 }
%    \end{macrocode}
%  Command to create a new length option.|\mdf@define@key@length{<Bezeichnungder Option der Laenge>}|
%    \begin{macrocode}
\newrobustcmd*{\mdf@define@key@length}[1]{%
   \define@key{mdf}{#1}{%
      \def\@tempa{##1}
      \mdf@iflength{\@tempa}%
       {\csxdef{mdfl@#1}{\the\mdf@templength}}%
       {\csxdef{mdfl@#1}{\the\mdf@templength}}%
       \expandafter\setlength\csname mdf@#1@length\endcsname{\csname mdfl@#1\endcsname}%
   }%
}
%    \end{macrocode}
% \end{macro}
%
% \begin{macro}{mdf@dolist}
%  Loop used by \Pack{mdframed}.
%    \begin{macrocode}
\DeclareListParser*{\mdf@dolist}{,}
%    \end{macrocode}
% \end{macro}
%
% \begin{macro}{mdf@do@lengthoption,mdf@lengthoption@doubledo}
% The loop of |\mdf@dolist| expected one argument. So I have to define to commands
% to allow a loop with two arguments. The separation for the input is |==|.
%    \begin{macrocode}
\def\mdf@do@lengthoption#1{%
  \mdf@lengthoption@doubledo#1\@nil%
}
\def\mdf@lengthoption@doubledo#1==#2\@nil{%
   \mdf@option@length{#1}{#2}%
   \mdf@define@key@length{#1}%
}
%    \end{macrocode}
% \end{macro}
%
%
% \begin{macro}{mdf@do@stringoption,mdf@stringoption@doubledo}
% Same as |\mdf@do@lengthoption| and |\mdf@lengthoption@doubledo|.
%    \begin{macrocode}
\def\mdf@do@stringoption#1{%
   \mdf@stringoption@doubledo#1\@nil%
}
\def\mdf@stringoption@doubledo#1==#2\@nil{%
   \expandafter\gdef\csname mdf@#1\endcsname{#2}%
   \define@key{mdf}{#1}{%
      \csdef{mdf@#1}{##1}%
   }%
}
%    \end{macrocode}
% \end{macro}
%
%
% \begin{macro}{mdf@do@booloption,mdf@booloption@doubledo}
% Same as |\mdf@do@lengthoption| and |\mdf@lengthoption@doubledo|.
%    \begin{macrocode}
\def\mdf@do@booloption#1{%
   \mdf@booloption@doubledo#1\@nil%
}
\def\mdf@booloption@doubledo#1==#2\@nil{%
   \newbool{mdf@#1}\setbool{mdf@#1}{#2}%
   \define@key{mdf}{#1}[#2]{%
      \setbool{mdf@#1}{##1}%
   }%
}
%    \end{macrocode}
% \end{macro}
%
%
% \begin{macro}{mdf@do@alignoption,mdf@alignoption@tripledo}
% Same as |\mdf@do@lengthoption| and |\mdf@lengthoption@doubledo|. Here
% three arguments are required.
%    \begin{macrocode}
\def\mdf@do@alignoption#1{%
   \mdf@alignoption@tripledo#1\@nil%
}
\def\mdf@alignoption@tripledo#1==#2==#3\@nil{%
   \csdef{mdf@align@#1@left}{\null\hspace*{#2}}%
   \csdef{mdf@align@#1@right}{\hspace*{#3}\null}%
}
%    \end{macrocode}
% \end{macro}
%
% Start declaration of options
%    \begin{macrocode}
\newcounter{mdf@globalstyle@cnt}
\defcounter{mdf@globalstyle@cnt}{0}
\newcommand*\mdfglobal@style{0}
%    \end{macrocode}
%   Only provide to be backward compatible
%    \begin{macrocode}
\define@key{mdf}{style}{%
        \mdf@PackageWarning{package option style is depreciated^^J
                           use framemethod instead\MessageBreak}%
        \renewcommand*\mdfglobal@style{#1}%
        \defcounter{mdf@globalstyle@cnt}{#1}%
         \ifcase\value{mdf@globalstyle@cnt}\relax
             %0 <- kein Grafikpaket
          \or\mdf@LoadFile@IfExist{tikz}%
          \or\mdf@LoadFile@IfExist{pstricks-add}%
          \or\defcounter{mdf@globalstyle@cnt}{2}%
              \mdf@LoadFile@IfExist{pstricks-add}%
          \or\mdf@LoadFile@IfExist{pstricks-add}%
          \else\mdf@PackageWarning{Unknown global style \value{mdf@globalstyle@cnt}}%
        \fi%
}
%    \end{macrocode}
%
% \begin{macro}{mdf@framemethod}
%    \begin{macrocode}
\providecommand*\mdf@framemethod{}
\def\mdf@framemethod@i{}%
\def\mdf@framemethod@ii{}%
\def\mdf@framemethod@iii{}%
%    \end{macrocode}
% \end{macro}
%
%    \begin{macrocode}
\define@key{mdf}{framemethod}[default]{%
  \lowercase{\def\mdf@tempa{#1}}
  \forcsvlist{\listadd\mdf@framemethod@i}{default,tex,latex,none,0}
  \forcsvlist{\listadd\mdf@framemethod@ii}{pgf,tikz,1}
  \forcsvlist{\listadd\mdf@framemethod@iii}{pstricks,ps,2,postscript}
  \xifinlist{\mdf@tempa}{\mdf@framemethod@i}%
     {\def\mdf@@framemethod{default}\defcounter{mdf@globalstyle@cnt}{0}}%
     {\xifinlist{\mdf@tempa}{\mdf@framemethod@ii}%
         {\def\mdf@@framemethod{tikz}\defcounter{mdf@globalstyle@cnt}{1}}%
         {\xifinlist{\mdf@tempa}{\mdf@framemethod@iii}%
             {\def\mdf@@framemethod{pstricks}\defcounter{mdf@globalstyle@cnt}{2}}%
             {%
              \mdf@LoadFile@IfExist{#1}%
             }%
         }%
     }%
  \ifcase\value{mdf@globalstyle@cnt}\relax%
         %0 <- kein Grafikpaket
     \or\mdf@LoadFile@IfExist{tikz}%
     \or\mdf@LoadFile@IfExist{pstricks-add}%
     \or\mdf@LoadFile@IfExist{pstricks-add}%
  \fi%
}
%    \end{macrocode}
%
% \changes{v1.1beta}{2011/12/17}{Added option \protect\Opt{titleaboveskip}, \protect\Opt{titlebelowskip}, \protect\Opt{frametitlerulewidth}}
% \begin{macro}{mdf@do@lengthoption}
% Here the declaration of the length option. The input method is explained above.
%    \begin{macrocode}
\mdf@dolist{\mdf@do@lengthoption}{%
   {skipabove==\z@},%
   {skipbelow==\z@},%
   {leftmargin==\z@},%
   {rightmargin==\z@},%
   {margin==20pt},%
   {innerleftmargin==10pt},%
   {innerrightmargin==10pt},%
   {innertopmargin==0.4\baselineskip},%
   {innerbottommargin==0.4\baselineskip},%
   {splittopskip==\z@},%
   {splitbottomskip==\z@},%
   {outermargin==\z@},%
   {innermargin==\z@},%
   {linewidth==0.4pt},%
   {innerlinewidth==\z@},%
   {middlelinewidth==\expandafter\mdf@linewidth@length},%
   {outerlinewidth==\z@},%
   {roundcorner==\z@},%
   {footenotedistance==\medskipamount},
   {userdefinedwidth==\linewidth},
   {titleaboveskip==5pt},
   {titlebelowskip==5pt},
   {frametitlerulewidth==.2pt}
} 
%special handling of userdefinedwidth
% \newlength\mdf@userdefinedwidth@length
% \setlength\mdf@userdefinedwidth@length{\linewidth}
% \define@key{mdf}{userdefinedwidth}[\linewidth]{
%    \def\mdf@reserveda{%
%        \def\@tempa{#1}%
%        \mdf@iflength{\@tempa}%
%        {\csxdef{mdfl@userdefinedwidth}{\the\mdf@templength}}%
%        {\csxdef{mdfl@userdefinedwidth}{\the\mdf@templength}}%
%         \setlength{\mdf@userdefinedwidth@length}{\mdfl@userdefinedwidth}%
%    }%
%    \mdf@reserveda
% }
%    \end{macrocode}
% \end{macro}
% \changes{v1.1beta}{2011/12/17}{Added option \protect\Opt{frametitlerulecolor}, \protect\Opt{frametitlebackgroundcolor}, \protect\Opt{font} }
% \begin{macro}{mdf@do@lengthoption}
% Here the declaration of the string option. The input method is explained above.
%    \begin{macrocode}
\mdf@dolist{\mdf@do@stringoption}{%
    {defaultunit==pt},%
    {settings=={}},%
    {linecolor==black},%
    {backgroundcolor==white},%
    {fontcolor==black},%
    {font==\relax},%
    {innerlinecolor==\mdf@linecolor},%
    {outerlinecolor==\mdf@linecolor},%
    {middlelinecolor==\mdf@linecolor},%
    {psroundlinecolor==\mdf@backgroundcolor},%
    {frametitlerulecolor==\mdf@linecolor},
    {frametitlebackgroundcolor==\mdf@backgroundcolor},%
    {frametitle=={}},%
    {frametitlefont==\normalfont\bfseries\large},%
    {printheight==none},%
}
%    \end{macrocode}
% \end{macro}
% \changes{v1.1beta}{2011/12/17}{Added option \protect\Opt{usetwoside}}
% \begin{macro}{mdf@do@booloption}
% Here the declaration of the string option. The input method is explained above.
%    \begin{macrocode}
\mdf@dolist{\mdf@do@booloption}{%
    {ntheorem==false},%
    {topline==true},%
    {leftline==true},%
    {bottomline==true},%
    {rightline==true},%
    {hidealllines==false},%
    {frametitlerule==true},%
    {nobreak==false},%
    {footnoteinside==true},%
    {usetwoside==true},%
}
%    \end{macrocode}
% \end{macro}
%
% \begin{macro}{mdf@do@alignoption}
% Here the declaration of the align option. The input method is explained above.
%    \begin{macrocode}
\mdf@dolist{\mdf@do@alignoption}{%
    {left==\mdf@leftmargin@length==\z@},%
    {center==\fill==\fill},%
    {right==\fill==\mdf@rightmargin@length},%
    {outer==\fill==\mdf@rightmargin@length},%not supported yet
    {outer==\mdf@leftmargin@length==\fill},%not supported yet
}
%    \end{macrocode}
% \end{macro}
%
% \begin{macro}{mdf@align,mdf@makeboxalign@left,mdf@makeboxalign@right,mdf@makeboxalign@right}
% Set the alignment.
%    \begin{macrocode}
\newcommand*\mdf@align{}%
\newcommand*\mdf@makeboxalign@left{\null\hspace*{\mdf@leftmargin@length}}%
\newcommand*\mdf@makeboxalign@right{}%
\define@key{mdf}{align}[left]{%
   \ifcsundef{mdf@align@#1@left}{%
       \mdf@PackageWarning{Unknown alignment #1\MessageBreak}%
       \letcs\mdf@makeboxalign@left{mdf@align@left@left}%
       \letcs\mdf@makeboxalign@right{mdf@align@left@right}%
   }{%
       \def\mdf@makeboxalign@left{\csuse{mdf@align@#1@left}}%
       \def\mdf@makeboxalign@right{\csuse{mdf@align@#1@right}}%
   }%
}
%    \end{macrocode}
% \end{macro}
%
% \changes{v1.1beta}{2011/12/17}{Added option \protect\Opt{frametitlealignment}}
% \begin{macro}{frametitlealignment}
% Set the alignment of the frametitle.
%    \begin{macrocode}
%\def\mdfraggedtitle{\centering}
% \def\mdf@frametitlealignment{}
\DeclareStringOption{frametitlealignment}[\raggedleft]
% \def\mdf@frametitlealignment{\centering}
% \define@key{mdf}{frametitlealignment}[\raggedleft]{%
%   \def\mdf@frametitlealignment{#1}
% %   \let\mdfraggedtitle\@tempa%
% }
%    \end{macrocode}
% \end{macro}
%
%
% \begin{macro}{mdf@tikzset@local,mdf@psset@local}
% Option to pass options to tikz or pstricks
%    \begin{macrocode}
\def\mdf@tikzset@local{\tikzset{tikzsetting/.style={}}}
\define@key{mdf}{tikzsetting}{%
  \def\mdf@tikzset@local{\tikzset{tikzsetting/.style={#1}}}%
}
\define@key{mdf}{apptotikzsetting}{%
  \appto\mdf@tikzset@local{#1}%
}
\def\mdf@psset@local{}
\define@key{mdf}{pstrickssetting}{%
  \def\mdf@psset@local{#1}
}
%    \end{macrocode}
% \end{macro}
%
% \begin{macro}{mdf@xcolor}
% \textcolor{red}{\bfseries Problem width xcolor. This part must be reworked!}
%    \begin{macrocode}
\def\mdf@xcolor{}
\define@key{mdf}{xcolor}[none]{%
  \@ifpackageloaded{xcolor}{%
     \let\mdf@xcolor\@empty %ignoriere die Eingabe der Optionen
     }{%
     \def\@tempa{none}%
      \ifx\mdf@xcolor\@tempa
         \else
         \PassOptionsToPackage{\mdf@xcolor}{xcolor}%
         \RequirePackage{xcolor}%     
      \fi%
   }%
 }
%    \end{macrocode}
% \end{macro}
%
% \begin{macro}{mdf@needspace}
% Defining the option needspace
%    \begin{macrocode}
\define@key{mdf}{needspace}[\z@]{%
     \begingroup%
        \setlength{\dimen@}{#1}%
        \vskip\z@\@plus\dimen@%
        \penalty -100\vskip\z@\@plus -\dimen@%
        \vskip\dimen@%
        \penalty 9999%
        \vskip -\dimen@%
        \vskip\z@skip % hide the previous |\vskip| from |\addvspace|
      \endgroup%
}
%    \end{macrocode}
% \end{macro}
%
%    \begin{macrocode}
\DeclareDefaultOption{%
   \mdf@PackageWarning{Unknown Option '\CurrentOption' for mdframed}}
\ProcessKeyvalOptions*\relax
%    \end{macrocode}
%
% \begin{macro}{mdfsetup}
% Short form of |\setkeys{mdf}|
%    \begin{macrocode}
\newrobustcmd*{\mdfsetup}{\setkeys{mdf}}
%    \end{macrocode}
% \end{macro}
%
% \begin{macro}{mdf@style}
% Redefinition of the option |style| to use the key in combination with |mdfdefinedstyle|.
%    \begin{macrocode}
\define@key{mdf}{style}{%
  \ifcsundef{mdf@definestyle@#1}{%
    \mdf@PackageWarning{Unknown definedstyle #1^^J
                       You have to define a style ^^J
                       via \string\mdfdefinedstyle\MessageBreak
                      }%
   }%
   {\expandafter\expandafter\expandafter\mdfsetup%
    \expandafter\expandafter\expandafter{\csname mdf@definestyle@#1\endcsname}}%
}%
%    \end{macrocode}
% \end{macro}
%
%
% \begin{macro}{mdf@print@space}
%  Option to type out the free vertical space of the current page.
%    \begin{macrocode}
\let\mdf@PackageNoInfo\@gobble
\newrobustcmd*\mdf@ifstrequal@expand{%
\expandafter\ifstrequal\expandafter{\mdf@printheight}%
}
\newrobustcmd*\mdf@print@space{%
  %case "none"
  \mdf@ifstrequal@expand{none}{\def\mdf@tempa{NoInfo}}{%
      %case "info"
      \mdf@ifstrequal@expand{info}{\def\mdf@tempa{Info}}{%
         %case "warning"
         \mdf@ifstrequal@expand{warning}{\def\mdf@tempa{Warning}}{%
            %case "unknown"
            \mdf@PackageWarning{Unknown key for printheight=\mdf@printheight^^J
                               use none, info or warning}%
             \def\mdf@tempa{none}%
         }%
      }%
  }%
\def\mdf@PackageInfoSpace{\csname mdf@Package\mdf@tempa\endcsname}%
}
%    \end{macrocode}
% \end{macro}
% \changes{v1.1beta}{2011/12/15}{Joining all new definitions}
% \changes{v1.1beta}{2011/12/15}{Create new \protect\Cmd{savebox} and renamed \protect\Cmd{@tempboxa}}
% \begin{macro}{new\ldots}
%  Initialize all commands and length which will we used later
%    \begin{macrocode}
\newsavebox\mdf@frametitlebox
\newsavebox\mdf@footnotebox
\newsavebox\mdf@splitbox@one
\newsavebox\mdf@splitbox@two
%Length of splitbox
\newlength\mdfsplitboxwidth
\newlength\mdfsplitboxtotalwidth
\newlength\mdfsplitboxheight
\newlength\mdfsplitboxdepth
\newlength\mdfsplitboxtotalheight
%Length of frame title
\newlength\mdfframetitleboxwidth
\newlength\mdfframetitleboxtotalwidth
\newlength\mdfframetitleboxheight
\newlength\mdfframetitleboxdepth
\newlength\mdfframetitleboxtotalheight
%Length of footnote box
\newlength\mdffootnoteboxwidth
\newlength\mdffootnoteboxtotalwidth
\newlength\mdffootnoteboxheight
\newlength\mdffootnoteboxdepth
\newlength\mdffootnoteboxtotalheight


\newlength\mdfboundingboxwidth
\newlength\mdfboundingboxtotalwidth

\newlength\mdfboundingboxheight
\newlength\mdfboundingboxdepth
\newlength\mdfboundingboxtotalheight

\newlength\mdf@freevspace@length
\newlength\mdf@horizontalwidthofbox@length
\newlength\mdf@verticalmarginwhole@length

% Command to expand the tikz code. (see md-frame-1.mdf)
\newrobustcmd\mdfcreateextratikz{}
%    \end{macrocode}
% \end{macro}
%
% \changes{v1.0b}{2011/11/26}{added command  \protect\Cmd{@parboxrestore} to \protect\Cmd{mdf@lrbox}}
% \begin{macro}{mdf@lrbox,endmdf@lrbox}
% Modification of the default |\lrbox| and |\endlrbox|
%    \begin{macrocode}
\def\mdf@lrbox#1{%
%   \leavevmode%
  \setbox#1\vbox\bgroup
    \color@setgroup%
    \color{\mdf@fontcolor}%
    \mdf@font%
    \hsize=\mdf@horizontalspaceofbox%
    \columnwidth=\hsize%
    \textwidth=\hsize%
    \@parboxrestore%
    \@minipagerestore
    \@setminipage
    \bgroup\mdf@@frametitle\egroup\noindent%
    \ignorespaces
}

\def\endmdf@lrbox{\unskip\color@endgroup\egroup}
%    \end{macrocode}
% \end{macro}
%
% \changes{v1.1beta}{2011/12/16}{added command to avoid overfull box warning by vsplit}
% \begin{macro}{mdf@ignorevbadness,mdf@restorevbadness}
% Avoiding warnings during the splitting process by \Cmd{vsplit}.
% see \href{http://tex.stackexchange.com/questions/38496/how-to-avoid-underfull-vbox-in-combination-with-vsplit/38499#38499}
% {How to avoid underfull vbox in combination with \Cmd{vsplit?}}
%    \begin{macrocode}
\newrobustcmd*\mdf@ignorevbadness{%
   \edef\mdf@currentvbadness{\the\vbadness}%
   \vbadness=\@M%
   \afterassignment\mdf@restorevbadness}
\newrobustcmd*\mdf@restorevbadness{\vbadness=\mdf@currentvbadness\relax}
%    \end{macrocode}
% \end{macro}
%
% \begin{macro}{mdf@trivlist,endmdf@trivlist}
% Modification of the default |\trivlist| and |\endtrivlist|.
%    \begin{macrocode}
\let\mdf@trivlist\trivlist
\let\endmdf@trivlist\endtrivlist
\patchcmd\endmdf@trivlist\@endparenv\mdf@endparenv{}{}
\def\mdf@endparenv{%
  \addpenalty\@endparpenalty\addvspace\mdf@skipbelow@length\@endpetrue}
%    \end{macrocode}
% \end{macro}
%
% \begin{macro}{mdf@makebox@out,mdf@makebox@in}
%    \begin{macrocode}
\newrobustcmd*\mdf@makebox@out[2][\linewidth]{%
 \noindent\hb@xt@\z@{%
    \noindent\makebox[\dimexpr #1\relax][l]{#2}%
 \hss}%
}%
\newrobustcmd*\mdf@makebox@in[2][\mdf@userdefinedwidth@length]{%
 \noindent\makebox[\dimexpr #1\relax][l]{#2}%
}
%    \end{macrocode}
% \end{macro}
%
% \begin{macro}{mdfdefinestyle,mdfapptodefinestyle}
% See explanation of this commands above.
%    \begin{macrocode}
\newrobustcmd*\mdfdefinestyle[2]{%
  \csdef{mdf@definestyle@#1}{#2}%
}
\newrobustcmd*\mdfapptodefinestyle[2]{%
 \ifcsundef{mdf@definestyle@#1}%
   {\mdf@PackageWarning{Unknown style #1}}%
   {\csappto{mdf@definestyle@#1}{,#2}}%
}
%    \end{macrocode}
% \end{macro}
%
% \changes{v1.1beta}{2011/12/15}{Redefinition of \protect\Cmd{newmdtheoremenv}.
%                                -- Now check of theorem definition.}
% \begin{macro}{newmdenv,renewmdenv,newmdtheoremenv}
% Defining of the new environment defintions.
%    \begin{macrocode}
\newrobustcmd*\newmdenv[2][]{%
  \newenvironment{#2}{%
     \mdfsetup{#1}%
     \begin{mdframed}%
    }{%
     \end{mdframed}%
  }%
}
\newrobustcmd*\renewmdenv[2][]{%
  \expandafter\let\csname #2\endcsname\relax%
  \expandafter\let\csname end#2\endcsname\relax%
  \newmdenv[#1]{#2}%
  }%

\newrobustcmd*\newmdtheoremenv[2][]{%
  \BeforeBeginEnvironment{#2}{\begin{mdframed}[#1]}%
  \AfterEndEnvironment{#2}{\end{mdframed}}%
  \newtheorem{#2}%
}
%    \end{macrocode}
% \end{macro}
%
% \changes{v1.1beta}{2011/12/17}{Added method frame title via \protect\Cmd{savebox}}
% \begin{macro}{mdfframedtitleenv,mdf@@frametitle}
% Default definition of the frame tile used by |mdframed|.
%    \begin{macrocode}
\newrobustcmd\mdfframedtitleenv[1]{%
  \setbox\mdf@frametitlebox=\vbox{%
    \begingroup
     \parindent\z@%
     \setlength{\parfillskip}{\z@ plus 1fil}%
     \color@begingroup
      \mdf@frametitlealignment\mdf@frametitlefont%
       {#1}%
     \color@endgroup
    \endgroup
  }%
  \mdf@ignorevbadness
  \global\setbox\mdf@frametitlebox\vbox{\unvbox\mdf@frametitlebox}%
  \global\setlength\mdfframetitleboxheight%
            {\ht\mdf@frametitlebox}
  \global\setlength\mdfframetitleboxdepth%
            {\dp\mdf@frametitlebox}
  \global\setlength\mdfframetitleboxtotalheight%
            {\dimexpr\ht\mdf@frametitlebox+\dp\mdf@frametitlebox
                    +\mdf@titlebelowskip@length+\mdf@titleaboveskip@length\relax}
}

\newrobustcmd*\mdf@frametitlerule{%
 \ifbool{mdf@frametitlerule}
   {\vskip\mdf@titlebelowskip@length
    \mdf@put@frametitlerule
    \vskip\@tempdima%\@afterheading%
   }{\vskip\mdf@titlebelowskip@length
     \vskip\@tempdima
   }%
}
%Styledefinition noch ändern.
%\providecommand*\mdf@put@frametitlerule{}

\newrobustcmd*\mdf@@frametitle{%
  \ifdefempty{\mdf@frametitle}{}%
   {%
    \color@begingroup\@nobreaktrue%
    \mdfframedtitleenv{\mdf@frametitle}%
    \color@endgroup
   }%
}
%    \end{macrocode}
% \end{macro}
%
% \begin{macro}{mdf@checkntheorem}
% Command which checks only |ntheorem|. Later I will support also
% |thmtools|.
%    \begin{macrocode}
\newrobustcmd*\mdf@checkntheorem{%
  \ifbool{mdf@ntheorem}%
    {\ifundef{\theorempreskipamount}%
          {\mdf@PackageWarning{You have not loaded ntheorem yet}}%
          {\setlength{\theorempreskipamount}{\z@}%
           \setlength{\theorempostskipamount}{\z@}%
    }%
  }{}%
}
%    \end{macrocode}
% \end{macro}
%
% \begin{macro}{mdf@footnoterule,mdf@footnoteoutput,mdf@footnoteinput}
% Support for footnotes.
%    \begin{macrocode}
\newrobustcmd*\mdf@footnoterule{%
    \kern0\p@%
    \hrule \@width 1in \kern 2.6\p@}
\newrobustcmd*\mdf@footnoteoutput{%
     \ifvoid\@mpfootins\else
          \nobreak%
          \vskip\mdf@footenotedistance@length%
          \normalcolor%
          \mdf@footnoterule
          \unvbox\@mpfootins
     \fi%
}
\newrobustcmd*\mdf@footnoteinput{%
   \def\@mpfn{mpfootnote}%
   \def\thempfn{\thempfootnote}%
   \c@mpfootnote\z@%
   \let\@footnotetext\@mpfootnotetext%
}
%    \end{macrocode}
% \end{macro}
%
% \begin{macro}{mdf@load@style,mdf@styledefinition}
% Load the method to draw the frame and set style definition.
%    \begin{macrocode}
\newrobustcmd*\mdf@load@style{%
\ifcase\value{mdf@globalstyle@cnt}\relax%
    \input{md-frame-0.mdf}%
 \or\input{md-frame-1.mdf}%
 \or\input{md-frame-2.mdf}%
 \or\input{md-frame-3.mdf}%
 \else%
    \IfFileExists{md-frame-\value{mdf@globalstyle@cnt}.mdf}%
    {\input{md-frame-\value{mdf@globalstyle@cnt}.mdf}}%
    {%
     \input{md-frame-0.mdf}%
     \mdf@PackageWarning{The style number \value{mdf@globalstyle@cnt} does not exist^^J
                        mdframed ues instead style=0 \mdframedpackagename}%
    }%
\fi%
}%
\mdf@load@style

\newrobustcmd*\mdf@styledefinition{%
   \ifnumequal{\value{mdf@globalstyle@cnt}}{0}% 
   {\deflength{\mdf@innerlinewidth@length}{\z@}%
    \deflength{\mdf@middlelinewidth@length}{\mdf@linewidth@length}%
    \deflength{\mdf@outerlinewidth@length}{\z@}%
    \let\mdf@innerlinecolor\mdf@linecolor%
    \let\mdf@middlelinecolor\mdf@linecolor%
    \let\mdf@outerlinecolor\mdf@linecolor%
   }{}%
   \ifnumequal{\value{mdf@globalstyle@cnt}}{2}% 
   {\deflength{\mdf@innerlinewidth@length}{\z@}%
   \deflength{\mdf@middlelinewidth@length}{\mdf@linewidth@length}%
    \deflength{\mdf@outerlinewidth@length}{\z@}%
    \let\mdf@innerlinecolor\mdf@linecolor%
   }{}%
   \ifnumequal{\value{mdf@globalstyle@cnt}}{3}% 
   {\deflength{\mdf@innerlinewidth@length}{\z@}%
   \deflength{\mdf@middlelinewidth@length}{\mdf@linewidth@length}%
    \deflength{\mdf@outerlinewidth@length}{\z@}%
    \let\mdf@innerlinecolor\mdf@linecolor%
   }{}%
}
%    \end{macrocode}
% \end{macro}
% \changes{v1.1beta}{2011/12/17}{Added frametitle detection to \protect\Cmd{detected@mdf@put@frame}}
% \begin{macro}{detected@mdf@put@frame}
% Detect whether inside a non breakable environment.
%    \begin{macrocode}
\let\mdf@reserved@a\@empty
\newrobustcmd*\detected@mdf@put@frame{%
%added frametitle:
  \ifvoid\mdf@frametitlebox\relax%
   \mdf@PackageInfo{No title given\MessageBreak}
  \else
   \@tempdima=\mdf@innertopmargin@length
   \mdf@innertopmargin@length=\mdf@titleaboveskip@length
   \mdf@ignorevbadness
   \setbox\mdf@splitbox@one\vbox{%
    \unvbox\mdf@frametitlebox%
    \mdf@frametitlerule
    \unvbox\mdf@splitbox@one%
   }%
   \mdf@ignorevbadness
   \setbox\mdf@splitbox@one\vbox{%
    \unvbox\mdf@splitbox@one}
  \fi
%test breakable
  \ifmdf@nobreak%Option nobreak=true?
     \def\mdf@reserved@a{\mdf@put@frame@standalone}%
  \else
     \def\mdf@reserved@a{\mdf@put@frame}%
     \ifnum\@floatpenalty<0\relax%Detecting float
        \if@twocolumn%
           \ifx\@captype\@undefined
               \def\mdf@reserved@a{\mdf@put@frame}%
           \else
               \mdf@PackageInfo{mdframed inside float  ^^J
                               mdframed uses option nobreak \mdframedpackagename}%
               \def\mdf@reserved@a{\mdf@put@frame@standalone}%   
           \fi
        \else
           \mdf@PackageInfo{mdframed inside float  ^^J
                           mdframed uses option nobreak \mdframedpackagename}%
           \def\mdf@reserved@a{\mdf@put@frame@standalone}%     
        \fi%
     \fi%
     \if@minipage%
           \mdf@PackageInfo{mdframed inside minipage  ^^J
                           mdframed uses option nobreak \mdframedpackagename}%
           \def\mdf@reserved@a{\mdf@put@frame@standalone}%
     \fi%
     \ifinner%
          \mdf@PackageInfo{mdframed inside a box ^^J
                          mdframed uses option nobreak \mdframedpackagename}%
          \def\mdf@reserved@a{\mdf@put@frame@standalone}%
     \fi%
  \fi%
\mdf@reserved@a%
}
%    \end{macrocode}
% \end{macro}
%
% \begin{macro}{mdf@hidealllines@check}
%    \begin{macrocode}
\newrobustcmd*\mdf@hidealllines@check{%
 \ifbool{mdf@hidealllines}{%
    \boolfalse{mdf@leftline}\boolfalse{mdf@rightline}%
    \boolfalse{mdf@topline}\boolfalse{mdf@bottomline}%
  }{}%
}
%    \end{macrocode}
% \end{macro}
%
% \changes{v1.1beta}{2011/12/16}{Defining \protect\Env{mdframed} with \protect\Cmd{newenvironment}}
% \begin{macro}{mdframed,mdframed@ii,mdframed@i}
% That the user environement.
%    \begin{macrocode}
%\def\mdframed{\@ifnextchar[\mdframed@i\mdframed@ii}%
%\def\mdframed@ii{\mdframed@i[]}%
%\def\mdframed@i[#1]{% default-Umgebung
\newenvironment{mdframed}[1][]{%
\begingroup
   \ifdimless{\linewidth}{\mdf@userdefinedwidth@length}%
     {\deflength\mdf@userdefinedwidth@length{\linewidth}}{}%
   \mdfsetup{#1}%
   \mdf@hidealllines@check%
   \mdf@settings%
   \mdf@twoside@checklength%
   \let\width\z@%
   \let\height\z@%
   \setlength{\topsep}{\mdf@skipabove@length}%
   \mdf@checkntheorem%
   \mdf@styledefinition%
   \mdf@footnoteinput%
   \begingroup%
   \let\partopsep\z@%
   \expandafter\endgroup%   
   \mdf@trivlist\item\relax%
   \mdf@horizontalmargin@equation%
   \mdf@lrbox{\mdf@splitbox@one}%
  }%
%    \end{macrocode}
% \end{macro}
%
%
% \begin{macro}{endmdframed}
%    \begin{macrocode}
%\def\endmdframed{%
  {%
    \ifmdf@footnoteinside%
      \def\mdf@reserveda{%
        \mdf@footnoteoutput%
        \endmdf@lrbox%
        \detected@mdf@put@frame}%
    \else%
      \def\mdf@reserveda{%
        \endmdf@lrbox%
        \detected@mdf@put@frame%
        \mdf@footnoteoutput%
        }%
    \fi%
    \mdf@reserveda%
    \endmdf@trivlist%
    \hrule \@height\z@ \@width\hsize
\endgroup\@endparenv%
}

%    \end{macrocode}
% \end{macro}
%
% \begin{macro}{mdf@twoside@checklength,mdf@zref@label,if@mdf@pageodd,mdf@pageisodd,mdf@pageiseven,mdf@@setzref}
% The whole bunch is used to work width twoside mode and uses the correct margins.
%    \begin{macrocode}
\newtoggle{md:checktwoside}
\settoggle{md:checktwoside}{false}
\newrobustcmd*\mdf@twoside@checklength{%
 \if@twoside
   \ifbool{mdf@usetwoside}%
      {\mdf@PackageInfo{mdframed works in twoside mode}%
       \settoggle{md:checktwoside}{true}%
       \setlength\mdf@rightmargin@length{\mdf@outermargin@length}%
       \setlength\mdf@leftmargin@length{\mdf@innermargin@length}%
      }%
      {\mdf@PackageInfo{mdframed inside twoside mode but^^J
                       works with oneside mode}%
       \settoggle{md:checktwoside}{false}%
      }%
 \fi%
}
\newcounter{mdf@zref@counter}%keine doppelten laebes
\zref@newprop*{mdf@pagevalue}[0]{\number\value{page}}
\zref@addprop{\ZREF@mainlist}{mdf@pagevalue}
\newrobustcmd*\mdf@zref@label{%
   \stepcounter{mdf@zref@counter}
   \zref@label{mdf@pagelabel-\number\value{mdf@zref@counter}}%
}
\newrobustcmd*\if@mdf@pageodd{%
     \zref@refused{mdf@pagelabel-\the\value{mdf@zref@counter}}%
     \ifodd\zref@extract{mdf@pagelabel-\the\value{mdf@zref@counter}}{mdf@pagevalue}%
         \edef\mdf@reserveda{\mdf@pageisodd}%
     \else
        \edef\mdf@reserveda{\mdf@pageiseven}%
     \fi
     \mdf@reserveda%
}
\newrobustcmd*\mdf@pageisodd{%
   \setlength\mdf@rightmargin@length{\mdf@outermargin@length}%
   \setlength\mdf@leftmargin@length{\mdf@innermargin@length}%
}
\newrobustcmd*\mdf@pageiseven{%
   \setlength\mdf@rightmargin@length{\mdf@innermargin@length}%
   \setlength\mdf@leftmargin@length{\mdf@outermargin@length}%
}
\newrobustcmd*\mdf@@setzref{\mdf@zref@label\if@mdf@pageodd}
%    \end{macrocode}
% \end{macro}
%
%
% \begin{macro}{mdf@freepagevspace}
%    \begin{macrocode}
\newrobustcmd*\mdf@freepagevspace{%
     \penalty\@M \vskip 2\baselineskip \vskip\height
     \penalty9999 \vskip -2\baselineskip \vskip-\height
     \penalty9999
     \ifdimequal{\pagegoal}{\maxdimen}%
          {\mdf@freevspace@length\vsize}%
          {\mdf@freevspace@length=\pagegoal\relax%
           \advance\mdf@freevspace@length by -\pagetotal\relax%
          }%
}
%    \end{macrocode}
% \end{macro}
%
%  \changes{v1.1beta}{2011/12/16}{Removing \protect\Cmd{@arrayparboxrestore}}
% \begin{macro}{mdf@advancelength@orizontalmargin@add,mdf@horizontalspaceofbox,mdf@horizontalmargin@equation}
% Width of the box
%    \begin{macrocode}
\newrobustcmd*\mdf@advancelength@orizontalmargin@sub[1]{%
  \advance\mdf@horizontalspaceofbox by -\csname mdf@#1@length\endcsname\relax%
}
\newlength\mdf@horizontalspaceofbox
\newrobustcmd*\mdf@horizontalmargin@equation{%
    \setlength{\mdf@horizontalspaceofbox}{\mdf@userdefinedwidth@length}%
    \mdf@dolist{\mdf@advancelength@orizontalmargin@sub}{%
             leftmargin,outerlinewidth,middlelinewidth,%
             innerlinewidth,innerleftmargin,innerrightmargin,%
             innerlinewidth,middlelinewidth,outerlinewidth,%
             rightmargin}%
    \notbool{mdf@leftline}{%
                \advance\mdf@horizontalspaceofbox by \mdf@innerlinewidth@length\relax%
                \advance\mdf@horizontalspaceofbox by \mdf@middlelinewidth@length\relax%
                \advance\mdf@horizontalspaceofbox by \mdf@outerlinewidth@length\relax%
           }{}%      
    \notbool{mdf@rightline}{%
                \advance\mdf@horizontalspaceofbox by \mdf@innerlinewidth@length\relax%
                \advance\mdf@horizontalspaceofbox by \mdf@middlelinewidth@length\relax%
                \advance\mdf@horizontalspaceofbox by \mdf@outerlinewidth@length\relax% 
           }{}%
    \ifdimless{\mdf@horizontalspaceofbox}{3cm}%
              {\mdf@PackageWarning{You have only a width of 3cm}}{}
    \hsize=\mdf@horizontalspaceofbox%
}
%    \end{macrocode}
% \end{macro}
%
% \begin{macro}{mdf@keeplines@single}
% horizontal space in relation of the lines.
%    \begin{macrocode}
\newrobustcmd*\mdf@keeplines@single{%
  \notbool{mdf@topline}{%
      \advance\mdf@verticalmarginwhole@length by -\mdf@middlelinewidth@length%
     }{}%
  \notbool{mdf@bottomline}{%
      \advance\mdf@verticalmarginwhole@length by -\mdf@middlelinewidth@length%
     }{}%
}
%    \end{macrocode}
% \end{macro}
%
%
% \begin{macro}{mdf@advancelength@verticalmarginwhole,mdf@advancelength@freevspace@sub,mdf@advancelength@freevspace@add}
% Loop macros to calculate the height. Used by |\mdf@dolist|.
%    \begin{macrocode}
\newrobustcmd*\mdf@advancelength@verticalmarginwhole[1]{%
  \advance\mdf@verticalmarginwhole@length by \csname mdf@#1@length\endcsname\relax%
}
\newrobustcmd*\mdf@advancelength@freevspace@sub[1]{%
  \advance\dimen@ by -\csname mdf@#1@length\endcsname\relax%
}
\newrobustcmd*\mdf@advancelength@freevspace@add[1]{%
  \advance\dimen@ by \csname mdf@#1@length\endcsname\relax%
}
%    \end{macrocode}
% \end{macro}
%
%
% \begin{macro}{mdf@reset}
% Reset changes
%    \begin{macrocode}
\protected@edef\mdf@reset{\boxmaxdepth\the\boxmaxdepth 
                          \splittopskip\the\splittopskip}%
%    \end{macrocode}
% \end{macro}
%
% \begin{macro}{mdf@put@frame@standalone}
% Output of |mdframed| inside a non breakable environement.
%    \begin{macrocode}
\newrobustcmd*\mdf@put@frame@standalone{\relax%
   \ifvoid\mdf@splitbox@one\relax
      \mdf@PackageWarning{The environment is empty\MessageBreak}%
      \let\mdf@reserved@a\relax%
   \else
      %Hier berechnung Box-Inhalt+Rahmen oben und unten
      \setlength{\mdf@verticalmarginwhole@length}%
                 {\dimexpr\ht\mdf@splitbox@one+\dp\mdf@splitbox@one\relax}%
      \mdf@dolist{\mdf@advancelength@verticalmarginwhole}{%
                  outerlinewidth,middlelinewidth,innerlinewidth,innertopmargin,
                  innerbottommargin,innerlinewidth,middlelinewidth,outerlinewidth}%
      \mdf@keeplines@single%
      \def\mdf@reserved@a{\mdf@putbox@single}%
   \fi
   \mdf@reserved@a%
}
%    \end{macrocode}
% \end{macro}
%
% \begin{macro}{mdf@put@frame}
% Output of |mdframed| inside a breakable environement.
% The comparison are onyl check whether the contents must be split or not.
%    \begin{macrocode}
\def\mdf@put@frame{\relax%
\ifvoid\mdf@splitbox@one\relax
\mdf@PackageWarning{The environment is empty\MessageBreak}%
\let\mdf@reserved@a\relax%
\else
  \mdf@print@space%
  \mdf@freepagevspace%
  \mdf@PackageInfoSpace{\the\mdf@freevspace@length before the beginning of ^^J
                       the environment ending on input line \MessageBreak}%
   \ifdimless{\mdf@freevspace@length}{2\baselineskip}
             {\mdf@PackageInfo{Not enough space on this page}
              \vfill\eject%
              \def\mdf@reserved@a{\mdf@put@frame}%
             }{%
               %Hier berechnung Box-Inhalt+Rahmen oben und unten
              \setlength{\mdf@verticalmarginwhole@length}%
                          {\dimexpr\ht\mdf@splitbox@one+\dp\mdf@splitbox@one\relax}%
              \mdf@dolist{\mdf@advancelength@verticalmarginwhole}{%
                     outerlinewidth,middlelinewidth,innerlinewidth,innertopmargin,
                     innerbottommargin,innerlinewidth,middlelinewidth,outerlinewidth}%
             \mdf@keeplines@single%
             \ifdimless{\mdf@verticalmarginwhole@length}{\mdf@freevspace@length}%
                {%passt auf Seite%
                  \begingroup
                   \iftoggle{md:checktwoside}{\mdf@@setzref}{}%
                    \mdf@putbox@single
                  \endgroup
                 \let\mdf@reserved@a\relax}%
                {\def\mdf@reserved@a{\mdf@put@frame@i}}%passt nicht auf Seite
             }%
\fi
\mdf@reserved@a%
}
%    \end{macrocode}
% \end{macro}
%
%  \changes{v1.0b}{2011/12/09}{removed \protect\Cmd{setbox\textbackslash mdf@splitbox@two \textbackslash vbox{\textbackslash unvbox \textbackslash mdf@splitbox@two}}}
% \begin{macro}{mdf@put@frame@i}
% Output of the first splitted box.
%    \begin{macrocode}
\def\mdf@put@frame@i{%Box muss gesplittet werden -- Ausgabe der ersten Teilbox
      %Berechnung der Splittgroesse -- Linien und Abstand oben
      \mdf@freepagevspace%
      \dimen@=\the\mdf@freevspace@length%
      \dimen@i=\mdf@innertopmargin@length%
      \advance\dimen@i by \mdf@linewidth@length%
      \advance\dimen@i by 2\baselineskip%
      \ifdim\dimen@<\dimen@i\relax
           \hrule \@height\z@ \@width\hsize%
           \vfill\eject%
           \def\mdf@reserved@a{\mdf@put@frame}%
      \else%
      \mdf@dolist{\mdf@advancelength@freevspace@sub}{%
                outerlinewidth,middlelinewidth,innerlinewidth,%
                innertopmargin,splitbottomskip}%
      \ifbool{mdf@topline}{}{\advance\dimen@ by \mdf@middlelinewidth@length}%
      \ifdimless{\ht\mdf@splitbox@one+\dp\mdf@splitbox@one}{\dimen@}%
         {\mdf@PackageWarning{You got a bad break\MessageBreak
                             you have to change it manually\MessageBreak
                             by changing the text, the space\MessageBreak
                             or something else}%
          \advance\dimen@ by -1.8\baselineskip\relax%
         }{}%
         \advance\dimen@ by -1pt\relax%Box darf nicht zu Groß werden.
         \splitmaxdepth\z@ \splittopskip\mdf@splittopskip@length%
         \mdf@ignorevbadness%
         \setbox\mdf@splitbox@two\vsplit\mdf@splitbox@one to \dimen@
         \setbox\mdf@splitbox@two\vbox{\unvbox\mdf@splitbox@two}%needed?
         \ifdimgreater{\ht\mdf@splitbox@two+\dp\mdf@splitbox@two}{\dimen@}{%Falsch gesplittet
            \mdf@PackageInfo{Box was splittet wrong\MessageBreak}%
             \dimen@i=\dimen@
              \advance\dimen@ by -\ht\mdf@splitbox@two
              \advance\dimen@ by -\dp\mdf@splitbox@two
              \advance\dimen@i by 0.5\dimen@
             \splittopskip\z@%
             \setbox\mdf@splitbox@one\vbox{\unvbox\mdf@splitbox@two%
                                     %benoetigt um Tiefe zu haben
                                     \hrule \@height\dp\strutbox \@width\z@
                                     \unvbox\mdf@splitbox@one}
             \splittopskip\mdf@splittopskip@length%
             \mdf@ignorevbadness%
             \setbox\mdf@splitbox@two\vsplit\mdf@splitbox@one to \dimen@i
             \setbox\mdf@splitbox@two\vbox{\unvbox\mdf@splitbox@two}%
             }{}%
         \setbox\mdf@splitbox@one\vbox{\unvbox\mdf@splitbox@one}%PRUEFEN!!!!
         \ifvoid\mdf@splitbox@one
           \mdf@PackageWarning{You got a bad break\MessageBreak
                               because the splittet box is empty\MessageBreak
                               You have to change the page settings\MessageBreak
                               like enlargethispage or something else}%
           \setbox\mdf@splitbox@one\vbox{\box\mdf@splitbox@two\box\mdf@splitbox@one}%
           \def\mdf@reserved@a{\mdf@put@frame}%
         \fi
         \ifvoid\mdf@splitbox@two%%pruefe, ob erste Box leer ist
          \hrule \@height\z@ \@width\hsize
          \vfill\eject%  
             \def\mdf@reserved@a{\mdf@put@frame}%
         \else
          \ifdimequal{\ht\mdf@splitbox@two}{0pt}%
            {\hrule \@height\z@ \@width\hsize%
             \vfill\eject%
             \setbox\mdf@splitbox@one\vbox{\unvbox\mdf@splitbox@two\unvbox\mdf@splitbox@one}
             \def\mdf@reserved@a{\mdf@put@frame}%
            }%
            {%
            \begingroup
               \iftoggle{md:checktwoside}{\mdf@@setzref}{}%
               \mdf@putbox@first%%%Groesse des Splittens passt
            \endgroup
            \hrule \@height\z@ \@width\hsize
            \vfill\eject%
            \def\mdf@reserved@a{\mdf@put@frame@ii}%
            }%
         \fi% 
      \fi%
\mdf@reserved@a%
}
%    \end{macrocode}
% \end{macro}
%
%
% \begin{macro}{mdf@put@frame@ii}
% Output of the middle and last box.
%    \begin{macrocode}
\def\mdf@put@frame@ii{%Ausgabe der mittleren Box(en) wenn vorhanden
  \setlength{\mdf@freevspace@length}{\vsize}%
  \setlength{\dimen@}{\dimexpr\ht\mdf@splitbox@one+\dp\mdf@splitbox@one\relax}%
  \mdf@dolist{\mdf@advancelength@freevspace@add}{%
                outerlinewidth,middlelinewidth,innerlinewidth,%
                innerbottommargin}%%%Addition der Linien unten
   \ifbool{mdf@bottomline}{}{\advance\dimen@ by -\mdf@middlelinewidth@length\relax}%
   \ifdimgreater{\dimen@}{\mdf@freevspace@length}%
    {%
        \advance\mdf@freevspace@length by -\mdf@splitbottomskip@length
        \splitmaxdepth\z@ \splittopskip\mdf@splittopskip@length%
        \mdf@ignorevbadness%
        \setbox\mdf@splitbox@two\vsplit\mdf@splitbox@one to \mdf@freevspace@length%
        \setbox\mdf@splitbox@two\vbox{\unvbox\mdf@splitbox@two}%PRUEFEN!!!
 %       \setbox\mdf@splitbox@one\vbox{\unvbox\mdf@splitbox@one}%PRUEFEN!!!!
        \ifvoid\mdf@splitbox@one\relax%
           \mdf@PackageWarning{You got a bad break\MessageBreak
                               because the split box is empty\MessageBreak
                               You have to change the settings}%
         \fi%
        \begingroup
           \iftoggle{md:checktwoside}{\mdf@@setzref}{}%
           \mdf@putbox@middle%
        \endgroup
        \hrule \@height\z@ \@width\hsize
        \vfill\eject
        \def\mdf@reserved@a{\mdf@put@frame@ii}%
     }%Hier die Ausgabe der mittleren Box
     {\ifvoid\mdf@splitbox@one
           \mdf@PackageWarning{You got a bad break\MessageBreak
                               because the last split box is empty\MessageBreak
                               You have to change the settings}%
      \fi%
      \begingroup
          \iftoggle{md:checktwoside}{\mdf@@setzref}{}%
          \mdf@putbox@second%
      \endgroup
      \let\mdf@reserved@a\relax%
     }%Hier kommt die Ausgabe der letzten Box
  \mdf@reserved@a%
}
%    \end{macrocode}
% \end{macro}
%
%
%    \begin{macrocode}
\DisableKeyvalOption[action=warning,package=mdframed]{mdf}{framemethod}%
\DisableKeyvalOption[action=warning,package=mdframed]{mdf}{xcolor}%

 \endinput
%    \end{macrocode}
%
%\iffalse
%    \begin{macrocode}
%</package>
%    \end{macrocode}
%\fi
%
%\iffalse
%    \begin{macrocode}
%<*mdframedefault>
%    \end{macrocode}
%\fi
% \subsection{The Explanation of md-frame-0.mdf}
%    \begin{macrocode}
%% Style file for mdframed for package option `framemethod=default`
%% 
%% This package may be distributed under the terms of the LaTeX Project
%% Public License, as described in lppl.txt in the base LaTeX distribution.
%% Either version 1.0 or, at your option, any later version.

%%$Id: mdframed.dtx 287 2011-12-19 20:12:09Z marco $
%
%    \end{macrocode}
%
% \begin{macro}{mdframedOpackagename,mdf@frameOdate@svn}
% local settings
%    \begin{macrocode}
\def\mdframedOpackagename{md-frame-0}
\def\mdf@frameOdate@svn$#1: #2 #3 #4-#5-#6 #7 #8${#4/#5/#6\space }
\ProvidesFile{md-frame-0.mdf}%
    [\mdf@frameOdate@svn$Id: mdframed.dtx 287 2011-12-19 20:12:09Z marco $%
     \mdversion: \mdframedOpackagename]
%    \end{macrocode}
% \end{macro}
%
%
% \begin{macro}{mdframedOpackagename,mdf@frameOdate@svn}
% short command
%    \begin{macrocode}
\def\mdf@background@default{\color{\mdf@backgroundcolor}}
\def\mdf@linecolor@default{\color{\mdf@middlelinecolor}}
\def\mdf@titlebackground@default{\color{\mdf@frametitlebackgroundcolor}}
%    \end{macrocode}
% \end{macro}
%
%
% \begin{macro}{mdf@put@frametitlerule}
% Draw the title line -- This command is for every output.
%    \begin{macrocode}
\newrobustcmd*\mdf@put@frametitlerule{%
     \noindent\rlap{\color{\mdf@frametitlerulecolor}%
         \hspace*{-\mdf@innerleftmargin@length}%
         \rule{\dimexpr\mdf@innerleftmargin@length+\hsize+\mdf@innerrightmargin@length}%
              {\mdf@frametitlerulewidth@length}%
         }%
}%
%    \end{macrocode}
% \end{macro}
%
%
%
%
% \begin{macro}{mdf@putbox@single,mdf@frame@background@single,mdf@frame@topandbottomline@single,mdf@frame@leftline@single,mdf@frame@rightline@single,mdf@frame@rightline@single}
% The frame of of a non splitted contents of mdframed
%    \begin{macrocode}
\def\mdf@frame@background@single{%
  \rlap{\mdf@background@default%
     \rule[-\mdfboundingboxdepth]%
          {\mdfboundingboxtotalwidth}%
          {\mdfboundingboxtotalheight}%
  }%
}%

\def\mdf@frame@titlebackground@single{%
\ifdefempty{\mdf@frametitle}{}{%
  \rlap{\mdf@titlebackground@default%
     \rule[\dimexpr-\mdfboundingboxdepth+\mdfboundingboxtotalheight-\mdfframetitleboxtotalheight\relax]%
          {\mdfboundingboxtotalwidth}%
          {\mdfframetitleboxtotalheight}%
  }%
  }%
}%

%
\def\mdf@frame@topandbottomline@single{%
  \rlap{\mdf@linecolor@default%
     \ifbool{mdf@topline}{%
          \rule[\dimexpr\mdfboundingboxheight-\mdfboundingboxdepth%
                       +\mdf@innerbottommargin@length+\mdf@innertopmargin@length\relax]%
               {\mdfboundingboxtotalwidth}%
               {\mdf@middlelinewidth@length}}%
         {}%
  }%
  \rlap{\mdf@linecolor@default%
     \ifbool{mdf@bottomline}{%
         \rule[\dimexpr-\mdfboundingboxdepth-\mdf@middlelinewidth@length\relax]%
              {\mdfboundingboxtotalwidth}%
              {\mdf@middlelinewidth@length}}%
         {}%
  }% 
}
%
\def\mdf@frame@leftline@single{%
  \llap{\mdf@linecolor@default%
     \rule[\dimexpr-\mdfboundingboxdepth%
           \ifbool{mdf@bottomline}{-\mdf@middlelinewidth@length}{}\relax]%
          {\mdf@middlelinewidth@length}%
          {\dimexpr\mdfboundingboxtotalheight%
           +\ifbool{mdf@bottomline}{\mdf@middlelinewidth@length}{0pt}%
           +\ifbool{mdf@topline}{\mdf@middlelinewidth@length}{0pt}\relax}%
  }% 
}%
%
\def\mdf@frame@rightline@single{%
  \rlap{\mdf@linecolor@default%
     \hspace*{\mdfboundingboxwidth}%
     \hspace*{\mdf@innerrightmargin@length}%
     \rule[\dimexpr-\mdfboundingboxdepth%
           \ifbool{mdf@bottomline}{-\mdf@middlelinewidth@length}{}\relax]%
          {\mdf@middlelinewidth@length}%
          {\dimexpr\mdfboundingboxtotalheight%
           +\ifbool{mdf@bottomline}{\mdf@middlelinewidth@length}{0pt}%
           +\ifbool{mdf@topline}{\mdf@middlelinewidth@length}{0pt}\relax}%
  }%
}%
%
\def\mdf@putbox@single{%%%%% Ausgabe der ungesplitteten Gesamtbox
  \ifvoid\mdf@splitbox@one
  \else%
    \mdf@makebox@out{%
      \mdf@makeboxalign@left%
      \setlength{\mdfboundingboxwidth}%
                   {\wd\mdf@splitbox@one}%
      \setlength{\mdfboundingboxtotalwidth}%
                   {\dimexpr\mdfboundingboxwidth+\mdf@innerleftmargin@length%
                    +\mdf@innerrightmargin@length\relax}%
      \setlength{\mdfboundingboxheight}%
                   {\dimexpr\ht\mdf@splitbox@one+\dp\mdf@splitbox@one\relax}%
      \setlength{\mdfboundingboxdepth}%
                   {\dimexpr\dp\mdf@splitbox@one+\mdf@innerbottommargin@length\relax}%
      \setlength{\mdfboundingboxtotalheight}%
                   {\dimexpr\mdfboundingboxheight+\mdf@innertopmargin@length%
                    +\mdf@innerbottommargin@length\relax}%
      \noindent%
      \setlength{\@tempdima}{\dimexpr\mdfboundingboxtotalwidth%
                             +\ifbool{mdf@leftline}%
                                     {\mdf@middlelinewidth@length}{\z@}%
                             +\ifbool{mdf@rightline}%
                                     {\mdf@middlelinewidth@length}{\z@}\relax}%
      \mdf@makebox@in[\@tempdima]{%
        \null%
        \ifbool{mdf@leftline}{%
           \hspace*{\mdf@middlelinewidth@length}%
           \mdf@frame@leftline@single%
            }{}%
        \mdf@frame@topandbottomline@single%
        \mdf@frame@background@single%
        \mdf@frame@titlebackground@single%
        \hspace*{\mdf@innerleftmargin@length}%
        \ifbool{mdf@rightline}{%
           \mdf@frame@rightline@single%
         }{}%
        {\box\mdf@splitbox@one}%
    }%
    \mdf@makeboxalign@right%
  }%
  \fi%
}
%    \end{macrocode}
% \end{macro}
%
%
% \begin{macro}{mdf@putbox@first,mdf@frame@background@first,mdf@frame@leftline@first,mdf@frame@topline@first,mdf@frame@rightline@first}
% The first frame of of a splitted contents of mdframed
%    \begin{macrocode}
\def\mdf@frame@background@first{%
  \rlap{\mdf@background@default%
     \rule[-\mdfboundingboxdepth]%
          {\mdfboundingboxtotalwidth}%
          {\mdfboundingboxtotalheight}%
  }%
}%
%
\def\mdf@frame@titlebackground@first{%
\ifdefempty{\mdf@frametitle}{}{%
  \ifdimgreater{\mdfboundingboxtotalheight}{\mdfframetitleboxtotalheight}{%
   \rlap{\mdf@titlebackground@default%
     \rule[\dimexpr-\mdfboundingboxdepth+\mdfboundingboxtotalheight-\mdfframetitleboxtotalheight\relax]%
          {\mdfboundingboxtotalwidth}%
          {\mdfframetitleboxtotalheight}%
   }%
   \global\setlength{\mdfframetitleboxtotalheight}{\z@}%
  }{%
   \rlap{\mdf@titlebackground@default%
     \rule[-\mdfboundingboxdepth]%
          {\mdfboundingboxtotalwidth}%
          {\mdfboundingboxtotalheight}%
   \global\setlength{\mdfframetitleboxtotalheight}%
          {\dimexpr\mdfframetitleboxtotalheight%
                  -\mdfboundingboxtotalheight%
                  -\mdf@middlelinewidth@length\relax}%
   }%
  }%
 }%
}%
%
\def\mdf@frame@leftline@first{%
  \llap{\mdf@linecolor@default%
     \rule[-\mdfboundingboxdepth]%
          {\mdf@middlelinewidth@length}%
          {\dimexpr\mdfboundingboxtotalheight%
            +\ifbool{mdf@topline}{\mdf@middlelinewidth@length}{0pt}\relax}%
  }%
}%
%
\def\mdf@frame@topline@first{%
  \rlap{\mdf@linecolor@default%
     \rule[\dimexpr\mdfboundingboxheight-\mdfboundingboxdepth+%
            \mdf@splitbottomskip@length+\mdf@innertopmargin@length\relax]%
          {\mdfboundingboxtotalwidth}%
          {\mdf@middlelinewidth@length}%
  }%
}
%
\def\mdf@frame@rightline@first{%
  \rlap{\mdf@linecolor@default\hspace*{\mdfboundingboxwidth}%
     \hspace*{\mdf@innerrightmargin@length}%
     \rule[-\mdfboundingboxdepth]%
          {\mdf@middlelinewidth@length}%
          {\dimexpr\mdfboundingboxtotalheight%
            +\ifbool{mdf@topline}{\mdf@middlelinewidth@length}{0pt}\relax}%
  }%
}%
%
\def\mdf@putbox@first{%%%% Ausgabe der Teilbox 1
  \ifvoid\mdf@splitbox@one
  \else%
    \mdf@makebox@out[\linewidth]{%
      \mdf@makeboxalign@left%
      \setlength{\mdfboundingboxwidth}{\wd\mdf@splitbox@two}%
      \setlength{\mdfboundingboxtotalwidth}%
                   {\dimexpr\mdfboundingboxwidth+\mdf@innerleftmargin@length%
                            +\mdf@innerrightmargin@length\relax}%
      \setlength{\mdfboundingboxheight}{\dimexpr\ht\mdf@splitbox@two+\dp\mdf@splitbox@two\relax}%
      \setlength{\mdfboundingboxdepth}%
                   {\dimexpr\dp\mdf@splitbox@two+\mdf@splitbottomskip@length\relax}%
      \setlength{\mdfboundingboxtotalheight}%
                   {\dimexpr\mdfboundingboxheight+\mdf@innertopmargin@length%
                           +\mdf@splitbottomskip@length\relax}%
      \setlength{\@tempdima}%
                   {\dimexpr\mdfboundingboxtotalwidth%
                           +\ifbool{mdf@leftline}{\mdf@middlelinewidth@length}{\z@}%
                           +\ifbool{mdf@rightline}{\mdf@middlelinewidth@length}{\z@}%
                    \relax}%
      \mdf@makebox@in[\@tempdima]{%
        \null%
        \ifbool{mdf@leftline}{%
           \hspace*{\mdf@middlelinewidth@length}%
           \mdf@frame@leftline@first}{}%
        \ifbool{mdf@topline}{%
            \mdf@frame@topline@first}{}%
        \mdf@frame@background@first%
        \mdf@frame@titlebackground@first%
        \hspace*{\mdf@innerleftmargin@length}%
        \ifbool{mdf@rightline}{%
            \mdf@frame@rightline@first}{}%
        {\box\mdf@splitbox@two}%
    }%
    \mdf@makeboxalign@right%
  }%
 \fi%
}
%    \end{macrocode}
% \end{macro}
%
%
% \begin{macro}{mdf@putbox@second,mdf@frame@background@second,mdf@frame@leftline@second,mdf@frame@bottomline@second,mdf@frame@rightline@second}
% The last frame of of a splitted contents of mdframed
%    \begin{macrocode}
\def\mdf@frame@background@second{%
  \rlap{\mdf@background@default%
     \rule[-\mdfboundingboxdepth]%
          {\mdfboundingboxtotalwidth}%
          {\mdfboundingboxtotalheight}%
  }%
}%
%
\def\mdf@frame@titlebackground@second{%
\ifdefempty{\mdf@frametitle}{}{%
  \ifdimgreater{\mdfframetitleboxtotalheight}{\z@}{%
   \rlap{\mdf@titlebackground@default%
     \rule[\dimexpr-\mdfboundingboxdepth+\mdfboundingboxtotalheight-\mdfframetitleboxtotalheight\relax]%
          {\mdfboundingboxtotalwidth}%
          {\mdfframetitleboxtotalheight}%
   }%
   \global\setlength{\mdfframetitleboxtotalheight}{\z@}%
  }{}%
 }%
}%
%
\def\mdf@frame@leftline@second{%
  \llap{\mdf@linecolor@default%
     \rule[\dimexpr-\mdfboundingboxdepth%
            -\ifbool{mdf@bottomline}{\mdf@middlelinewidth@length}{0pt}\relax]%
          {\mdf@middlelinewidth@length}%
          {\dimexpr\mdfboundingboxtotalheight%
              +\ifbool{mdf@bottomline}{\mdf@middlelinewidth@length}{0pt}\relax}%
  }%     
}%
%
\def\mdf@frame@bottomline@second{%
  \rlap{\mdf@linecolor@default%
     \rule[\dimexpr-\mdfboundingboxdepth-\mdf@middlelinewidth@length\relax]%
          {\mdfboundingboxtotalwidth}%
          {\mdf@middlelinewidth@length}%
  }%
}%
%
\def\mdf@frame@rightline@second{%
  \rlap{\mdf@linecolor@default\hspace*{\mdfboundingboxwidth}%
     \hspace*{\mdf@innerrightmargin@length}%
     \rule[\dimexpr-\mdfboundingboxdepth%
            -\ifbool{mdf@bottomline}{\mdf@middlelinewidth@length}{0pt}\relax]%
          {\mdf@middlelinewidth@length}%
          {\dimexpr\mdfboundingboxtotalheight%
              +\ifbool{mdf@bottomline}{\mdf@middlelinewidth@length}{0pt}\relax}%
  }%
}% 
%
\def\mdf@putbox@second{%
  \ifvoid\mdf@splitbox@one%
  \else
   \mdf@makebox@out{%
      \mdf@makeboxalign@left%
      \setlength{\mdfboundingboxwidth}{\wd\mdf@splitbox@one}%
      \setlength{\mdfboundingboxtotalwidth}%
                   {\dimexpr\mdfboundingboxwidth+\mdf@innerleftmargin@length%
                        +\mdf@innerrightmargin@length\relax}%
      \setlength{\mdfboundingboxheight}{\dimexpr\ht\mdf@splitbox@one+\dp\mdf@splitbox@one\relax}%
      \setlength{\mdfboundingboxdepth}%
                   {\dimexpr\dp\mdf@splitbox@one+\mdf@innerbottommargin@length\relax}%
      \setlength{\mdfboundingboxtotalheight}%
                   {\dimexpr\mdfboundingboxheight+\mdf@innerbottommargin@length\relax}%
      \setlength{\@tempdima}{\dimexpr\mdfboundingboxtotalwidth%
                             +\ifbool{mdf@leftline}{\mdf@middlelinewidth@length}{\z@}%
                             +\ifbool{mdf@rightline}{\mdf@middlelinewidth@length}{\z@}%
                            \relax}%
      \mdf@makebox@in[\@tempdima]{%
      \null%
        \ifbool{mdf@leftline}{%
           \hspace*{\mdf@middlelinewidth@length}%
           \mdf@frame@leftline@second}{}%
        \ifbool{mdf@bottomline}{%
            \mdf@frame@bottomline@second}{}%
        \mdf@frame@background@second%
        \mdf@frame@titlebackground@second%
        \hspace*{\mdf@innerleftmargin@length}%
        \ifbool{mdf@rightline}{%
            \mdf@frame@rightline@second}{}%
        {\box\mdf@splitbox@one}%
    }%
    \mdf@makeboxalign@right%
  }%
  \fi%
}%
%    \end{macrocode}
% \end{macro}
%
%
% \begin{macro}{mdf@putbox@middle,mdf@frame@background@middle,mdf@frame@leftline@middle,mdf@frame@rightline@middle}
% The last frame of of a splitted contents of mdframed
%    \begin{macrocode}
\def\mdf@frame@background@middle{%
  \rlap{\mdf@background@default%
     \rule[-\mdfboundingboxdepth]%
          {\mdfboundingboxtotalwidth}%
          {\mdfboundingboxtotalheight}%
  }%
}%
\def\mdf@frame@titlebackground@middle{%
 \ifdefempty{\mdf@frametitle}{}{%
  \ifdimgreater{\mdfframetitleboxtotalheight}{\z@}{%
   \rlap{\mdf@titlebackground@default%
     \rule[\dimexpr-\mdfboundingboxdepth+\mdfboundingboxtotalheight-\mdfframetitleboxtotalheight\relax]%
          {\mdfboundingboxtotalwidth}%
          {\mdfframetitleboxtotalheight}%
   }%
   \global\setlength{\mdfframetitleboxtotalheight}{\z@}%
  }{}%
 }%
}%
% 
\def\mdf@frame@leftline@middle{%
  \llap{\mdf@linecolor@default%
     \rule[-\mdfboundingboxdepth]%
          {\mdf@middlelinewidth@length}%
          {\mdfboundingboxtotalheight}%
  }%     
}%
%
\def\mdf@frame@rightline@middle{%
  \rlap{\mdf@linecolor@default\hspace*{\mdfboundingboxwidth}%
     \hspace*{\mdf@innerrightmargin@length}%
     \rule[-\mdfboundingboxdepth]%
          {\mdf@middlelinewidth@length}%
          {\mdfboundingboxtotalheight}%
  }%
}% 
%
\def\mdf@putbox@middle{%
  \ifvoid\mdf@splitbox@one%
  \else
   \mdf@makebox@out{%
      \mdf@makeboxalign@left%
      \setlength{\mdfboundingboxwidth}{\wd\mdf@splitbox@two}%
      \setlength{\mdfboundingboxtotalwidth}%
                   {\dimexpr\mdfboundingboxwidth+\mdf@innerleftmargin@length%
                           +\mdf@innerrightmargin@length\relax}%
      \setlength{\mdfboundingboxheight}{\dimexpr\ht\mdf@splitbox@two+\dp\mdf@splitbox@two\relax}%
      \setlength{\mdfboundingboxdepth}%
                   {\dimexpr\dp\mdf@splitbox@two+\mdf@splitbottomskip@length\relax}%
      \setlength{\mdfboundingboxtotalheight}%
                   {\dimexpr\mdfboundingboxheight+\mdf@splitbottomskip@length\relax}%
      \setlength{\@tempdima}{\dimexpr\mdfboundingboxtotalwidth%
                             +\ifbool{mdf@leftline}{\mdf@middlelinewidth@length}{\z@}%
                             +\ifbool{mdf@rightline}{\mdf@middlelinewidth@length}{\z@}%
                    \relax}%
      \mdf@makebox@in[\@tempdima]{%
        \null%
        \ifbool{mdf@leftline}{%
           \hspace*{\mdf@middlelinewidth@length}%
           \mdf@frame@leftline@middle}{}%
        \mdf@frame@background@middle%
        \mdf@frame@titlebackground@middle%
        \hspace*{\mdf@innerleftmargin@length}%
        \ifbool{mdf@rightline}{%
            \mdf@frame@rightline@middle}{}%
           {\box\mdf@splitbox@two}%
    }%
    \mdf@makeboxalign@right%
  }
  \fi%
}
%    \end{macrocode}
% \end{macro}
%
%    \begin{macrocode}
 \endinput
%    \end{macrocode}
%
% \iffalse
%</mdframedefault>
% \fi
%
%
%\iffalse
%    \begin{macrocode}
%<*mdframetikz>
%    \end{macrocode}
%\fi
% \subsection{The Explanation of md-frame-1.mdf}
%    \begin{macrocode}
%% Style file for mdframed for package option `framemethod=default`
%% 
%% This package may be distributed under the terms of the LaTeX Project
%% Public License, as described in lppl.txt in the base LaTeX distribution.
%% Either version 1.0 or, at your option, any later version.

%%$Id: mdframed.dtx 287 2011-12-19 20:12:09Z marco $
%
%    \end{macrocode}
% \changes{v1.1beta}{2011/12/16}{added lost semicolons}
% \begin{macro}{mdframedIpackagename,mdf@frameIdate@svn}
% local settings
%    \begin{macrocode}
\def\mdframedIpackagename{md-frame-1}
\def\mdf@frameIdate@svn$#1: #2 #3 #4-#5-#6 #7 #8${#4/#5/#6\space }
\ProvidesFile{md-frame-1.mdf}%
          [\mdf@frameIdate@svn$Id: mdframed.dtx 287 2011-12-19 20:12:09Z marco $ %
           \mdversion: \mdframedIpackagename]
%
%    \end{macrocode}
% \end{macro}
%
% \begin{macro}{mdf@test@ltrb,mdf@test@ltr,mdf@test@ltb,mdf@test@trb,mdf@test@lrb,mdf@test@lb,mdf@test@rb,^^A
%                mdf@test@tr,mdf@test@lt,mdf@test@lr,mdf@test@tb,mdf@test@l,mdf@test@r,mdf@test@t,mdf@test@b,mdf@test@noline}
% Short forms of checking the option which lines should be drawn.
%    \begin{macrocode}
%%%%    _____t_____
%%%%   |           |
%%%%   |           |
%%%%   |           |
%%%%  l|           |r
%%%%   |           |
%%%%   |           |
%%%%   |___________|
%%%%         b
%%Zusammenhaenge abfragen:
\newrobustcmd*\mdf@test@ltrb{%
    \ifboolexpr{ (bool {mdf@topline}) and (bool {mdf@bottomline})
                  and (bool {mdf@leftline}) and (bool {mdf@rightline})}}
%3-set
\newrobustcmd*\mdf@test@ltr{%
    \ifboolexpr{ (bool {mdf@topline}) and not (bool {mdf@bottomline})
                  and (bool {mdf@leftline}) and (bool {mdf@rightline})}}
\newrobustcmd*\mdf@test@ltb{%
    \ifboolexpr{ (bool {mdf@topline}) and (bool {mdf@bottomline})
                  and (bool {mdf@leftline}) and not (bool {mdf@rightline})}}
\newrobustcmd*\mdf@test@trb{%
    \ifboolexpr{ (bool {mdf@topline}) and (bool {mdf@bottomline})
                  and not (bool {mdf@leftline}) and (bool {mdf@rightline})}}
\newrobustcmd*\mdf@test@lrb{%
    \ifboolexpr{ not (bool {mdf@topline}) and (bool {mdf@bottomline})
                  and (bool {mdf@leftline}) and (bool {mdf@rightline})}}
%2-set
\newrobustcmd*\mdf@test@lb{%
    \ifboolexpr{ not (bool {mdf@topline}) and (bool {mdf@bottomline})
                  and (bool {mdf@leftline}) and not (bool {mdf@rightline})}}
\newrobustcmd*\mdf@test@rb{%
    \ifboolexpr{ not (bool {mdf@topline}) and (bool {mdf@bottomline})
                  and not (bool {mdf@leftline}) and (bool {mdf@rightline})}}
\newrobustcmd*\mdf@test@tr{%
    \ifboolexpr{ (bool {mdf@topline}) and not (bool {mdf@bottomline})
                  and not (bool {mdf@leftline}) and (bool {mdf@rightline})}}
\newrobustcmd*\mdf@test@lt{%
    \ifboolexpr{ (bool {mdf@topline}) and not (bool {mdf@bottomline})
                  and (bool {mdf@leftline}) and not (bool {mdf@rightline})}}
\newrobustcmd*\mdf@test@lr{%
    \ifboolexpr{not (bool {mdf@topline}) and not (bool {mdf@bottomline})
                  and (bool {mdf@leftline}) and (bool {mdf@rightline})}}
\newrobustcmd*\mdf@test@tb{%
    \ifboolexpr{ (bool {mdf@topline}) and (bool {mdf@bottomline})
                  and not (bool {mdf@leftline}) and not (bool {mdf@rightline})}}
%Einzellinien
\newrobustcmd*\mdf@test@l{%
    \ifboolexpr{ not (bool {mdf@topline}) and not (bool {mdf@bottomline})
                  and (bool {mdf@leftline}) and not (bool {mdf@rightline})}}
\newrobustcmd*\mdf@test@r{%
    \ifboolexpr{ not (bool {mdf@topline}) and not (bool {mdf@bottomline})
                  and not (bool {mdf@leftline}) and (bool {mdf@rightline})}}
\newrobustcmd*\mdf@test@t{%
    \ifboolexpr{ (bool {mdf@topline}) and not (bool {mdf@bottomline})
                  and not (bool {mdf@leftline}) and not (bool {mdf@rightline})}}
\newrobustcmd*\mdf@test@b{%
    \ifboolexpr{ not (bool {mdf@topline}) and (bool {mdf@bottomline})
                  and not (bool {mdf@leftline}) and not (bool {mdf@rightline})}}
%keine Linien
\newrobustcmd*\mdf@test@noline{%
    \ifboolexpr{ not (bool {mdf@topline}) and not (bool {mdf@bottomline})
                  and not (bool {mdf@leftline}) and not (bool {mdf@rightline})}}
%
%    \end{macrocode}
% \end{macro}
%
% \begin{macro}{mdf@tikz@settings}
% Define settings for tikz
%    \begin{macrocode}
%Allgemeine Einstellungen fuer tikz
\def\mdf@tikz@settings{%
%
  \tikzset{mdfbox/.style={anchor=south west,%
                          inner sep=0pt,%
                          outer sep=0pt,%
                          \mdf@fontcolor,}}% anchor der Ausgabebox ist unten links
  \tikzset{mdfcorners/.style={rounded corners=\mdf@roundcorner@length}}% 
  \tikzset{mdfbackground/.style={fill=\mdf@backgroundcolor,%
                                 draw=\mdf@backgroundcolor}}% 
  \tikzset{mdfframetitlebackground/.style={fill=\mdf@frametitlebackgroundcolor,%
                                 draw=\mdf@frametitlebackgroundcolor,
                                 rounded corners={max(\mdf@roundcorner@length%
                                                 -\mdf@innerlinewidth@length%
                                                 -.5\mdf@middlelinewidth@length,0)}}}%
%
  \tikzset{mdfouterline/.style={}}%
% nur wenn outerlinewidth>0 wird aussere Linie gezeichnet
  \ifdimgreater{\mdf@outerlinewidth@length}{\z@}
    {\tikzset{mdfouterline/.append style={%
      draw=\mdf@outerlinecolor,%
      line width=2\mdf@outerlinewidth@length+\mdf@middlelinewidth@length}}}{}%
%
  \tikzset{mdfinnerline/.style={}}%
% nur wenn innerlinewidth>0 wird innere Linie gezeichnet
  \ifdimgreater{\mdf@innerlinewidth@length}{\z@}
    {\tikzset{mdfinnerline/.append style={%
      draw=\mdf@innerlinecolor,%
      line width=2\mdf@innerlinewidth@length+\mdf@middlelinewidth@length}}}{}%
%
  \mdf@tikzset@local
  \tikzset{mdfmiddleline/.style={}}%
% nur wenn middlelinewidth>0 wird mittlere Linie gezeichnet
  \ifdimgreater{\mdf@middlelinewidth@length}{\z@}
    {\tikzset{mdfmiddleline/.append style={%
      preaction={draw=\mdf@middlelinecolor,%
                 line width=\mdf@middlelinewidth@length},%
      line width=\mdf@middlelinewidth@length,%
      tikzsetting}}%
    }{}%
}%
%    \end{macrocode}
% \end{macro}
%
% \begin{macro}{mdf@tikzbox@tfl,mdf@tikzbox@otl}
% Befehle fuer Ausgabe von Rahmen und Hintergrund
%    \begin{macrocode}
\newrobustcmd*\mdf@tikzbox@tfl[1]{%three or four borders
    \clip(0,0)rectangle(\mdfboundingboxwidth,\mdfboundingboxheight);%
    \begin{scope}[mdfcorners]%
       \clip[preaction=mdfouterline]%
            [postaction=mdfbackground]%
            [postaction=mdfinnerline]#1;%
    \end{scope}%
    \path[mdfmiddleline,mdfcorners]#1;
  }%



\newrobustcmd*\mdf@tikzbox@otl[2]{%one or two borders
    \clip(0,0)rectangle(\mdfboundingboxwidth,\mdfboundingboxheight);%
    \begin{scope}
       \path[mdfouterline,mdfcorners]#1;%
       \clip[postaction=mdfbackground]#2;%
       \path[mdfinnerline,mdfcorners]#1;%
    \end{scope}%
    \path[mdfmiddleline,mdfcorners]#1;}%
%    \end{macrocode}
% \end{macro}
%
%
%
% \begin{macro}{mdf@put@frametitlerule}
% frametitlerule with tikz
%    \begin{macrocode}
\tikzset{mdfframetitlerule/.style={%
   draw=\mdf@frametitlerulecolor,
   fill=\mdf@frametitlerulecolor,
  }%
}
\newrobustcmd*\mdf@put@frametitlerule{%
  \noindent%
  \rlap{\hspace*{-\mdf@innerleftmargin@length}%
  \pgfmathsetlength{\dimen@}{\hsize+\mdf@innerleftmargin@length+\mdf@innerrightmargin@length}%
  \tikz\draw[mdfframetitlerule] (0,0)%
             rectangle (\dimen@,\mdf@frametitlerulewidth@length);
  }%
}%
%    \end{macrocode}
% \end{macro}
%
% \begin{macro}{mdf@putbox@single}
%  Output of the non breakable contents.
%    \begin{macrocode}
% Info zu den verwendeten Punkten:
% O ist die untere linke Ecke der Mitte der middleline
% P ist die obere rechte Ecke der Mitte der middleline
% A ist der Punkt fuer den anchor (d.h. die untere linke Ecke) der Ausgabebox
%
\def\mdf@putbox@single{%
  \ifvoid\mdf@splitbox@one
  \else%
   \mdf@makebox@out{%
    \mdf@makeboxalign@left%
    \mdf@tikz@settings%
%
    \setlength\mdfboundingboxwidth{\wd\mdf@splitbox@one}%
    \advance\mdfboundingboxwidth by \mdf@innerleftmargin@length\relax%
    \advance\mdfboundingboxwidth by \mdf@innerrightmargin@length\relax%
    \ifbool{mdf@leftline}{%
      \advance\mdfboundingboxwidth by \mdf@innerlinewidth@length\relax%
      \advance\mdfboundingboxwidth by \mdf@middlelinewidth@length\relax%
      \advance\mdfboundingboxwidth by \mdf@outerlinewidth@length\relax}{}%
    \ifbool{mdf@rightline}{%
      \advance\mdfboundingboxwidth by \mdf@innerlinewidth@length\relax%
      \advance\mdfboundingboxwidth by \mdf@middlelinewidth@length\relax%
      \advance\mdfboundingboxwidth by \mdf@outerlinewidth@length\relax}{}%
%
    \setlength\mdfboundingboxheight{\dimexpr\ht\mdf@splitbox@one+\dp\mdf@splitbox@one\relax}%
    \advance\mdfboundingboxheight by \mdf@innertopmargin@length\relax%
    \advance\mdfboundingboxheight by \mdf@innerbottommargin@length\relax%
    \ifbool{mdf@topline}{%
      \advance\mdfboundingboxheight by \mdf@innerlinewidth@length\relax%
      \advance\mdfboundingboxheight by \mdf@middlelinewidth@length\relax%
      \advance\mdfboundingboxheight by \mdf@outerlinewidth@length\relax}{}%
    \ifbool{mdf@bottomline}{%
      \advance\mdfboundingboxheight by \mdf@innerlinewidth@length\relax%
      \advance\mdfboundingboxheight by \mdf@middlelinewidth@length\relax%
      \advance\mdfboundingboxheight by \mdf@outerlinewidth@length\relax}{}%
    \mdf@makebox@in[\mdfboundingboxwidth]{%
    \null%
    \begin{tikzpicture}[remember picture]%
     \begin{scope}
      \pgfmathsetlengthmacro\mdf@Ax{+\mdf@innerleftmargin@length}%
      \pgfmathsetlengthmacro\mdf@Ay{+\mdf@innerbottommargin@length}%
      \pgfmathsetlengthmacro\mdf@Ox{+0pt}%
      \pgfmathsetlengthmacro\mdf@Oy{+0pt}%
      \pgfmathsetlengthmacro\mdf@Px{+\mdfboundingboxwidth}%
      \pgfmathsetlengthmacro\mdf@Py{+\mdfboundingboxheight}%
      \ifbool{mdf@leftline}%
        {%
         \pgfmathsetlengthmacro\mdf@Ax%
              {\mdf@Ax+\mdf@outerlinewidth@length+%
               \mdf@middlelinewidth@length+\mdf@innerlinewidth@length}%
         \pgfmathsetlengthmacro\mdf@Ox%
              {\mdf@Ox+\mdf@outerlinewidth@length+0.5\mdf@middlelinewidth@length}%
        }{}%
      \ifbool{mdf@rightline}%
        {%
         \pgfmathsetlengthmacro\mdf@Px%
              {\mdf@Px-\mdf@outerlinewidth@length-0.5\mdf@middlelinewidth@length}%
        }{}%
      \ifbool{mdf@bottomline}%
        {%
         \pgfmathsetlengthmacro\mdf@Ay%
              {\mdf@Ay+\mdf@outerlinewidth@length+\mdf@middlelinewidth@length%
                +\mdf@innerlinewidth@length}%
         \pgfmathsetlengthmacro\mdf@Oy%
              {\mdf@Oy+\mdf@outerlinewidth@length+0.5\mdf@middlelinewidth@length}%
        }{}%
      \ifbool{mdf@topline}%
        {%
         \pgfmathsetlengthmacro\mdf@Py%
              {\mdf@Py-\mdf@outerlinewidth@length-0.5\mdf@middlelinewidth@length}%
        }{}%
%
      \coordinate(O)at(\mdf@Ox,\mdf@Oy);%
      \coordinate(P)at(\mdf@Px,\mdf@Py);%
%
      \mdf@test@ltrb{\mdf@tikzbox@tfl{(O)--(O|-P)--(P)--(P|-O)--cycle}}{}%
%
      \mdf@test@ltb{\mdf@tikzbox@tfl{(P|-O)--(O)--(O|-P)--(P)}}{}%
      \mdf@test@trb{\mdf@tikzbox@tfl{(O|-P)--(P)--(P|-O)--(O)}}{}%
      \mdf@test@ltr{\mdf@tikzbox@tfl{(O)--(O|-P)--(P)--(P|-O)}}{}%
      \mdf@test@lrb{\mdf@tikzbox@tfl{(P-|O)--(O)--(O-|P)--(P)}}{}%
%
      \mdf@test@lb{\mdf@tikzbox@otl{(P|-O)--(O)--(O|-P)}%
                                 {(P)--(P|-O)[mdfcorners]--(O)--(O|-P)}%
                 }{}%
      \mdf@test@rb{\mdf@tikzbox@otl{(P)--(P|-O)--(O)}%
                                 {(O|-P)--(P)[mdfcorners]--(P|-O)--(O)}%
                 }{}% 
      \mdf@test@tr{\mdf@tikzbox@otl{(O-|P)--(P)--(P-|O)}%
                                 {(O)--(O|-P)[mdfcorners]--(P)--(P|-O)}%
                 }{}%
      \mdf@test@lt{\mdf@tikzbox@otl{(O)--(O|-P)--(P)}%
                                 {(P|-O)--(O)[mdfcorners]--(O|-P)--(P)}%
                 }{}%
      \mdf@test@lr{\mdf@tikzbox@otl{(O)--(O|-P)(P)--(P|-O)}%
                                 {(O)rectangle(P)}%
                 }{}%
      \mdf@test@tb{\mdf@tikzbox@otl{(O)--(O-|P)(O|-P)--(P)}%
                                 {(O)rectangle(P)}%
                 }{}%
%
      \mdf@test@l{\mdf@tikzbox@otl{(O)--(O|-P)}%
                                 {(O)rectangle(P)}%
                 }{}%
      \mdf@test@r{\mdf@tikzbox@otl{(O-|P)--(P)}%
                                 {(O)rectangle(P)}%
                 }{}%
      \mdf@test@t{\mdf@tikzbox@otl{(O|-P)--(P)}%
                                 {(O)rectangle(P)}%
                 }{}%
      \mdf@test@b{\mdf@tikzbox@otl{(O)--(O-|P)}%
                                 {(O)rectangle(P)}%
                 }{}%
%
      \mdf@test@noline{\path[mdfbackground,mdfcorners](O)rectangle(P);}{}%
%
      \begin{scope}%background frame title
        \pgfmathsetlengthmacro\mdf@Ox%
            {\mdf@Ox+\mdf@innerlinewidth@length+0.5\mdf@middlelinewidth@length}
        \pgfmathsetlengthmacro\mdf@Px%
            {\mdf@Px-\mdf@innerlinewidth@length-0.5\mdf@middlelinewidth@length}
        \pgfmathsetlengthmacro\mdf@Py%
            {\mdf@Py-\mdf@innerlinewidth@length-0.5\mdf@middlelinewidth@length}
        \pgfmathsetlengthmacro\mdf@Fy
            {\mdf@Py-\mdfframetitleboxtotalheight}
%       \draw[mdfframetitlebackground,mdfcorners]
                  (\mdf@Ox,\mdf@Fy) rectangle (\mdf@Px,\mdf@Py);  
        \path[mdfframetitlebackground]
            (\mdf@Ox,\mdf@Fy) -- (\mdf@Ox,\mdf@Py)%
            --(\mdf@Px,\mdf@Py) --(\mdf@Px,\mdf@Fy);
      \end{scope}
%
      \node[mdfbox]at(\mdf@Ax,\mdf@Ay){\box\mdf@splitbox@one};% Ausgabebox einfuegen
     \end{scope}
     %HIER KOMMT EIN WEITERES MAKRO
     \mdfcreateextratikz
    \end{tikzpicture}%
    }%
   \mdf@makeboxalign@right%
  }%
 \fi
}%
%    \end{macrocode}
% \end{macro}
%
% \begin{macro}{mdf@putbox@first}
%  Output of the first breakable contents.
%    \begin{macrocode}
\def\mdf@putbox@first{%
  \ifvoid\mdf@splitbox@one
  \else%
        \mdf@makebox@out{%
    \mdf@makeboxalign@left%
    \mdf@tikz@settings%
    \setlength\mdfboundingboxwidth{\wd\mdf@splitbox@two}%
    \advance\mdfboundingboxwidth by \mdf@innerleftmargin@length\relax%
    \advance\mdfboundingboxwidth by \mdf@innerrightmargin@length\relax%
    \ifbool{mdf@leftline}{%
      \advance\mdfboundingboxwidth by \mdf@innerlinewidth@length\relax%
      \advance\mdfboundingboxwidth by \mdf@middlelinewidth@length\relax%
      \advance\mdfboundingboxwidth by \mdf@outerlinewidth@length\relax}{}%
    \ifbool{mdf@rightline}{%
      \advance\mdfboundingboxwidth by \mdf@innerlinewidth@length\relax%
      \advance\mdfboundingboxwidth by \mdf@middlelinewidth@length\relax%
      \advance\mdfboundingboxwidth by \mdf@outerlinewidth@length\relax}{}%
%
    \setlength\mdfboundingboxheight{\dimexpr\ht\mdf@splitbox@two+\dp\mdf@splitbox@two\relax}%
    \advance\mdfboundingboxheight by \mdf@innertopmargin@length\relax%
    \advance\mdfboundingboxheight by \mdf@splitbottomskip@length\relax%
    \ifbool{mdf@topline}{%
      \advance\mdfboundingboxheight by \mdf@innerlinewidth@length\relax%
      \advance\mdfboundingboxheight by \mdf@middlelinewidth@length\relax%
      \advance\mdfboundingboxheight by \mdf@outerlinewidth@length\relax}{}%
%
    %\ifdimequal{\pagegoal}{\maxdimen}{\enlargethispage{\baselineskip}}{}% ???
    \ifdimgreater{\pagegoal-\maxdimen}{0pt}{}{\enlargethispage{\baselineskip}}%
    \mdf@makebox@in[\mdfboundingboxwidth]{%
    \null%
    \begin{tikzpicture}[remember picture]
     \begin{scope}
%
      \pgfmathsetlengthmacro\mdf@Ax{+\mdf@innerleftmargin@length}%
      \pgfmathsetlengthmacro\mdf@Ay{+\mdf@splitbottomskip@length}%
      \pgfmathsetlengthmacro\mdf@Ox{+0pt}%
      \pgfmathsetlengthmacro\mdf@Oy{+0pt}%
      \pgfmathsetlengthmacro\mdf@Px{+\mdfboundingboxwidth}%
      \pgfmathsetlengthmacro\mdf@Py{+\mdfboundingboxheight}%
      \ifbool{mdf@leftline}
        {%
         \pgfmathsetlengthmacro\mdf@Ax%
              {\mdf@Ax+\mdf@outerlinewidth@length+%
               \mdf@middlelinewidth@length+\mdf@innerlinewidth@length}%
         \pgfmathsetlengthmacro\mdf@Ox%
              {\mdf@Ox+\mdf@outerlinewidth@length+0.5\mdf@middlelinewidth@length}%
        }{}%
      \ifbool{mdf@rightline}{%
          \pgfmathsetlengthmacro\mdf@Px%
              {\mdf@Px-\mdf@outerlinewidth@length-0.5\mdf@middlelinewidth@length}%
        }{}%
      \ifbool{mdf@topline}{%
          \pgfmathsetlengthmacro\mdf@Py%
              {\mdf@Py-\mdf@outerlinewidth@length-0.5\mdf@middlelinewidth@length}%
        }{}%
%
      \coordinate(O)at(\mdf@Ox,\mdf@Oy);%
      \coordinate(P)at(\mdf@Px,\mdf@Py);%
%
      \ifboolexpr{test {\mdf@test@ltrb} or test {\mdf@test@ltr}}%
        {\mdf@tikzbox@tfl{(O)--(O|-P)--(P)--(P|-O)}}%
        {}%
      \ifboolexpr{test {\mdf@test@ltb} or test {\mdf@test@lt}}%
        {\mdf@tikzbox@otl{(O)--(O|-P)--(P)}{(P|-O)--(O)[mdfcorners]--(O|-P)--(P)}}%
        {}%
      \ifboolexpr{test {\mdf@test@trb} or test {\mdf@test@tr}}%
        {\mdf@tikzbox@otl{(O-|P)--(P)--(P-|O)}{(O)--(O|-P)[mdfcorners]--(P)--(P|-O)}}%
        {}%
      \ifboolexpr{test {\mdf@test@lrb} or test {\mdf@test@lr}}%
        {\mdf@tikzbox@otl{(O)--(O|-P)(P)--(P|-O)}{(O)rectangle(P)}}%
        {}%
      \ifboolexpr{test {\mdf@test@tb} or test {\mdf@test@t}}%
        {\mdf@tikzbox@otl{(O|-P)--(P)}{(O)rectangle(P)}}%
        {}%
      \ifboolexpr{test {\mdf@test@lb} or test {\mdf@test@l}}%
        {\mdf@tikzbox@otl{(O)--(O|-P)}{(O)rectangle(P)}}%
        {}%
      \ifboolexpr{test {\mdf@test@rb} or test {\mdf@test@r}}%
        {\mdf@tikzbox@otl{(O-|P)--(P)}{(O)rectangle(P)}}%
        {}%
      \mdf@test@b{\path[mdfbackground](O)rectangle(P);}{}%
%
      \mdf@test@noline{\path[mdfbackground,mdfcorners](O)--(O|-P)--(P)--(P|-O);}{}%
%
      \node[mdfbox]at(\mdf@Ax,\mdf@Ay){\box\mdf@splitbox@two};% Ausgabebox einfuegen
     \end{scope}
     %HIER KOMMT EIN WEITERES MAKRO
     \mdfcreateextratikz%
    \end{tikzpicture}%
    }%
   \mdf@makeboxalign@right%
  }%
 \fi
}%
%    \end{macrocode}
% \end{macro}
%
% \begin{macro}{mdf@putbox@middle}
%  Output of the middle breakable contents.
%    \begin{macrocode}
\def\mdf@putbox@middle{%
  \ifvoid\mdf@splitbox@one
  \else%
        \mdf@makebox@out{%
    \mdf@makeboxalign@left%
    \mdf@tikz@settings%
%
    \setlength\mdfboundingboxwidth{\wd\mdf@splitbox@two}%
    \advance\mdfboundingboxwidth by \mdf@innerleftmargin@length\relax%
    \advance\mdfboundingboxwidth by \mdf@innerrightmargin@length\relax%
    \ifbool{mdf@leftline}{%
      \advance\mdfboundingboxwidth by \mdf@innerlinewidth@length\relax%
      \advance\mdfboundingboxwidth by \mdf@middlelinewidth@length\relax%
      \advance\mdfboundingboxwidth by \mdf@outerlinewidth@length\relax}{}%
    \ifbool{mdf@rightline}{%
      \advance\mdfboundingboxwidth by \mdf@innerlinewidth@length\relax%
      \advance\mdfboundingboxwidth by \mdf@middlelinewidth@length\relax%
      \advance\mdfboundingboxwidth by \mdf@outerlinewidth@length\relax}{}%
%
    \setlength\mdfboundingboxheight{\dimexpr\ht\mdf@splitbox@two+\dp\mdf@splitbox@two\relax}%
    \advance\mdfboundingboxheight by \mdf@splitbottomskip@length\relax%
%
    \mdf@makebox@in[\mdfboundingboxwidth]{%
    \null%
    \begin{tikzpicture}[remember picture]
     \begin{scope}
      \pgfmathsetlengthmacro\mdf@Ax{+\mdf@innerleftmargin@length}%
      \pgfmathsetlengthmacro\mdf@Ay{+\mdf@splitbottomskip@length}%
      \pgfmathsetlengthmacro\mdf@Ox{+0pt}%
      \pgfmathsetlengthmacro\mdf@Oy{+0pt}%
      \pgfmathsetlengthmacro\mdf@Px{+\mdfboundingboxwidth}%
      \pgfmathsetlengthmacro\mdf@Py{+\mdfboundingboxheight}%
      \ifbool{mdf@leftline}%
        {%
         \pgfmathsetlengthmacro\mdf@Ax%
              {\mdf@Ax+\mdf@outerlinewidth@length+%
               \mdf@middlelinewidth@length+\mdf@innerlinewidth@length}%
         \pgfmathsetlengthmacro\mdf@Ox%
              {\mdf@Ox+\mdf@outerlinewidth@length+0.5\mdf@middlelinewidth@length}%
         }{}%
      \ifbool{mdf@rightline}%
         {%
          \pgfmathsetlengthmacro\mdf@Px%
              {\mdf@Px-\mdf@outerlinewidth@length-0.5\mdf@middlelinewidth@length}%
         }{}%
%
      \coordinate(O)at(\mdf@Ox,\mdf@Oy);%
      \coordinate(P)at(\mdf@Px,\mdf@Py);%
%
      \ifboolexpr{bool {mdf@leftline} and bool {mdf@rightline}}%
               {\mdf@tikzbox@otl{(O)--(O|-P)(P)--(P|-O)}{(O)rectangle(P)}}{}%
      \ifboolexpr{bool {mdf@leftline} and not (bool {mdf@rightline})}%
               {\mdf@tikzbox@otl{(O)--(O|-P)}{(O)rectangle(P)}}{}%
      \ifboolexpr{not (bool {mdf@leftline}) and bool {mdf@rightline}}%
               {\mdf@tikzbox@otl{(P)--(P|-O)}{(O)rectangle(P)}}{}%
      \ifboolexpr{not (bool {mdf@leftline}) and not (bool {mdf@rightline})}%
               {\path[mdfbackground](O)rectangle(P);}{}%
%
      \node[mdfbox]at(\mdf@Ax,\mdf@Ay){\box\mdf@splitbox@two};% Ausgabebox einfuegen
     \end{scope}
     %HIER KOMMT EIN WEITERES MAKRO
    \end{tikzpicture}%
    }%
   \mdf@makeboxalign@right%
  }%
 \fi
}%
%    \end{macrocode}
% \end{macro}
%
% \begin{macro}{mdf@putbox@second}
%  Output of the last breakable contents.
%    \begin{macrocode}
\def\mdf@putbox@second{%
  \ifvoid\mdf@splitbox@one
  \else%
        \mdf@makebox@out{%
    \mdf@makeboxalign@left%
    \mdf@tikz@settings%
%
    \setlength\mdfboundingboxwidth{\wd\mdf@splitbox@one}%
    \advance\mdfboundingboxwidth by \mdf@innerleftmargin@length\relax%
    \advance\mdfboundingboxwidth by \mdf@innerrightmargin@length\relax%
    \ifbool{mdf@leftline}{%
      \advance\mdfboundingboxwidth by \mdf@innerlinewidth@length\relax%
      \advance\mdfboundingboxwidth by \mdf@middlelinewidth@length\relax%
      \advance\mdfboundingboxwidth by \mdf@outerlinewidth@length\relax}{}%
    \ifbool{mdf@rightline}{%
      \advance\mdfboundingboxwidth by \mdf@innerlinewidth@length\relax%
      \advance\mdfboundingboxwidth by \mdf@middlelinewidth@length\relax%
      \advance\mdfboundingboxwidth by \mdf@outerlinewidth@length\relax}{}%
%
    \setlength\mdfboundingboxheight{\dimexpr\ht\mdf@splitbox@one+\dp\mdf@splitbox@one\relax}%
    \advance\mdfboundingboxheight by \mdf@innerbottommargin@length\relax%
    \ifbool{mdf@bottomline}{%
      \advance\mdfboundingboxheight by \mdf@innerlinewidth@length\relax%
      \advance\mdfboundingboxheight by \mdf@middlelinewidth@length\relax%
      \advance\mdfboundingboxheight by \mdf@outerlinewidth@length\relax}{}%
%
    \mdf@makebox@in[\mdfboundingboxwidth]{%
    \null%
    \begin{tikzpicture}[remember picture]
     \begin{scope}
      \pgfmathsetlengthmacro\mdf@Ax{+\mdf@innerleftmargin@length}%
      \pgfmathsetlengthmacro\mdf@Ay{+\mdf@innerbottommargin@length}%
      \pgfmathsetlengthmacro\mdf@Ox{+0pt}%
      \pgfmathsetlengthmacro\mdf@Oy{+0pt}%
      \pgfmathsetlengthmacro\mdf@Px{+\mdfboundingboxwidth}%
      \pgfmathsetlengthmacro\mdf@Py{+\mdfboundingboxheight}%
      \ifbool{mdf@leftline}%
        {%
         \pgfmathsetlengthmacro\mdf@Ax%
              {\mdf@Ax+\mdf@outerlinewidth@length+% 
               \mdf@middlelinewidth@length+\mdf@innerlinewidth@length}%
          \pgfmathsetlengthmacro\mdf@Ox%
              {\mdf@Ox+\mdf@outerlinewidth@length+0.5\mdf@middlelinewidth@length}%
         }{}%
      \ifbool{mdf@rightline}%
         {%
          \pgfmathsetlengthmacro\mdf@Px%
              {\mdf@Px-\mdf@outerlinewidth@length-0.5\mdf@middlelinewidth@length}%
         }{}%
      \ifbool{mdf@bottomline}%
         {%
          \pgfmathsetlengthmacro\mdf@Ay%
              {\mdf@Ay+\mdf@outerlinewidth@length+%
               \mdf@middlelinewidth@length+\mdf@innerlinewidth@length}%
          \pgfmathsetlengthmacro\mdf@Oy%
              {\mdf@Oy+\mdf@outerlinewidth@length+0.5\mdf@middlelinewidth@length}%
         }{}%
%
      \coordinate(O)at(\mdf@Ox,\mdf@Oy);%
      \coordinate(P)at(\mdf@Px,\mdf@Py);%
%
      \ifboolexpr{test {\mdf@test@ltrb} or test {\mdf@test@lrb}}%
        {\mdf@tikzbox@tfl{(P-|O)--(O)--(O-|P)--(P)}}%
        {}%
      \ifboolexpr{test {\mdf@test@ltb} or test {\mdf@test@lb}}%
        {\mdf@tikzbox@otl{(P-|O)--(O)--(O-|P)}{(P)--(P|-O)[mdfcorners]--(O)--(O|-P)}}%
        {}%
      \ifboolexpr{test {\mdf@test@trb} or test {\mdf@test@rb}}%
        {\mdf@tikzbox@otl{(P)--(P|-O)--(O)}{(O|-P)--(P)[mdfcorners]--(P|-O)--(O)}}%
        {}%
      \ifboolexpr{test {\mdf@test@ltr} or test {\mdf@test@lr}}%
        {\mdf@tikzbox@otl{(O)--(O|-P)(P)--(P|-O)}{(O)rectangle(P)}}%
        {}%
      \ifboolexpr{test {\mdf@test@tb} or test {\mdf@test@b}}%
        {\mdf@tikzbox@otl{(O)--(O-|P)}{(O)rectangle(P)}}%
        {}%
      \ifboolexpr{test {\mdf@test@lt} or test {\mdf@test@l}}%
        {\mdf@tikzbox@otl{(O)--(O|-P)}{(O)rectangle(P)}}%
        {}%
      \ifboolexpr{test {\mdf@test@tr} or test {\mdf@test@r}}%
        {\mdf@tikzbox@otl{(O-|P)--(P)}{(O)rectangle(P)}}%
        {}%
      \mdf@test@t{\path[mdfbackground](O)rectangle(P)}{}%
%
      \mdf@test@noline{\path[mdfbackground,mdfcorners](O|-P)--(O)--(O-|P)--(P);}{}%
%
      \node[mdfbox] at (\mdf@Ax,\mdf@Ay){\box\mdf@splitbox@one};% Ausgabebox einfuegen
     \end{scope}
     %HIER KOMMT EIN WEITERES MAKRO
    \end{tikzpicture}%
    }%
   \mdf@makeboxalign@right%
  }%
 \fi
}%
%    \end{macrocode}
% \end{macro}
%
%    \begin{macrocode}
 \endinput
%    \end{macrocode}
% \iffalse
%</mdframetikz>
% \fi
%
%
%\iffalse
%    \begin{macrocode}
%<*mdframepstricks>
%    \end{macrocode}
%\fi
% \subsection{The Explanation of md-frame-2.mdf / md-frame-3.mdf}
%    \begin{macrocode}
%% Style file for mdframed for package option `framemethod=default`
%% 
%% This package may be distributed under the terms of the LaTeX Project
%% Public License, as described in lppl.txt in the base LaTeX distribution.
%% Either version 1.0 or, at your option, any later version.

%%$Id: mdframed.dtx 287 2011-12-19 20:12:09Z marco $
%
%    \end{macrocode}
%
% \begin{macro}{mdframedIIpackagename,mdf@frameIIdate@svn}
% local settings
%    \begin{macrocode}
\def\mdframedIIpackagename{md-frame-2}
\def\mdf@frameIIdate@svn$#1: #2 #3 #4-#5-#6 #7 #8${#4/#5/#6\space }
\ProvidesFile{md-frame-2.mdf}%
         [\mdf@frameIIdate@svn$Id: mdframed.dtx 287 2011-12-19 20:12:09Z marco $ %
          \mdversion: \mdframedIIpackagename]
%    \end{macrocode}
% \end{macro}
%
% \begin{macro}{mdf@ptlength@to@pscode,ptTps}
% Command to calculate a latex length to postscript
%    \begin{macrocode}
\def\mdf@ptlength@to@pscode#1{\pst@number{#1} \pst@number\psxunit div }
\let\ptTps\mdf@ptlength@to@pscode\relax
%    \end{macrocode}
% \end{macro}
%
% \begin{macro}{mdf@test@ltrb,mdf@test@ltr,mdf@test@ltb,mdf@test@trb,mdf@test@lrb,mdf@test@lb,mdf@test@rb,^^A
%                mdf@test@tr,mdf@test@lt,mdf@test@lr,mdf@test@tb,mdf@test@l,mdf@test@r,mdf@test@t,mdf@test@b,mdf@test@single}
% Short forms of checking the option which lines should be drawn.
%    \begin{macrocode}
%%%%    _____t_____
%%%%   |           |
%%%%   |           |
%%%%   |           |
%%%%  l|           |r
%%%%   |           |
%%%%   |           |
%%%%   |___________|
%%%%         b
%%Zusammenhaenge abfragen:
\newrobustcmd*\mdf@test@ltrb{%
    \ifboolexpr{ (bool {mdf@topline}) and (bool {mdf@bottomline})
                  and (bool {mdf@leftline}) and (bool {mdf@rightline})}}
%3-set
\newrobustcmd*\mdf@test@ltr{%
    \ifboolexpr{ (bool {mdf@topline}) and not (bool {mdf@bottomline})
                  and (bool {mdf@leftline}) and (bool {mdf@rightline})}}
\newrobustcmd*\mdf@test@ltb{%
    \ifboolexpr{ (bool {mdf@topline}) and (bool {mdf@bottomline})
                  and (bool {mdf@leftline}) and not (bool {mdf@rightline})}}
\newrobustcmd*\mdf@test@trb{%
    \ifboolexpr{ (bool {mdf@topline}) and (bool {mdf@bottomline})
                  and not (bool {mdf@leftline}) and (bool {mdf@rightline})}}
\newrobustcmd*\mdf@test@lrb{%
    \ifboolexpr{ not (bool {mdf@topline}) and (bool {mdf@bottomline})
                  and (bool {mdf@leftline}) and (bool {mdf@rightline})}}
%2-set
\newrobustcmd*\mdf@test@lb{%
    \ifboolexpr{ not (bool {mdf@topline}) and (bool {mdf@bottomline})
                  and (bool {mdf@leftline}) and not (bool {mdf@rightline})}}
\newrobustcmd*\mdf@test@rb{%
    \ifboolexpr{ not (bool {mdf@topline}) and (bool {mdf@bottomline})
                  and not (bool {mdf@leftline}) and (bool {mdf@rightline})}}
\newrobustcmd*\mdf@test@tr{%
    \ifboolexpr{ (bool {mdf@topline}) and not (bool {mdf@bottomline})
                  and not (bool {mdf@leftline}) and (bool {mdf@rightline})}}
\newrobustcmd*\mdf@test@lt{%
    \ifboolexpr{ (bool {mdf@topline}) and not (bool {mdf@bottomline})
                  and (bool {mdf@leftline}) and not (bool {mdf@rightline})}}
%Einzellinien
\newrobustcmd*\mdf@test@single{%
    \ifboolexpr{ not (test {\mdf@test@ltrb} or test {\mdf@test@ltr} or 
                  test {\mdf@test@ltb} or test {\mdf@test@trb} or 
                  test {\mdf@test@lrb}  or test {\mdf@test@lb} or 
                  test {\mdf@test@rb} or test {\mdf@test@tr} or
                  test {\mdf@test@lt} ) }}

%    \end{macrocode}
% \end{macro}
%
% \begin{macro}{mdfbackgroundstyle,mdflinestyle}
% background and line settings for pstricks
%    \begin{macrocode}
\newpsstyle{mdfbackgroundstyle}{linecolor=\mdf@backgroundcolor,fillstyle=solid,%
                                fillcolor=\mdf@backgroundcolor,linestyle=solid,%
                                linewidth=\mdf@middlelinewidth@length,dimen=outer,%
}%
%
\newpsstyle{mdflinestyle}{linewidth=\mdf@middlelinewidth@length,%
                          linecolor=\mdf@linecolor,%
                          fillstyle=none,dimen=middle,}%
%    \end{macrocode}
% \end{macro}
%
% \begin{macro}{mdf@putbox@single}
% Single output
%    \begin{macrocode}
\def\mdf@putbox@single{%
  \ifvoid\mdf@splitbox@one
  \else%
   \mdf@makebox@out{%
     \mdf@makeboxalign@left%
     \ifbool{mdf@leftline}{\hspace*{\mdf@middlelinewidth@length}}{}%
     \setlength\mdfboundingboxwidth{\wd\mdf@splitbox@one}%
     \setlength\mdfboundingboxheight{\dimexpr\ht\mdf@splitbox@one+\dp\mdf@splitbox@one\relax}%
     \advance\mdfboundingboxheight by \mdf@innertopmargin@length\relax%
     \advance\mdfboundingboxheight by \mdf@innerbottommargin@length\relax%
     \ifbool{mdf@topline}%
        {\advance\mdfboundingboxheight by \mdf@middlelinewidth@length\relax}{}%
     \ifbool{mdf@bottomline}%
        {\advance\mdfboundingboxheight by \mdf@middlelinewidth@length\relax}{}%
     \advance\mdfboundingboxwidth by \mdf@innerleftmargin@length\relax%
     \advance\mdfboundingboxwidth by \mdf@innerrightmargin@length\relax%
     \psset{linearc=\mdf@roundcorner@length,cornersize=absolut,}%
     \expandafter\psset\expandafter{\mdf@psset@local}%
     \psset{unit=1truecm}%
     \mdf@makebox@in[\mdfboundingboxwidth]{%
       \null%
       \begin{pspicture}(0,0)(\mdfboundingboxwidth,\mdfboundingboxheight)
          \mdf@test@ltrb{%
            \psframe[style=mdflinestyle]%
              (! \ptTps{\mdf@middlelinewidth@length} neg 0.5 mul %
                 \ptTps{\mdf@middlelinewidth@length} 0.5 mul)%
              (! \ptTps{\mdfboundingboxwidth} %
                 \ptTps{\mdf@middlelinewidth@length} 0.5 mul add %X
                 \ptTps{\mdfboundingboxheight}%
                 \ptTps{\mdf@middlelinewidth@length} neg 0.5 mul add)
          }{}
          \mdf@test@ltr{%
            \psline[style=mdflinestyle]%
              (! \ptTps{\mdf@middlelinewidth@length} neg 0.5 mul 0)%
              (! \ptTps{\mdf@middlelinewidth@length} neg 0.5 mul 
                 \ptTps{\mdfboundingboxheight} 
                 \ptTps{\mdf@middlelinewidth@length} neg 0.5 mul add)%
              (! \ptTps{\mdfboundingboxwidth} 
                 \ptTps{\mdf@middlelinewidth@length} 0.5 mul add %X
                 \ptTps{\mdfboundingboxheight} 
                 \ptTps{\mdf@middlelinewidth@length} neg 0.5 mul add)%
              (! \ptTps{\mdfboundingboxwidth} 
                 \ptTps{\mdf@middlelinewidth@length} 0.5 mul add %X
                 0)
          }{}
          \mdf@test@ltb{%
            \psline[style=mdflinestyle]%
              (! \ptTps{\mdfboundingboxwidth} 
                 \ptTps{\mdf@middlelinewidth@length} 0.5 mul)% 
              (! \ptTps{\mdf@middlelinewidth@length} neg 0.5 mul 
                 \ptTps{\mdf@middlelinewidth@length} 0.5 mul)%
              (! \ptTps{\mdf@middlelinewidth@length} neg 0.5 mul 
                 \ptTps{\mdfboundingboxheight} 
                 \ptTps{\mdf@middlelinewidth@length} neg 0.5 mul add)%
              (! \ptTps{\mdfboundingboxwidth}  %X
                 \ptTps{\mdfboundingboxheight} 
                 \ptTps{\mdf@middlelinewidth@length} neg 0.5 mul add)%
          }{}
          \mdf@test@trb{%
            \psline[style=mdflinestyle]%
              (! 0
                 \ptTps{\mdfboundingboxheight} 
                 \ptTps{\mdf@middlelinewidth@length} neg 0.5 mul add)%
              (! \ptTps{\mdfboundingboxwidth} 
                 \ptTps{\mdf@middlelinewidth@length} 0.5 mul add %X
                 \ptTps{\mdfboundingboxheight} 
                 \ptTps{\mdf@middlelinewidth@length} neg 0.5 mul add)%
              (! \ptTps{\mdfboundingboxwidth} 
                 \ptTps{\mdf@middlelinewidth@length} 0.5 mul add %X
                 \ptTps{\mdf@middlelinewidth@length} 0.5 mul)
              (! 0  \ptTps{\mdf@middlelinewidth@length} 0.5 mul)
          }{}
          \mdf@test@lrb{%
            \psline[style=mdflinestyle]%
              (! \ptTps{\mdf@middlelinewidth@length} 0.5 mul neg
                 \ptTps{\mdfboundingboxheight})%
              (! \ptTps{\mdf@middlelinewidth@length} 0.5 mul neg 
                 \ptTps{\mdf@middlelinewidth@length} 0.5 mul )%
              (! \ptTps{\mdfboundingboxwidth} 
                 \ptTps{\mdf@middlelinewidth@length} 0.5 mul add %X
                 \ptTps{\mdf@middlelinewidth@length} 0.5 mul )
              (! \ptTps{\mdfboundingboxwidth} 
                 \ptTps{\mdf@middlelinewidth@length} 0.5 mul add %X
                 \ptTps{\mdfboundingboxheight})
          }{}
          \mdf@test@lb{%
            \psline[style=mdflinestyle]%
              (! \ptTps{\mdf@middlelinewidth@length} 0.5 mul neg
                 \ptTps{\mdfboundingboxheight})%
              (! \ptTps{\mdf@middlelinewidth@length} 0.5 mul neg
                 \ptTps{\mdf@middlelinewidth@length} 0.5 mul )%
              (! \ptTps{\mdfboundingboxwidth} 
                 \ptTps{\mdf@middlelinewidth@length} 0.5 mul )
          }{}
          \mdf@test@rb{%
            \psline[style=mdflinestyle]%
              (! 0 \ptTps{\mdf@middlelinewidth@length} 0.5 mul )%
              (! \ptTps{\mdfboundingboxwidth} 
                 \ptTps{\mdf@middlelinewidth@length} 0.5 mul add %X
                 \ptTps{\mdf@middlelinewidth@length} 0.5 mul )
              (! \ptTps{\mdfboundingboxwidth} 
                 \ptTps{\mdf@middlelinewidth@length} 0.5 mul add %X
                 \ptTps{\mdfboundingboxheight})
          }{}
          \mdf@test@tr{%
            \psline[style=mdflinestyle]%
              (! 0
                 \ptTps{\mdfboundingboxheight} 
                 \ptTps{\mdf@middlelinewidth@length} neg 0.5 mul add)%
              (! \ptTps{\mdfboundingboxwidth} 
                 \ptTps{\mdf@middlelinewidth@length} 0.5 mul add %X
                 \ptTps{\mdfboundingboxheight} 
                 \ptTps{\mdf@middlelinewidth@length} neg 0.5 mul add)%
              (! \ptTps{\mdfboundingboxwidth} 
                 \ptTps{\mdf@middlelinewidth@length} 0.5 mul add %X
                 0)
            }{}
          \mdf@test@lt{%
            \psline[style=mdflinestyle]%
              (! \ptTps{\mdf@middlelinewidth@length} neg 0.5 mul 0)%
              (! \ptTps{\mdf@middlelinewidth@length} neg 0.5 mul 
                 \ptTps{\mdfboundingboxheight} 
                 \ptTps{\mdf@middlelinewidth@length} neg 0.5 mul add)%
              (! \ptTps{\mdfboundingboxwidth}  %X
                 \ptTps{\mdfboundingboxheight} 
                 \ptTps{\mdf@middlelinewidth@length} neg 0.5 mul add)%
          }{}
          \mdf@test@single{%
            \ifbool{mdf@bottomline}{%
             \psline[style=mdflinestyle]%
              (! 0 \ptTps{\mdf@middlelinewidth@length} 0.5 mul )%
              (! \ptTps{\mdfboundingboxwidth} 
                 \ptTps{\mdf@middlelinewidth@length} 0.5 mul )
            }{}%
            \ifbool{mdf@topline}{%
             \psline[style=mdflinestyle]%
              (! 0 \ptTps{\mdfboundingboxheight} 
                 \ptTps{\mdf@middlelinewidth@length} 0.5 mul neg add)%
              (! \ptTps{\mdfboundingboxwidth} 
                 \ptTps{\mdfboundingboxheight} 
                 \ptTps{\mdf@middlelinewidth@length} 0.5 mul neg add)
                }{}%
            \ifbool{mdf@leftline}{%
             \psline[style=mdflinestyle]%
              (! \ptTps{\mdf@middlelinewidth@length} neg 0.5 mul 0)%
              (! \ptTps{\mdf@middlelinewidth@length} neg 0.5 mul
                 \ptTps{\mdfboundingboxheight} )%
            }{}%
            \ifbool{mdf@rightline}{%
             \psline[style=mdflinestyle]%
              (! \ptTps{\mdfboundingboxwidth} 
                 \ptTps{\mdf@middlelinewidth@length} 0.5 mul add 0)%
              (! \ptTps{\mdfboundingboxwidth} 
                 \ptTps{\mdf@middlelinewidth@length} 0.5 mul add 
                 \ptTps{\mdfboundingboxheight} )%
            }{}%
          }{}%
          %Hintergrund
          \psframe[style=mdfbackgroundstyle]%
              (!0 \ifbool{mdf@bottomline}{\ptTps{\mdf@middlelinewidth@length}}{0})
              (! \ptTps{\mdfboundingboxwidth} %
                 \ptTps{\mdfboundingboxheight} 
                 \ifbool{mdf@topline}{\ptTps{\mdf@middlelinewidth@length} sub}{})
          %Ausgabe
          \rput[b]%
              (!\ptTps{\mdfboundingboxwidth} 0.5 mul % X-Koordinate
                 \ptTps{\mdf@innerbottommargin@length} 
                 \ifbool{mdf@bottomline}{\ptTps{\mdf@middlelinewidth@length} add}{}
              ){\box\mdf@splitbox@one}
        \end{pspicture}%
    }%
   \mdf@makeboxalign@right%
  }%
 \fi
}%
%    \end{macrocode}
% \end{macro}
%
% \begin{macro}{mdf@putbox@first}
% First output
%    \begin{macrocode}
\def\mdf@putbox@first{%
  \ifvoid\mdf@splitbox@one
  \else%
   \mdf@makebox@out{%
     \mdf@makeboxalign@left%
     \ifbool{mdf@leftline}{\hspace*{\mdf@middlelinewidth@length}}{}%
     \setlength\mdfboundingboxwidth{\wd\mdf@splitbox@two}%
     \setlength\mdfboundingboxheight{\dimexpr\ht\mdf@splitbox@two+\dp\mdf@splitbox@two\relax}%
     \advance\mdfboundingboxheight by \mdf@innertopmargin@length\relax
     \advance\mdfboundingboxheight by \mdf@splitbottomskip@length\relax%
     \ifbool{mdf@topline}%
        {\advance\mdfboundingboxheight by \mdf@middlelinewidth@length\relax}{}%
     \advance\mdfboundingboxwidth by \mdf@innerleftmargin@length\relax%
     \advance\mdfboundingboxwidth by \mdf@innerrightmargin@length\relax%
     \psset{linearc=\mdf@roundcorner@length,cornersize=absolute}%
     \expandafter\psset\expandafter{\mdf@psset@local}%
     \mdf@makebox@in[\mdfboundingboxwidth]{%
       \null%
       \psset{unit=1truecm}%
       \ifdimgreater{\mdfboundingboxheight}{\vsize}
         {\begin{pspicture}(0,0)(\mdfboundingboxwidth,\vsize)}
         {\begin{pspicture}(0,0)(\mdfboundingboxwidth,\mdfboundingboxheight)}
          %Hintergrund
          \psline[style=mdfbackgroundstyle,%
                  linearc={\dimexpr\mdf@roundcorner@length%
                                  -\mdf@middlelinewidth@length\relax},]
              (! \ptTps{\mdf@middlelinewidth@length} 0.5 mul 0)%
              (! \ptTps{\mdf@middlelinewidth@length}  0.5 mul 
                 \ptTps{\mdfboundingboxheight} 
                 \ifbool{mdf@topline}%
                        {\ptTps{\mdf@middlelinewidth@length} neg 1.5 mul add}%
                        {\ptTps{\mdf@middlelinewidth@length} neg 0.5 mul add} )%
              (! \ptTps{\mdfboundingboxwidth} 
                 \ptTps{\mdf@middlelinewidth@length} 0.5 mul neg add %X
                 \ptTps{\mdfboundingboxheight}
                 \ifbool{mdf@topline}% 
                        {\ptTps{\mdf@middlelinewidth@length} neg 1.5 mul add}%
                        {\ptTps{\mdf@middlelinewidth@length} neg 0.5 mul add} )%
              (! \ptTps{\mdfboundingboxwidth} 
                 \ptTps{\mdf@middlelinewidth@length} 0.5 mul neg add %X
                 0) 
          \ifboolexpr{ test {\mdf@test@ltrb} or test {\mdf@test@ltr}}{%
            \psline[style=mdflinestyle]%
              (! \ptTps{\mdf@middlelinewidth@length} neg 0.5 mul 0)%
              (! \ptTps{\mdf@middlelinewidth@length} neg 0.5 mul 
                 \ptTps{\mdfboundingboxheight} 
                 \ptTps{\mdf@middlelinewidth@length} neg 0.5 mul add)%
              (! \ptTps{\mdfboundingboxwidth} 
                 \ptTps{\mdf@middlelinewidth@length} 0.5 mul add %X
                 \ptTps{\mdfboundingboxheight} 
                 \ptTps{\mdf@middlelinewidth@length} neg 0.5 mul add)%
              (! \ptTps{\mdfboundingboxwidth} 
                 \ptTps{\mdf@middlelinewidth@length} 0.5 mul add %X
                 0)
          }{}
          \ifboolexpr{ test {\mdf@test@tr} or test {\mdf@test@trb}}{%
            \psline[style=mdflinestyle]%
              (! 0
                 \ptTps{\mdfboundingboxheight} 
                 \ptTps{\mdf@middlelinewidth@length} neg 0.5 mul add)%
              (! \ptTps{\mdfboundingboxwidth} 
                 \ptTps{\mdf@middlelinewidth@length} 0.5 mul add %X
                 \ptTps{\mdfboundingboxheight} 
                 \ptTps{\mdf@middlelinewidth@length} neg 0.5 mul add)%
              (! \ptTps{\mdfboundingboxwidth} 
                 \ptTps{\mdf@middlelinewidth@length} 0.5 mul add %X
                 0)
          }{}
          \ifboolexpr{ test {\mdf@test@ltb} or test {\mdf@test@lt}}{%
            \psline[style=mdflinestyle]%
              (! \ptTps{\mdf@middlelinewidth@length} neg 0.5 mul 0)%
              (! \ptTps{\mdf@middlelinewidth@length} neg 0.5 mul 
                 \ptTps{\mdfboundingboxheight} 
                 \ptTps{\mdf@middlelinewidth@length} neg 0.5 mul add)%
              (! \ptTps{\mdfboundingboxwidth} 
                 \ptTps{\mdf@roundcorner@length} neg add  %X
                 \ptTps{\mdfboundingboxheight} 
                 \ptTps{\mdf@middlelinewidth@length} neg 0.5 mul add)%
               }{}
          \ifboolexpr{ test {\mdf@test@single} or test {\mdf@test@rb} or
                       test {\mdf@test@lrb} or test {\mdf@test@lb}}{%
           \ifbool{mdf@topline}{%
            \psline[style=mdflinestyle]%
              (! 0 \ptTps{\mdfboundingboxheight} 
                 \ptTps{\mdf@middlelinewidth@length} 0.5 mul neg add)%
              (! \ptTps{\mdfboundingboxwidth} 
                 \ptTps{\mdfboundingboxheight}  
                 \ptTps{\mdf@middlelinewidth@length} 0.5 mul neg add)
           }{}%
           \ifbool{mdf@leftline}{%
            \psline[style=mdflinestyle]%
              (! \ptTps{\mdf@middlelinewidth@length} neg 0.5 mul 0)%
              (! \ptTps{\mdf@middlelinewidth@length} neg 0.5 mul 
                 \ptTps{\mdfboundingboxheight} )%
           }{}%
           \ifbool{mdf@rightline}{%
             \psline[style=mdflinestyle]%
              (! \ptTps{\mdfboundingboxwidth} 
                 \ptTps{\mdf@middlelinewidth@length} 0.5 mul add 0)%
              (! \ptTps{\mdfboundingboxwidth} 
                 \ptTps{\mdf@middlelinewidth@length} 0.5 mul add 
                 \ptTps{\mdfboundingboxheight} )%
           }{}%
          }{}%
          %Ausgabe
          \rput[b](!
                 \ptTps{\mdfboundingboxwidth} 0.5 mul % X-Koordinate
                 \ptTps{\mdf@splitbottomskip@length}){\box\mdf@splitbox@two}
         \end{pspicture}%
    }%
   \mdf@makeboxalign@right%
  }%
 \fi
}%
%    \end{macrocode}
% \end{macro}
%
% \begin{macro}{mdf@putbox@middle}
% Middle output
%    \begin{macrocode}
\def\mdf@putbox@middle{%
  \ifvoid\mdf@splitbox@one
  \else%
   \mdf@makebox@out{%
    \mdf@makeboxalign@left%
     \ifbool{mdf@leftline}{\hspace*{\mdf@middlelinewidth@length}}{}%
     \setlength\mdfboundingboxwidth{\wd\mdf@splitbox@two}%
     \setlength\mdfboundingboxheight{\dimexpr\ht\mdf@splitbox@two+\dp\mdf@splitbox@two\relax}%
     \advance\mdfboundingboxheight by \mdf@splitbottomskip@length\relax%
     \advance\mdfboundingboxwidth by \mdf@innerleftmargin@length\relax%
     \advance\mdfboundingboxwidth by \mdf@innerrightmargin@length\relax%
     \expandafter\psset\expandafter{\mdf@psset@local}%
     \psset{linearc=\mdf@roundcorner@length,cornersize=absolute}%
     \mdf@makebox@in[\mdfboundingboxwidth]{%
       \null%
       \psset{unit=1truecm}%
       \ifdimgreater{\mdfboundingboxheight}{\vsize}
         {\begin{pspicture}(0,0)(\mdfboundingboxwidth,\vsize)}
         {\begin{pspicture}(0,0)(\mdfboundingboxwidth,\mdfboundingboxheight)}
          \ifbool{mdf@leftline}{%
            \psline[style=mdflinestyle]%
              (! \ptTps{\mdf@middlelinewidth@length} neg 0.5 mul 0)%
              (! \ptTps{\mdf@middlelinewidth@length} neg 0.5 mul 
                 \ptTps{\mdfboundingboxheight} )%
          }{}%
          \ifbool{mdf@rightline}{%
             \psline[style=mdflinestyle]%
              (! \ptTps{\mdfboundingboxwidth} 
                 \ptTps{\mdf@middlelinewidth@length} 0.5 mul add 0)%
              (! \ptTps{\mdfboundingboxwidth} 
                 \ptTps{\mdf@middlelinewidth@length} 0.5 mul add 
                 \ptTps{\mdfboundingboxheight} )%
          }{}%
          %Hintergrund
          \psframe[style=mdfbackgroundstyle,%
                    linearc=\z@]%
              (!0 0)
              (! \ptTps{\mdfboundingboxwidth} 
                 \ptTps{\mdfboundingboxheight})
           %Ausgabe
          \rput[b](!
                 \ptTps{\mdfboundingboxwidth} 0.5 mul % X-Koordinate
                 \ptTps{\mdf@splitbottomskip@length} ){\box\mdf@splitbox@two}
       \end{pspicture}%
    }%
   \mdf@makeboxalign@right%
  }%
 \fi
}%
%    \end{macrocode}
% \end{macro}
%
% \begin{macro}{mdf@putbox@second}
% Last output
%    \begin{macrocode}
\def\mdf@putbox@second{
  \ifvoid\mdf@splitbox@one
  \else%
   \mdf@makebox@out{%
     \mdf@makeboxalign@left%
     \ifbool{mdf@leftline}{\hspace*{\mdf@middlelinewidth@length}}{}%
     \setlength\mdfboundingboxwidth{\wd\mdf@splitbox@one}%
     \setlength\mdfboundingboxheight{\dimexpr\ht\mdf@splitbox@one+\dp\mdf@splitbox@one\relax}%
     \advance\mdfboundingboxheight by \mdf@innerbottommargin@length\relax%
     \ifbool{mdf@bottomline}%
            {\advance\mdfboundingboxheight by \mdf@middlelinewidth@length\relax}{}%
     \advance\mdfboundingboxwidth by \mdf@innerleftmargin@length\relax%
     \advance\mdfboundingboxwidth by \mdf@innerrightmargin@length\relax%
     \expandafter\psset\expandafter{\mdf@psset@local}
     \psset{linearc=\mdf@roundcorner@length,cornersize=absolute}%
     \psset{unit=1truecm}%
     \mdf@makebox@in[\mdfboundingboxwidth]{%
       \null%
       \begin{pspicture}(0,0)(\mdfboundingboxwidth,\mdfboundingboxheight)
          %Hintergrund
          \psline[style=mdfbackgroundstyle,%
                  linearc={\dimexpr\mdf@roundcorner@length%
                                  -\mdf@middlelinewidth@length\relax},]%%
              (! \ptTps{\mdf@middlelinewidth@length} 0.5 mul 
                 \ptTps{\mdfboundingboxheight} )%
              (! \ptTps{\mdf@middlelinewidth@length} 0.5 mul 
                 \ifbool{mdf@bottomline}% 
                        {\ptTps{\mdf@middlelinewidth@length} neg 1.5 mul neg}%
                        {\ptTps{\mdf@middlelinewidth@length} neg 0.5 mul neg} )%
              (! \ptTps{\mdfboundingboxwidth} 
                 \ptTps{\mdf@middlelinewidth@length} 0.5 mul neg add 
                 \ifbool{mdf@bottomline}% 
                        {\ptTps{\mdf@middlelinewidth@length} neg 1.5 mul neg}%
                        {\ptTps{\mdf@middlelinewidth@length} neg 0.5 mul neg} )%
              (! \ptTps{\mdfboundingboxwidth} 
                 \ptTps{\mdf@middlelinewidth@length} 0.5 mul neg add
                 \ptTps{\mdfboundingboxheight} )
           %%%%%%%%%%%%
          \ifboolexpr{ test {\mdf@test@ltrb} or test {\mdf@test@lrb}}{%
            \psline[style=mdflinestyle]%
              (! \ptTps{\mdf@middlelinewidth@length} 0.5 mul neg
                 \ptTps{\mdfboundingboxheight})%
              (! \ptTps{\mdf@middlelinewidth@length} 0.5 mul neg 
                 \ptTps{\mdf@middlelinewidth@length} 0.5 mul )%
              (! \ptTps{\mdfboundingboxwidth} 
                 \ptTps{\mdf@middlelinewidth@length} 0.5 mul add %X
                 \ptTps{\mdf@middlelinewidth@length} 0.5 mul )
              (! \ptTps{\mdfboundingboxwidth} 
                 \ptTps{\mdf@middlelinewidth@length} 0.5 mul add %X
                 \ptTps{\mdfboundingboxheight})
          }{}
          \ifboolexpr{ test {\mdf@test@ltb} or test {\mdf@test@lb}}{%
            \psline[style=mdflinestyle]%
              (! \ptTps{\mdf@middlelinewidth@length} 0.5 mul neg 
                 \ptTps{\mdfboundingboxheight})%
              (! \ptTps{\mdf@middlelinewidth@length} 0.5 mul neg 
                 \ptTps{\mdf@middlelinewidth@length} 0.5 mul )%
              (! \ptTps{\mdfboundingboxwidth} 
                 \ptTps{\mdf@middlelinewidth@length} 0.5 mul )
          }{}
          \ifboolexpr{ test {\mdf@test@trb} or test {\mdf@test@rb}}{%
            \psline[style=mdflinestyle]%
              (! 0 \ptTps{\mdf@middlelinewidth@length} 0.5 mul )%
              (! \ptTps{\mdfboundingboxwidth} 
                 \ptTps{\mdf@middlelinewidth@length} 0.5 mul add %X
                 \ptTps{\mdf@middlelinewidth@length} 0.5 mul )
              (! \ptTps{\mdfboundingboxwidth} 
                 \ptTps{\mdf@middlelinewidth@length} 0.5 mul add %X
                 \ptTps{\mdfboundingboxheight})
          }{}
          \ifboolexpr{ test {\mdf@test@single} or test {\mdf@test@ltr} or
                       test {\mdf@test@tr} or test {\mdf@test@lt}}{%
           \ifbool{mdf@bottomline}{%
            \psline[style=mdflinestyle]%
              (! 0 \ptTps{\mdf@middlelinewidth@length} 0.5 mul )%
              (! \ptTps{\mdfboundingboxwidth}  
                 \ptTps{\mdf@middlelinewidth@length} 0.5 mul )
           }{}%
           \ifbool{mdf@leftline}{%
            \psline[style=mdflinestyle]%
              (! \ptTps{\mdf@middlelinewidth@length} neg 0.5 mul 0)%
              (! \ptTps{\mdf@middlelinewidth@length} neg 0.5 mul 
                 \ptTps{\mdfboundingboxheight} )%
           }{}%
           \ifbool{mdf@rightline}{%
            \psline[style=mdflinestyle]%
              (! \ptTps{\mdfboundingboxwidth} 
                 \ptTps{\mdf@middlelinewidth@length} 0.5 mul add 0)%
              (! \ptTps{\mdfboundingboxwidth} 
                 \ptTps{\mdf@middlelinewidth@length} 0.5 mul add 
                 \ptTps{\mdfboundingboxheight} )%
           }{}%
          }{}%
          %Ausgabe
          \rput[b](!
                 \ptTps{\mdfboundingboxwidth} 0.5 mul % X-Koordinate
                 \ptTps{\mdf@innerbottommargin@length}
                 \ifbool{mdf@bottomline}{\ptTps{\mdf@middlelinewidth@length} add}{}
                ){\box\mdf@splitbox@one}
       \end{pspicture}%
    }%
   \mdf@makeboxalign@right%
  }%
 \fi
}%
%    \end{macrocode}
% \end{macro}
%
%    \begin{macrocode}
 \endinput
%eof
%    \end{macrocode}
% \iffalse
%</mdframepstricks>
% \fi
% \Finale
%
\endinput

